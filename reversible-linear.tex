    The simple model assumed both the autocatalytic and the branch reactions are irreversible.
    Here we assume the branch reaction is reversible, and let $Y$ denote its product.
    For simplicity, we further assume that $K_{\text{eq}}=1$, noting that this assumption can always be satisfied by measuring the concentration of $Y$ in units of $K_{\text{eq}}X$.
    We recall that the reversible Michaelis-Menten equation states that:
   \begin{equation*}
       f_b=\frac{V_{\max,b}(X-Y)}{K_X+X+\frac{K_X}{K_Y}Y}
   \end{equation*}

   We assume that a third reaction, $f_c$, irreversibly consumes $Y$.
   While assuming $f_c$ follows irreversible Michaelis-Menten kinetics is analytically tractable, the analysis is simpler, and as informative, under the assumption that $f_c=DY$ for some constant $D$.
   This simplification is equivalent to assuming $f_c$ follows Michaelis-Menten kinetics with $\frac{V_{\max,c}}{K_{M,c}} \approx D$, and $V_{\max,c} >> \max(V_{\max,a},V_{\max,b})$.

   We start by deriving the necessary conditions for steady state existence.
   Because at steady state $f_a=f_c$, it follows that:
    \begin{equation}
        \label{eq:branchrevy}
        \frac{V_{\max,a}X^*}{K_{M,a}+X^*}=DY^* \Rightarrow Y^*=\frac{\frac{V_{\max,a}}{D}X^*}{K_{M,a}+X^*}
    \end{equation}
    Furthermore, as at the steady state $f_a=f_b$, we get that:
    \begin{equation*}
        \frac{V_{\max,a}X^*}{K_{M,a}+X^*}=\frac{V_{\max,b}(X^*-Y^*)}{K_X+X^*+\frac{K_X}{K_Y}Y^*}
    \end{equation*}
    Substituting $Y^*$ from equation \ref{eq:branchrevy} gives:
    \begin{equation*}
        \frac{V_{\max,a}X^*}{K_{M,a}+X^*}=\frac{V_{\max,b}(X^*-\frac{\frac{V_{\max,a}}{D}X^*}{K_{M,a}+X^*})}{K_X+X^*+\frac{K_X}{K_Y}\frac{\frac{V_{\max,a}}{D}X^*}{K_{M,a}+X^*}}
    \end{equation*}
    Which is satisfied when $X^*=0$ (implying that $X^*=Y^*=0$ is a steady state), or when $X^*$ satisfied the quadratic equation:
    \begin{align*}
        0=&(X^*)^2+\frac{2K_{M,a}V_{\max,b}-(K_{M,a}+K_X)V_{\max,a}-\frac{K_XV_{\max,a}^2}{K_YD}-\frac{V_{\max,a}V_{\max,b}}{D}}{V_{\max,b}-V_{\max,a}}X^*+\\
        &\frac{K_{M,a}(V_{\max,b}K_{M,a}-V_{\max,a}K_X-\frac{V_{\max,a}V_{\max,b}}{D})}{V_{\max,b}-V_{\max,a}}
    \end{align*}

Albeit intimidating, this quadratic equation can be used to derive the conditions for existence of a positive steady state.
Only if both of the roots of this equation are negative, no positive steady state exists.
We recall that the two roots of a quadratic equation of the form $0=aX^2+bX+c$ are negative iff:
\begin{align*}
    \begin{dcases}
        b&=\frac{2K_{M,a}V_{\max,b}-(K_{M,a}+K_X)V_{\max,a}-\frac{K_XV_{\max,a}^2}{K_YD}-\frac{V_{\max,a}V_{\max,b}}{D}}{V_{\max,b}-V_{\max,a}}>0 \\
        c&=\frac{K_{M,a}(V_{\max,b}K_{M,a}-V_{\max,a}K_X-\frac{V_{\max,a}V_{\max,b}}{D})}{V_{\max,b}-V_{\max,a}}>0
    \end{dcases}
\end{align*}
As in the irreversible case, the sign of $V_{\max,b}-V_{\max,a}$ determines the required condition on the numerators.
We assume that $V_{\max,b}>V_{\max,a}$, noting that if $V_{\max,b}<V_{\max,a}$, a positive steady state cannot be globally stable because for $X$ such that $f_a(X)>V_{\max,b}$, the system will diverge regardless of the value of $Y$.

Under the assumption that $V_{\max,b}>V_{\max,a}$, the denominator of both $b$ and $c$ is positive, meaning a positive steady state exists only if the nominators of $b$ or $c$ (or both) are negative.
Thus, two options may arise.
\begin{itemize}
    \item If $K_{M,a}>V_{\max,a}/D$ (implying $D>V_{\max,a}/K_{M,a}$, qualitatively suggesting rapid removal of $Y$) then an upper bound on $V_{\max,b}$ exists, above which the two solutions are negative, implying no positive steady state exists.
    A sufficient condition for existence in this case is that $\frac{V_{\max,b}}{K_X}<\frac{V_{\max,a}}{K_{M,a}}$, ensuring that $c<0$.
    This condition is equivalent to the condition in the irreversible case.
    We further show below that for large enough $D$, the resulting steady state is stable.
    \item If $D<V_{\max,a}/K_{M,a}$, then for any $V_{\max,b}>V_{\max,a}$, $c<0$ implying a positive steady state exists.
    As we show below, in this case both when $V_{\max,b}\rightarrow V_{\max,a}$, and when $V_{\max,b}\rightarrow \infty$, the steady state is stable.
\end{itemize}

We now turn to analyze the stability of the steady state.
For a steady state to be stable, the eigenvalues of the Jacobian matrix must have negative real values.
In our system it holds that
\begin{align*}
    \dot{X} & =f_a-f_b \\
    \dot{Y} & =f_b-f_c
\end{align*}

We use the following notation:
\begin{align*}
  \alpha & =\frac{df_a}{dX}=\frac{V_{\max,a}K_{M,a}}{(K_{M,a}+X)^2} \\
  \beta_x & =\frac{\partial f_b}{\partial X}= \frac{V_{\max,b}(K_X+Y(1+\frac{K_X}{K_Y}))}{(K_X+X+\frac{K_XY}{K_Y})^2}
\\
  \beta_y & =\frac{\partial f_b}{\partial Y}= \frac{-V_{\max,b}(K_X+X(1+\frac{K_X}{K_Y}))}{(K_X+X+\frac{K_XY}{K_Y})^2} \\
  \frac{df_c}{dY}&=D
\end{align*}

We can use this notation to write the Jacobian matrix as:
   \begin{equation*}
        J=
        \begin{pmatrix}
            \alpha-\beta_x & -\beta_y \\
            \beta_x & \beta_y-D
        \end{pmatrix}
    \end{equation*}
which gives a characteristic polynomial of:
\begin{equation*}
    (\alpha-\beta_x-\lambda)(\beta_y-D-\lambda)+\beta_y\beta_x=0
\end{equation*}

 In order for the real values of the roots of the characteristic polynomial to be negative it must hold that $b>0$ and $c>0$, where $b$ and $c$ are now the coefficients of the quadratic equation $a\lambda^2+b\lambda+c=0$.
We therefore get that:
\begin{align*}
    \begin{dcases}
        b & =\beta_x-\alpha-\beta_y+D>0\\
        c & =(\alpha-\beta_x)(\beta_y-D)+\beta_y\beta_x=\beta_x D+\alpha\beta_y-\alpha D >0
    \end{dcases}
\end{align*}

We denote by $f^*$ the steady state flux in the system, such that $f^*=f_a=f_b=f_c$
We note that for MM kinetics and positive concentrations it holds that:
    \begin{align*}
        \alpha & >0 \\
        \beta_x & >0 \\
        -\beta_y & >\beta_x \\
        \beta_x+\beta_y & = -f^*\frac{1+\frac{K_X}{K_Y}}{K_X+X+\frac{K_XY}{K_Y}}\\
    \end{align*}

First we note that if $\alpha\geq\beta_x$ then the steady state cannot be stable as, looking at the value of $c$, we see that in such a case $(\beta_x -\alpha) D<0$ and since $\alpha\beta_y<0$, $c<0$ violating the stability conditions.
However, because we assume that $V_{\max,b}>V_{\max,a}$, then for $Y=Y^*$, at $X=0$, $f_b<f_a$, but for $X\rightarrow\infty$, $f_b\rightarrow V_{\max,b}$ and $f_a \rightarrow V_{\max,a}$, so that $f_b>f_a$.
It then follows that, because the two fluxes can only intersect once for positive $X$ and fixed $Y$, at the steady state point, where $f_a=f_b$, $\alpha<\beta_x$, so this condition is satisfied.
We note that this condition is sufficient to ensure that $b>0$.
We also note that as $\alpha<\beta_x$, a large enough value of $D$ exists at which the steady state is stable, concluding that if $D$ is large enough, then a stable steady state exists if:
\begin{align*}
    \begin{dcases}
        &V_{\max,b}>V_{\max,a} \\
        &\frac{V_{\max,b}}{K_X}<\frac{V_{\max,a}}{K_{M,a}}
    \end{dcases}
\end{align*}

If $D$ is small, such that $D<V_{\max,a}/K_{M,a}$, and $V_{\max,b}>V_{\max,a}$ (implying that $\alpha<\beta_x$), we need to check what other conditions are necessary in order to ensure that $\beta_x D+\alpha\beta_y-\alpha D >0$.
We look at the limit $V_{\max,b}\rightarrow\infty$.
At this limit, the quadratic equation for $X^*$ converges to:
\begin{equation*}
        0=(X^*)^2+(2K_{M,a}-\frac{V_{\max,a}}{D})X^*+K_{M,a}^2-\frac{V_{\max,a}K_{M,a}}{D}
\end{equation*}
For this equation, $c<0$, implying that one of the roots is negative and one is positive.
The positive root is:
\begin{equation*}
  X^*=\frac{V_{\max,a}}{D}-K_{M,a}
\end{equation*}
As this $X^*$ is finite, we get that when $V_{\max,b}\rightarrow\infty$, $Y^*$ also converges to $\frac{V_{\max,a}}{D}-K_{M,a}$.
At this limit, $\beta_x$ increases infinitely and $\beta_y$ decreases infinitely, but $\beta_x+\beta_y$ converges to:
\begin{equation*}
  -f^*\frac{(1+\frac{K_X}{K_Y})}{K_X+(\frac{V_{\max,a}}{D}-K_{M,a})(1+\frac{K_X}{K_Y})}
\end{equation*}
that is constant.
Therefore, rearranging c such that:
\begin{equation*}
c =(\beta_x+\beta_y) D - \beta_y(D-\alpha)-\alpha D >0
\end{equation*}
we note that as $V_{\max,b}$ increases, the dominant term becomes $-\beta_y(D-\alpha)>0$ ensuring that $c>0$ and therefore stability.

On the other hand, when $V_{\max,b}\rightarrow V_{\max,a}$, we note that because $f_c<V_{\max,a}$, $Y^*$ is bounded by $Y^*<\frac{V_{\max,a}}{D}$, but $X^*\rightarrow \infty$.
Thus, both $\alpha$ and $\beta_x$ diminish like $\frac{1}{X^*}^2$, and $\beta_y$ diminishes like $\frac{1}{X}$.
The dominant term in $c=\beta_x D+\beta_y\alpha-\alpha D$ therefore becomes $(\beta_x-\alpha) D>0$ so again stability is maintained.

Therefore, for small values of $D$, as long as $V_{\max,b}>V_{\max,a}$, a positive stable steady state exists both in the lower limit of $V_{\max,b}\rightarrow V_{\max,a}$, and in the upper limit of $V_{\max,b}\rightarrow\infty$.

Our conclusions are therefore as follows:
As in the irreversible case, $V_{\max,b}>V_{\max,a}$ is a necessary condition for existence of a globally stable steady state.
For large values of $D$, the reversible reaction is far from equilibrium, resulting in an additional condition, equivalent to the condition we obtained for the irreversible case, namely that $V_{\max,b}/K_X$ is upper bounded by a term that is larger than $V_{\max,a}/K_{M,a}$, but approaches it as $D$ increases.
This condition is sufficient for existence and stability of the steady state.
For small values of $D$, a steady state always exists (given that $V_{\max,b}>V_{\max,a}$).
We can show that this steady state is stable both when $V_{\max,b}\rightarrow \infty$, and when $V_{\max,b}\rightarrow V_{\max,a}$.
We therefore conclude that in this case, no further restrictions apply on $K_X$, $K_Y$, or $K_{M,a}$ but rather that a steady state can always be achieved at most by changing $V_{\max,b}$.

Qualitatively, the cases we analyze show that, on top of the required $V_{\max,b}>V_{\max,a}$ condition, the second condition is that either the slope of $f_c=D$ is smaller than $V_{\max,a}/K_{M,a}$, or that the maximal slope of $f_b$, $V_{\max,b}/K_X$, is smaller than $V_{\max,a}/K_{M,a}$.
