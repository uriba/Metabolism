\begin{figure}[h!]
    \iftoggle{elifesubmission} { }
    {
    \centering{
        \input{extra-cycles-figure1.tex}}
    }
\caption{
    \label{fig:extrasamps1}
    An autocatalytic cycle assimilating ribose-5-phosphate using the pentose phosphate pathway.
    This cycle contains a direct input reaction (rpi, dashed line) allowing the cycle to operate with broader sets of kinetic parameters than cycles missing this feature.
    A knockout strain where rpi is eliminated, does not grow under ribose despite having the theoretical ability to do so.
}
\end{figure}
\begin{figure}[h!]
    \iftoggle{elifesubmission} { }
    {
    \centering{
        \input{extra-cycles-figure2.tex}}
    }
\caption{
    \label{fig:extrasamps2}
    An autocatalytic cycle assimilating dhap while consuming gap using the fba reaction in the gluconeogenic direction.
    This cycle contains a direct input reaction (tpi, dashed line) allowing the cycle to operate with broader sets of kinetic parameters than cycles missing this feature.
    Accurding to fluxomics data this cycle does not operate in vivo as a more energy efficient alternative in growth under glycerol is to use the tpi reaction and proceed in the glycolitic direction in the lower part of glycolysis.
    A knockout strain where tpi reaction is eliminated, does not grow under glycerol despite having the theoretical ability to do so.
}
\end{figure}
