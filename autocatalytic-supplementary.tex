\begin{figure}[H]
    \iftoggle{elifesubmission} { }
    {
    \centering{
        \begin{tikzpicture}
  \colorlet{ppinit}{magenta}

  \pgfmathsetlength{\assimwidth}{1.5pt};

  \begin{scope}

  \node[metaboliteStyle] (g6p) {g6p};


  \node[metaboliteStyle,below=of g6p.center] (f6p) {f6p};
  \node[metaboliteStyle,below=of f6p] (fbp) {fbp};
  \node[metaboliteStyle,shape=coordinate,below=of fbp.center](fbamid) {};
  \node[metaboliteStyle,below left=of fbamid.center] (dhap) {dhap};
  \node[metaboliteStyle]at([xshift=1.4cm]dhap -|fbamid.center) (gap) {gap};
  \node[metaboliteStyle,right=of g6p] (6pgi) {6pgi};
  \node[metaboliteStyle,shape=coordinate,right=of f6p] (s7pspace) {};
  \node[metaboliteStyle,right=of s7pspace] (s7p) {s7p};
  \node[metaboliteStyle,] at (fbp.center -| s7p.center) (e4p) {e4p};
  \node[metaboliteStyle,right=of e4p] (xu5p) {xu5p};
  \node[metaboliteStyle,above right=of xu5p] (ru5p) {ru5p};
  \node[metaboliteStyle,right=of ru5p,yshift=0.2cm,xshift=-0.2cm] (co2) {\ce{CO2}};
  \node[metaboliteStyle,above=of ru5p.center] (6pgc) {6pgc};
  \node[metaboliteStyle,above=of xu5p,yshift=2.5cm,rectangle,draw=assimcol,rounded corners=2pt] (r5p) {r5p};
  \node[shape=coordinate] at ([shift={(-0.8cm,-5mm)}]s7p -|r5p) (midtkt1) {};
  \path[] (e4p) -- (gap) coordinate [pos=0.4] (midtkt2) {};
  \path[] (e4p) -- (s7pspace) coordinate [pos=0.5] (midtal) {};
  \draw[<-] (g6p) -- (f6p);
  \draw[<-] (f6p.south) -- (fbp.north);
  \draw [] (fbamid) [out=-90,in=45] to (dhap);
  \draw [] (fbamid) [out=-90,in=135] to (gap);
  \draw[->] (ru5p) -- (xu5p);
  \draw[-] (xu5p) [out=180,in=-90] to (midtkt1);
  \draw[->] (midtkt1) [out=90,in=0] to (s7p);
  \draw[assimcol,line width=\assimwidth] (r5p) [out=-110,in=90] to (midtkt1);
  \draw[->] (midtkt1) [out=270,in=0] to ([yshift=1mm]gap.east);
  \draw[] (e4p) [out=-60,in=0] to (midtkt2);
  \draw[->] (midtkt2) [out=180,in=90] to (gap);
  \draw[] (xu5p) [out=245,in=0] to (midtkt2);
  \draw[->] (midtkt2) [out=180,in=-30] to (f6p);
  \draw[->] (midtal) [out=90,in=0] to (f6p);
  \draw[] (s7p) [out=180,in=90] to (midtal);
  \draw[->] (midtal) [out=-90,in=180] to (e4p);
  \draw[] (gap) [out=55,in=-90] to (midtal);
  \draw [<-] (fbp) [out=-90,in=90] to (fbamid);
  \draw [<-] (dhap) -- (gap);
  \draw[->] (g6p) -- (6pgi);
  \draw[->] (6pgi) -- (6pgc);
  \draw[->] (6pgc) -- (ru5p);
  \draw[->] ([yshift=-1mm] 6pgc) [out=-90,in=180] to (co2);
  \draw[<-,dashed] (ru5p) [out=180,in=-45] to (r5p);
  
  \draw[opacity=0.2,fill=ppinit,rounded corners=\highlightrad,visible on=<3->,even odd rule] ([shift={(\highlightrad,\highlightrad)}] 6pgc.north east) -- ([shift={(\highlightrad,-\highlightrad)}]ru5p.south -| 6pgc.east) -- ([xshift=5mm]dhap.south  -| gap.east) -- node [midway] (ppshade) {} (dhap.south west) -- (dhap.north west) -- ([xshift=-\highlightrad]fbamid  -| g6p.west) -- ([shift={(-\highlightrad,\highlightrad)}] g6p.north west)--cycle;

  %\draw[very thick,dashed,ppinit,->,visible on=<4->] (ppshade) -- ++(0cm,-1.1cm); 
  \end{scope}

  %% r5p cycle
  \begin{scope} [shift={(-5cm,-2cm)},radius=2cm,visible on=<2->]
  \draw[lightgray,rounded corners=\highlightrad] (-2.4cm,-2.1cm) rectangle +(5.1,4.6);

  \node[anchor=north] at(0.3cm,-2.2cm) (glyreac) {{\fontfamily{cmss}\selectfont 4} xu5p + {\fontfamily{cmss}\selectfont 2} r5p $\rightarrow$ {\fontfamily{cmss}\selectfont 5} xu5p + {\fontfamily{cmss}\selectfont 5} \ce{CO2}};
    \pgfmathsetlength{\ptsierad}{\autocatalrad*0.5};
    \pgfmathsetlength{\ptsimrad}{\autocatalrad-0.5*\arcwidth};

    \assim{2*\arcwidth}{100}{-30}{\ptsimrad}{3/2}
    \arrowhead{2*\arcwidth}{-45}{\autocatalrad}{ppinit}

    \pgfmathsetlength{\ptsesrad}{\autocatalrad+1/4*\arcwidth};
    \pgfmathsetmacro{\startbranch}{-15}
    \colorgradarc{5/2*\arcwidth}{40}{\startbranch}{\ptsesrad}{autocatacyc}{ppinit}


    \draw[color=autocatacyc,line width=3*\arcwidth] (100:\autocatalrad) arc(100:40:\autocatalrad);

    \draw[color=ppinit,line width=2*\arcwidth] (\startbranch:\autocatalrad) arc(\startbranch:-45:\autocatalrad);

    \pgfmathsetlength{\ptsarcwidth}{\autocatalrad-\arcwidth/2};

    \shadedarc[2*\arcwidth]{-95}{-260}{\autocatalrad}{\ptsarcwidth}{autocatacyc}{ppinit}


    \shadedarc[\arcwidth/2]{40}{-15}{\ptsarcwidth-3/4*\arcwidth}{\ptsarcwidth-3/4*\arcwidth}{white}{autocatacyc}

    \begin{scope}[shift={(\startbranch:2*\autocatalrad+5/4*\arcwidth)}]
        \draw[color=ppinit,line width=\arcwidth/2] (\startbranch+180:\autocatalrad) arc (\startbranch+180:\startbranch+180+25:\autocatalrad);
        \revarrowhead{\arcwidth/2}{190}{\autocatalrad}{ppinit}
    \end{scope}

    \node at (-37:\autocatalrad+3.5*\arcwidth) (xu5p) {+xu5p};

    \node at(135:\autocatalrad+2.5*\arcwidth) (r5p) {{\fontfamily{cmss}\selectfont 2} r5p};

    \node at (-70:\autocatalrad+\arcwidth/3) (xu5p) {{\fontfamily{cmss}\selectfont 4} xu5p};
    \node[gray] at (-35:2*\arcwidth) (co2) {{\fontfamily{cmss}\selectfont 5} \ce{CO2}};
  \end{scope}
\end{tikzpicture}

}
    }
\caption{
    \label{fig:extrasamps1}
    An autocatalytic cycle assimilating ribose-5-phosphate using the pentose phosphate pathway.
    This cycle contains a direct input reaction (rpi, dashed line) allowing the cycle to operate with broader sets of kinetic parameters than cycles missing this feature.
    A knockout strain where rpi is eliminated, does not grow under ribose despite having the theoretical ability to do so.
}
\end{figure}
\begin{figure}[H]
    \iftoggle{elifesubmission} { }
    {
    \centering{
        \begin{tikzpicture}
  \colorlet{fbainit}{cyan}

  \pgfmathsetlength{\assimwidth}{1.5pt};

  \begin{scope}
  \node[metaboliteStyle] (g6p) {g6p};


  \node[metaboliteStyle,below=of g6p.center] (f6p) {f6p};
  \node[metaboliteStyle,below=of f6p] (fbp) {fbp};
  \node[metaboliteStyle,shape=coordinate,below=of fbp.center](fbamid) {};
  \node[metaboliteStyle,below left=of fbamid.center,rectangle,draw=assimcol,rounded corners=2pt] (dhap) {dhap};
  \node[metaboliteStyle]at([xshift=1.4cm]dhap -|fbamid.center) (gap) {gap};
  \node[metaboliteStyle,below=of gap.center] (bpg) {bpg};
  \node[metaboliteStyle,below=of bpg.center] (3pg) {3pg};
  \node[metaboliteStyle,below=of 3pg.center] (2pg) {2pg};
  \node[metaboliteStyle,below=of 2pg.center] (pep) {pep};
  \node[metaboliteStyle,below=of pep.center] (pyr) {pyr};
  \node[metaboliteStyle,right=of g6p] (6pgi) {6pgi};
  \node[metaboliteStyle,right=of 6pgi] (6pgc) {6pgc};
  \draw[<-] (g6p) -- (f6p);
  \draw[<-] (f6p.south) -- (fbp.north);
  \draw [assimcol,line width=\assimwidth] (fbamid) [out=-90,in=45] to (dhap);
  \draw [] (fbamid) [out=-90,in=135] to (gap);
  \draw[<-] (3pg) -- (2pg);
  \draw[<-] (2pg) -- (pep);
  \draw[<-] (pep) -- (pyr);
  \draw[<-] (gap) -- (bpg);
  \draw[<-] (bpg) -- (3pg);
  \draw [<-] (fbp) [out=-90,in=90] to (fbamid);
  \draw [dashed,->] (dhap) -- (gap);
  \draw[->] (g6p) -- (6pgi);
  \draw[->] (6pgi) -- (6pgc);
  \node[metaboliteStyle,below=of 6pgc] (kdg) {kdg};
  \node[shape=coordinate,] at (fbamid -| kdg.center) (eddtop) {};
  \node[shape=coordinate,] at (2pg.center -| kdg.center) (eddbottom) {};
  \draw[->] (6pgc) -- (kdg);
  \draw[] (kdg) [out=-90,in=90] to (eddtop);
  \draw[->] (eddtop) [out=-90,in=0] to (gap);
  \draw[] (eddtop) [out=-90,in=90] to (eddbottom);
  \draw[->] (eddbottom) [out=-90,in=0] to (pyr);
  
  \draw[opacity=0.2,fill=fbainit,rounded corners=\highlightrad,visible on=<3->] ([shift={(-\highlightrad,\highlightrad)}]g6p.north west) -- ([shift={(\highlightrad,\highlightrad)}] 6pgc.north east) -- ([shift={(\highlightrad,\highlightrad)}]kdg.north east)-- ([shift={(\highlightrad,-\highlightrad)}]2pg.south -| kdg.east) -- ([shift={(\highlightrad,-\highlightrad)}]pyr.south -| kdg.east) -- ([shift={(-\highlightrad,-\highlightrad)}] pyr.south west) -- node [midway] (fbashade) {} ([xshift=-\highlightrad] gap.west) -- ([shift={(-\highlightrad,-\highlightrad)}] fbamid.south -| g6p.west) -- cycle;

%  \draw[very thick,dashed,fbainit,->,visible on=<4->] (fbashade) -- ++(-1.6cm,0cm); 
  \end{scope}

  %% FBA cycle
  \begin{scope} [shift={(-3.5cm,-9cm)},radius=2cm,visible on=<2->]
  \draw[lightgray,rounded corners=\highlightrad] (-2.4cm,-2.1cm) rectangle +(5.1,4.6);

  \node[anchor=north] at(0cm,-2.2cm) (glyreac) {gap + dhap $\rightarrow$ {\fontfamily{cmss}\selectfont 2} gap};

    \preassim{\arcwidth}{-90}{-270}{\autocatalrad}{fbainit}
    \postassim{\arcwidth}{90}{-45}{\autocatalrad}{fbainit}{2}
    \assim{\arcwidth}{90}{-30}{\autocatalrad}{2}
    \arrowhead{\arcwidth}{-45}{\autocatalrad}{fbainit}

    \node at (-37:\autocatalrad+3.5*\arcwidth) (gapp) {+gap};

    \node at(127:\autocatalrad+2.5*\arcwidth) (dhap) {dhap};

    \node at (-70:\autocatalrad) (gap) {gap};
  \end{scope}

\end{tikzpicture}

}
    }
\caption{
    \label{fig:extrasamps2}
    An autocatalytic cycle assimilating dhap while consuming gap using the fba reaction in the gluconeogenic direction.
    This cycle contains a direct input reaction (tpi, dashed line) allowing the cycle to operate with broader sets of kinetic parameters than cycles missing this feature.
    Accurding to fluxomics data this cycle does not operate in vivo as a more energy efficient alternative in growth under glycerol is to use the tpi reaction and proceed in the glycolitic direction in the lower part of glycolysis.
    A knockout strain where tpi reaction is eliminated, does not grow under glycerol despite having the theoretical ability to do so.
}
\end{figure}
\iftoggle{elifesubmission} {
    \subsubsection{Supplementary File 1}
    Supplementary File 1 contains the tables used in the data analysis in this work.
    The ``contents'' sheet includes the description of the different tables and is provided here as well:
    \begin{itemize}
        \item Fluxes source: The metabolic fluxes sheet from Data S1 in \cite{Gerosa2015-oq}.
        \item Cell size source: The cell sizes used for calculations as taken from Supplementary tables, Table ``Content and abbrevations'' in \cite{Schmidt2015}.
        \item Protein abundance: Protein abundance data from Supplementary tables, Table S6 in \cite{Schmidt2015}.
        \item Reaction-Protein mapping: Mapping between reactions and genes of catalyzing enzymes.
        \item Flux per enzyme: Calculation of the flux per enzyme for all the reactions listed in the ``Reaction-Protein mapping'' table.
        \item Reaction Saturation: Estimated saturation of enzymes across conditions.
        \item Non autocatalytic cycles reaction saturation: Comparison of saturation levels of branch versus cycle reactions for non-autocatalytic cycles.
        \item Allosteric regulation: Listing allosteric interactions between autocatalytic components.
    \end{itemize}
}
{
%    \begin{figure}[H]
%    \centering{
%\newlength\cyclerad
\pgfmathsetlength{\cyclerad}{2cm}
\begin{tikzpicture}
  \node[] at (0:\cyclerad) (atp) {atp};
  \node[gray] at (350:1.5*\cyclerad) (gluc) {gluc};
  \node[gray] at (330:1.5*\cyclerad) (adp1) {adp};
  \node[] at (320:\cyclerad) (g6p) {g6p};
  \node[] at (280:\cyclerad) (f6p) {f6p};
  \node[] at (235:\cyclerad) (fbp) {fbp};
  \node[gray] at (250:1.5*\cyclerad) (adp2) {adp};
  \node[] at (200:1.5*\cyclerad) (dhap) {dhap};
  \node[] at (180:\cyclerad) (gap) {gap};
  \node[gray] at (170:1.5*\cyclerad) (pi) {p};
  \node[] at (135:\cyclerad) (bpg) {bpg};
  \node[gray] at (120:1.5*\cyclerad) (adp3) {adp};
  \node[] at (100:\cyclerad) (3pg) {3pg};
  \node[] at (65:\cyclerad) (2pg) {2pg};
  \node[] at (30:\cyclerad) (pep) {pep};
  \node[gray] at (25:1.5*\cyclerad) (adp4) {adp};
  \node[gray] at (7:1.5*\cyclerad) (pyr) {pyr};
  \draw[->] (atp) [out=-135,in=0] to node [pos=0.9,shape=coordinate] (midpfk) {} (fbp);
  \draw[] (f6p) [out=140,in=0] to (midpfk);
  \draw[->] (midpfk) [out=190,in=90] to (adp2);
  \draw[->] (g6p) [out=220,in=20] to (f6p);
  \draw[->] (atp) [out=-90,in=60] to node [pos=0.5,shape=coordinate] (midpts) {} (g6p);
  \draw[] (gluc) [out=180,in=60] to (midpts);
  \draw[->] (midpts) [out=240] to (adp1);
  \draw[->] (fbp) [out=140,in=-85]to node [pos=0.3,shape=coordinate] (midfba) {} (gap) ;
  \draw[->] (midfba) [out=135,in=0] to (dhap);
  \draw[->] (dhap) -- (gap);
  \draw[->] (gap) [out=85,in=235] to node [pos=0.5,shape=coordinate] (mid3pg) {} (bpg) ;
  \draw[] (pi) [out=0,in=240] to (mid3pg);
  \draw[->] (bpg) [out=0,in=150] to node [pos=0.07,shape=coordinate] (midbpg) {} (atp);
  \draw[] (adp3) [out=-80,in=170] to (midbpg);
  \draw[->] (midbpg) [out=10,in=250] to (3pg);
  \draw[->] (3pg)  [out=0,in=166] to (2pg);
  \draw[->] (2pg) [out=-38,in=135] to (pep);
  \draw[->] (pep) [out=-59,in=90] to node [pos=0.5,shape=coordinate] (midpck) {}(atp);
  \draw[] (adp4) [out=210,in=120] to (midpck);
  \draw[->] (midpck) [out=300,in=180] to (pyr);
 \end{tikzpicture}

%}
%\caption{
% ATP autocatalysis in glycolysis.
% This autocatalytic cycle considers high energy phosphate bonds as the moiety that is required for the production of larger quantities of high energy phosphate bonds.
% Cycle intermediates are black, external metabolites are gray.
% The high energy bonds of ADP are neglected.
% The total autocatalytic reaction is: $2\text{ATP}+\text{gluc}+2\text{P}_i+2\text{ADP}\rightarrow 4\text{ATP}+2\text{pyr}$ (neglecting the double nad $\rightarrow$ nadh reaction coupled to the gap $\rightarrow$ bpg reaction).
% }
%\end{figure}
%\subsection{Analysis of a simple cycle with reversible branch reaction}
We analyze the simple autocatalytic cycle, with a reversible branch reaction, presented in figure \ref{fig:reversible}.
\begin{figure}[!htb]
  \centering
\begin{tikzpicture}[>=latex']
    \begin{scope}%[node distance = 2cm]
        \node at (-60:1cm) (X) {$X$};
        \node[shape=coordinate] (orig) {};
        \draw [-,line width=1pt,autocatacyc] (X.south west) arc (285:0:1cm) node [pos=0.65,above] (fa) {$f_a:$\small{$A+X\rightarrow2X$}} node [pos=0.45,shape=coordinate] (midauto) {} node [pos=1,shape=coordinate] (endcommon) {};
        \draw [->,line width=1pt,autocatacyc] (endcommon) arc (-25:-44:2cm);
        \draw [->,line width=1pt,autocatacyc] (endcommon) arc (-5:-32:1.5cm);
        \draw [line width=1pt,assimcol] (midauto) arc (-60:-90:1cm) node [pos=1,left] (e) {$A$};
        \draw [<->,line width=1pt,magenta] (X.south east) arc (225:270:1cm) node [pos=0.6,above] {$f_r$} node [pos=1,right,black] (Y) {$Y$};
        \node[right=of Y, shape=coordinate] (out) {};
        \draw [->,line width=1pt,branchout] (Y) -- node [pos=0.5,above] {$f_b$}(out);
%        \draw [->,line width=1pt,autocatacyc] (X.south) [out=-90,in=90] to  node [pos=0.5,above] (fa) {$f_a$} (X.north);
        %\draw [->,line width=1pt,autocatacyc] (X2.north) [out=90,in=90] to node [pos=0.5,shape=coordinate,yshift=0.5pt] (assimpt) {} node [pos=0.6,above] (fa2) {$f_{a_2}$} (X1.north);
        %\draw [line width=1pt,assimcol] (assimpt) arc (-90:-140:0.7cm) node [pos=1,above] (e) {$A$};
        %\draw [->,line width=1pt,branchout] (X1.south) arc (190:270:1cm) node [pos=0.3,right,xshift=1mm] {$f_{b_1}$};
        %\draw [->,line width=1pt,branchout] (X2.north) arc (10:90:1cm) node [pos=0.3,left] {$f_{b_2}$};
    \end{scope}
\end{tikzpicture}
     \caption{
         A simple autocatalytic cycle with a reversible branch reaction
 }

     \label{fig:reversible}
\end{figure}

\subsubsection{Steady state existance conditions}
We assume that $K_{\text{eq}}=1$.
As at a positive steady state, $f_r>0$, it follows that $X>Y$.
Because at steady state it also holds that $f_a$=$f_b$, it follows that $f_b$ attains the same flux as $f_a$ under a lower concentration of its substrate.
It must therefore hold that $V_{\max,a}/K_{M,a}<V_{\max,b}/K_{M,b}$.

The reversible Michaelis Menten equation (assuming $K_{\text{eq}}=1$)  states that:

\begin{equation*}
  f(X,Y)=\frac{\frac{V_{\max}}{K_{M,X}}(X-Y)}{1+\frac{X}{K_{M,X}}+\frac{Y}{K_{M,Y}}} = \frac{V_{\max}(X-Y)}{K_{M,X}+X+\frac{K_{M,X}Y}{K_{M,Y}}} 
\end{equation*}

In this case, if the branch reaction is reversible, then for every concentration of the intermediate metabolite $X$, $Y$ must allow the reversible flux to balance the autocatalytic flux:
\begin{equation*}
  f_a(X)=f(X,Y)\Rightarrow \frac{V_{\max,a}X}{K_{M,a}+X}=\frac{V_{\max}(X-Y)}{K_{M,X}+X+\frac{K_{M,X}Y}{K_{M,Y}}}
\end{equation*}

which implies:

\begin{equation*}
  -\frac{V_{\max,a}X}{K_{M,a}+X}(K_{M,X}+X)+V_{\max}X=V_{\max}Y+\frac{V_{\max,a}X}{K_{M,a}+X}\frac{K_{M,X}Y}{K_{M,Y}}
\end{equation*}
That results in:
\begin{equation*}
  \frac{V_{\max}X-\frac{V_{\max,a}X}{K_{M,a}+X}(K_{M,X}+X)}{V_{\max}+\frac{K_{M,X}V_{\max,a}X}{K_{M,Y}(K_{M,a}+X)}}=Y
\end{equation*}

That is equal to:
\begin{equation*}
  \frac{V_{\max}X(K_{M,a}+X)-V_{\max,a}X(K_{M,X}+X)}{V_{\max}(K_{M,a}+X)+\frac{K_{M,X}V_{\max,a}X}{K_{M,Y}}}=Y
\end{equation*}
As $Y>0$ at the steady state, it follows that:
\begin{equation*}
  V_{\max}(K_{M,a}+X)-V_{\max,a}(K_{M,X}+X)>0 \Rightarrow \frac{V_{\max}}{K_{M,X}+X}>\frac{V_{\max,a}}{K_{M,a}+X}
\end{equation*}
which implies that at least one of the two inequations must hold for a solution to exist:
\begin{align*}
  V_{\max} &> V_{\max,a} \\
  \frac{V_{\max}}{K_{M,X}} &>\frac{V_{\max,a}}{K_{M,a}}
\end{align*}


For a concentration to represent a steady state it must also hold that $f_b(Y)=f_a(X)$, so:
\begin{equation*}
    \frac{V_{\max,a}X}{K_{M,a}+X}=\frac{V_{\max,b}Y}{K_{M,b}+Y}
\end{equation*}
which results in:
\begin{equation*}
    K_{M,b}\frac{V_{\max,a}X}{K_{M,a}+X}=Y(V_{\max,b}-\frac{V_{\max,a}X}{K_{M,a}+X})
\end{equation*}
That can be simplified to:
\begin{equation*}
    \frac{K_{M,b}V_{\max,a}X}{V_{\max,b}(K_{M,a}+X)-V_{\max,a}X}=Y
\end{equation*}
As here, again, $Y>0$, it follows that:
\begin{equation*}
  V_{\max,b}>\frac{V_{\max,a}X}{K_{M,a}+X}
\end{equation*}


Combining the two conditions yields that at steady state $X$ satisfies:
\begin{equation*}
    \frac{K_{M,b}V_{\max,a}X}{V_{\max,b}(K_{M,a}+X)-V_{\max,a}X}=\frac{XV_{\max}(K_{M,a}+X)-XV_{\max,a}(K_{M,X}+X)}{V_{\max}(K_{M,a}+X)+\frac{K_{M,X}V_{\max,a}X}{K_{M,Y}}}
\end{equation*}

So $X=0$ is a steady state and for the other solution we get:
\begin{equation*}
    \frac{K_{M,b}V_{\max,a}}{V_{\max,b}K_{M,a}+(V_{\max,b}-V_{\max,a})X}=\frac{V_{\max}K_{M,a}-V_{\max,a}K_{M,X}+(V_{\max}-V_{\max,a})X}{V_{\max}K_{M,a}+(V_{\max}+\frac{K_{M,X}V_{\max,a}}{K_{M,Y}})X}
\end{equation*}

Where the solutions to this equation must yield $X>0$ and $Y>0$.

Simplifying give:
\begin{equation*}
  K_{M,b}V_{\max,a}
  V_{\max}K_{M,a}+
  K_{M,b}V_{\max,a}
  (V_{\max}+\frac{K_{M,X}V_{\max,a}}{K_{M,Y}})X=
  (V_{\max,b}K_{M,a}+(V_{\max,b}-V_{\max,a})X)
  (V_{\max}K_{M,a}-V_{\max,a}K_{M,X}+(V_{\max}-V_{\max,a})X)
\end{equation*}

\subsubsection{Stability analysis of a steady state}
For a steady state to be stable, the eigenvalues of the Jacobian matrix must have negative real values.
In our system it holds that:
    \begin{align*}
        \dot{X} & =f_a-f_r \\
        \dot{Y} & =f_r-f_b
    \end{align*}
We use the following notation:
\begin{align*}
  \alpha & =\frac{df_a}{dX}=\frac{V_{\max,a}K_{M,a}}{(K_{M,a}+X)^2} \\
  \beta & =\frac{df_b}{dY}=\frac{V_{\max,b}K_{M,b}}{(K_{M,b}+X)^2} \\
  \gamma_x & =\frac{\partial f_r}{\partial X}= \frac{V_{\max}(K_{M,X}+Y(1+\frac{K_{M,X}}{K_{M,Y}}))}{(K_{M,X}+X+\frac{K_{M,X}Y}{K_{M,Y}})^2}
\\
  \gamma_y & =\frac{\partial f_r}{\partial Y}= \frac{-V_{\max}(K_{M,X}+X(1+\frac{K_{M,X}}{K_{M,Y}}))}{(K_{M,X}+X+\frac{K_{M,X}Y}{K_{M,Y}})^2} 
\end{align*}

We note that for MM kinetics and positive concentrations it holds that:
    \begin{align*}
        \alpha & >0 \\
        \beta & >0 \\
        \gamma_x & >0 \\
        -\gamma_y & >\gamma_x \\
    \end{align*}

We can use this notation to write the Jacobian matrix as:
   \begin{equation*}
        J=
        \begin{pmatrix}
            \alpha-\gamma_x & -\gamma_y \\
            \gamma_x & \gamma_y-\beta
        \end{pmatrix}
    \end{equation*}
which gives a characteristic polynomial of:
\begin{equation*}
    (\alpha-\gamma_x-\lambda)(\gamma_y-\beta-\lambda)+\gamma_y\gamma_x=0
\end{equation*}
In order for the real values of the roots of the characteristic polynomial to be negative it must hold that $b>0$ and $c>0$, where $b$ and $c$ are the coefficients of the quadratic equation $a\lambda^2+b\lambda+c=0$.
We therefore get that:
\begin{align*}
b & =\gamma_x-\alpha-\gamma_y+\beta>0\\
c & =(\alpha-\gamma_x)(\gamma_y-\beta)+\gamma_y\gamma_x>0
\end{align*}
We note that as $\gamma_y\gamma_x<0$ and $\gamma_y-\beta<0$, it is a necessary, but not sufficient, condition that $\alpha-\gamma_x<0$ for the second inequality to hold.
This is also a sufficient condition for the first inequality to hold.
The sufficient condition for the second inequality to hold is that:
\begin{equation*}
  \gamma_x-\alpha>\gamma_y\gamma_x/(\gamma_y-\beta)=\frac{\gamma_x}{1-\frac{\beta}{\gamma_y}}\Rightarrow (1-\frac{\alpha}{\gamma_x})(1-\frac{\beta}{\gamma_y})>1
\end{equation*}
which implies that the smaller $\beta$ is, the larger the ratio between $\gamma_x$ and $\alpha$ needs to be in order to ensure stability.
Moreover, as $-\gamma_y >\gamma_x$ it follows that $\beta>\alpha$.
We can use the last inequality to get that:
\begin{equation*}
    \beta>\alpha \Rightarrow \frac{V_{\max,b}K_{M,b}}{(K_{M,b}+X)^2}>\frac{V_{\max,a}K_{M,a}}{(K_{M,a}+X)^2} \Rightarrow \frac{V_{\max,b}XK_{M,b}}{(K_{M,b}+X)^2}>\frac{V_{\max,a}XK_{M,a}}{(K_{M,a}+X)^2} \Rightarrow f_b(X)\frac{K_{M,b}}{K_{M,b}+X}>f_a(X)\frac{K_{M,a}}{K_{M,a}+X}
\end{equation*}
But, since $f_a(X)=f_b(X)$ at the steady state point we can simplify and get:
\begin{equation*}
    \frac{K_{M,b}}{K_{M,b}+X}>\frac{K_{M,a}}{K_{M,a}+X} \Rightarrow \frac{1}{1+X/K_{M,b}}>\frac{1}{1+X/K_{M,a}} \Rightarrow K_{M,b}>K_{M,a}
\end{equation*}
From the fact that $\alpha<\gamma_x$ we can similarly get that:
\begin{equation*}
\alpha<\gamma_x \Rightarrow
\frac{V_{\max,a}K_{M,a}}{(K_{M,a}+X)^2} < \frac{V_{\max}(K_{M,X}+Y(1+\frac{K_{M,X}}{K_{M,Y}}))}{(K_{M,X}+X+\frac{K_{M,X}Y}{K_{M,Y}})^2}
\Rightarrow \frac{V_{\max,a}XK_{M,a}}{(K_{M,a}+X)^2}<\frac{V_{\max}X(K_{M,X}+Y(1+\frac{K_{M,X}}{K_{M,Y}}))}{(K_{M,X}+X+\frac{K_{M,X}Y}{K_{M,Y}})^2}
\Rightarrow f_r(X)\frac{K_{M,a}}{K_{M,a}+X}<\frac{V_{\max}X(K_{M,X}+Y(1+\frac{K_{M,X}}{K_{M,Y}}))}{(K_{M,X}+X+\frac{K_{M,X}Y}{K_{M,Y}})^2}
\end{equation*}
So we get:
\begin{equation*}
\frac{V_{\max}(X-Y)}{K_{M,X}+X+\frac{K_{M,X}Y}{K_{M,Y}}}\frac{K_{M,a}}{K_{M,a}+X}<\frac{V_{\max}X(K_{M,X}+Y(1+\frac{K_{M,X}}{K_{M,Y}}))}{(K_{M,X}+X+\frac{K_{M,X}Y}{K_{M,Y}})^2}
\end{equation*}
that can be simplified to:
\begin{equation*}
\frac{(X-Y)}{1}\frac{K_{M,a}}{K_{M,a}+X}<\frac{X(K_{M,X}+Y(1+\frac{K_{M,X}}{K_{M,Y}}))}{K_{M,X}+X+\frac{K_{M,X}Y}{K_{M,Y}}}
\end{equation*}
And further to:
\begin{equation*}
    \frac{(X-Y)}{1}\frac{K_{M,a}}{V_{\max,a}X}<\frac{X(K_{M,X}+Y(1+\frac{K_{M,X}}{K_{M,Y}}))}{V_{\max}(X-Y)}
\end{equation*}
\subsubsection{Conclusion}
In this case for a positive steady state to exist it must hold that:
\begin{align*}
    &V_{\max,a}/K_{M,a}<V_{\max,b}/K_{M,b} \quad \text{and}, \\
    &\begin{dcases}
        V_{\max} &> V_{\max,a} \quad \text{or}, \\
        \frac{V_{\max}}{K_{M,X}} &>\frac{V_{\max,a}}{K_{M,a}}
    \end{dcases}
\end{align*}

For the positive steady state to be stable, it must hold that:
\begin{align*}
(1-\frac{\alpha}{\gamma_x})(1-\frac{\beta}{\gamma_y})>1 \Rightarrow
    \begin{dcases}
        \alpha &< \gamma_x \\
        \alpha &< \beta \Rightarrow K_{M,b}>K_{M,a}
    \end{dcases}
\end{align*}

%\subsection{Connecting autocatalytic theoretical results to experimental observations in the evolution of hemiautotrophy \emph{E.coli}}
In this section we analyze the connection between the theoretical model and analyses in this chapter, and the experimental observations in \cite{Antonovsky2016-jy}.
We assume the kinetic parameters of the initial strain were inappropriate to allow stable autocatalytic flux through the CBB cycle.
We therefore deduce that the initial state was not case (I) in figure \ref{fig:simplecycle}.
As all of the evolved strains had specific mutations in the vicinity of the active site of the branch reaction, prs, we assume the required change in parameters was \emph{not} limited to a change in $V_{\max,b}$, as such a change could have been achieved by increasing the expression of prs via mutations in the promoter or ribosome binding site sequences.
We also note that the proteomics measurements taken did not reveal a significant change in the abundance of prs between the different evolved strains and the initial strain.
We therefore deduce that for the initial strain, $K_{M,b}/K_{M,a}<1$, forcing a mutation in the active site of prs to increase $K_{M,b}$.
Because such a mutation could have an arbitrary effect on $K_{\text{cat},b}$, we cannot deduce where in the phase space depicted in figure \ref{fig:simplecycle}B the initial strain lay, but can only assume it was in the left half of the phase domain, reflecting the $K_{M,b}/K_{M,a}<1$ assumption.

The experimental, in vitro reaction rate measurements show that $V_{\max,b}/K_{M,b}$ decreased in the evolved strain.
This result can be interpreted visually as implying that the initial strain lay on a 45\degree isoline higher than the 45\degree isoline the evolved strains lie on.
However, as concrete information on the initial or final $K_{M,b}$ or $V_{\max,b}$ is missing, one cannot determine where along the isoline the initial and evolved strains lie.
If one assumes that the initial strain belonged to cases (II) or (III), then such a change is essential, and therefore could have been predicted.
However, theoretically, the initial strain could have been in the left, triangular part of case (IV), in which case a change in $V_{\max,b}/K_{M,b}$ was not essential to achieve stability.
In this case, a change in $V_{\max,b}/K_{M,b}$ suggests that the mandatory increase needed in $K_{M,b}$ had a side effect increasing $V_{\max,b}$ in just the right amount to produce a stable flux phenotype.

}
