\begin{figure}[h!]
    \iftoggle{elifesubmission} { }
    {
    \centering{
        \begin{tikzpicture}
  \colorlet{ppinit}{magenta}

  \pgfmathsetlength{\assimwidth}{1.5pt};

  \begin{scope}

  \node[metaboliteStyle] (g6p) {g6p};


  \node[metaboliteStyle,below=of g6p.center] (f6p) {f6p};
  \node[metaboliteStyle,below=of f6p] (fbp) {fbp};
  \node[metaboliteStyle,shape=coordinate,below=of fbp.center](fbamid) {};
  \node[metaboliteStyle,below left=of fbamid.center] (dhap) {dhap};
  \node[metaboliteStyle]at([xshift=1.4cm]dhap -|fbamid.center) (gap) {gap};
  \node[metaboliteStyle,right=of g6p] (6pgi) {6pgi};
  \node[metaboliteStyle,shape=coordinate,right=of f6p] (s7pspace) {};
  \node[metaboliteStyle,right=of s7pspace] (s7p) {s7p};
  \node[metaboliteStyle,] at (fbp.center -| s7p.center) (e4p) {e4p};
  \node[metaboliteStyle,right=of e4p] (xu5p) {xu5p};
  \node[metaboliteStyle,above right=of xu5p] (ru5p) {ru5p};
  \node[metaboliteStyle,right=of ru5p,yshift=0.2cm,xshift=-0.2cm] (co2) {\ce{CO2}};
  \node[metaboliteStyle,above=of ru5p.center] (6pgc) {6pgc};
  \node[metaboliteStyle,above=of xu5p,yshift=2.5cm,rectangle,draw=assimcol,rounded corners=2pt] (r5p) {r5p};
  \node[shape=coordinate] at ([shift={(-0.8cm,-5mm)}]s7p -|r5p) (midtkt1) {};
  \path[] (e4p) -- (gap) coordinate [pos=0.4] (midtkt2) {};
  \path[] (e4p) -- (s7pspace) coordinate [pos=0.5] (midtal) {};
  \draw[<-] (g6p) -- (f6p);
  \draw[<-] (f6p.south) -- (fbp.north);
  \draw [] (fbamid) [out=-90,in=45] to (dhap);
  \draw [] (fbamid) [out=-90,in=135] to (gap);
  \draw[->] (ru5p) -- (xu5p);
  \draw[-] (xu5p) [out=180,in=-90] to (midtkt1);
  \draw[->] (midtkt1) [out=90,in=0] to (s7p);
  \draw[assimcol,line width=\assimwidth] (r5p) [out=-110,in=90] to (midtkt1);
  \draw[->] (midtkt1) [out=270,in=0] to ([yshift=1mm]gap.east);
  \draw[] (e4p) [out=-60,in=0] to (midtkt2);
  \draw[->] (midtkt2) [out=180,in=90] to (gap);
  \draw[] (xu5p) [out=245,in=0] to (midtkt2);
  \draw[->] (midtkt2) [out=180,in=-30] to (f6p);
  \draw[->] (midtal) [out=90,in=0] to (f6p);
  \draw[] (s7p) [out=180,in=90] to (midtal);
  \draw[->] (midtal) [out=-90,in=180] to (e4p);
  \draw[] (gap) [out=55,in=-90] to (midtal);
  \draw [<-] (fbp) [out=-90,in=90] to (fbamid);
  \draw [<-] (dhap) -- (gap);
  \draw[->] (g6p) -- (6pgi);
  \draw[->] (6pgi) -- (6pgc);
  \draw[->] (6pgc) -- (ru5p);
  \draw[->] ([yshift=-1mm] 6pgc) [out=-90,in=180] to (co2);
  \draw[<-,dashed] (ru5p) [out=180,in=-45] to (r5p);
  
  \draw[opacity=0.2,fill=ppinit,rounded corners=\highlightrad,visible on=<3->,even odd rule] ([shift={(\highlightrad,\highlightrad)}] 6pgc.north east) -- ([shift={(\highlightrad,-\highlightrad)}]ru5p.south -| 6pgc.east) -- ([xshift=5mm]dhap.south  -| gap.east) -- node [midway] (ppshade) {} (dhap.south west) -- (dhap.north west) -- ([xshift=-\highlightrad]fbamid  -| g6p.west) -- ([shift={(-\highlightrad,\highlightrad)}] g6p.north west)--cycle;

  %\draw[very thick,dashed,ppinit,->,visible on=<4->] (ppshade) -- ++(0cm,-1.1cm); 
  \end{scope}

  %% r5p cycle
  \begin{scope} [shift={(-5cm,-2cm)},radius=2cm,visible on=<2->]
  \draw[lightgray,rounded corners=\highlightrad] (-2.4cm,-2.1cm) rectangle +(5.1,4.6);

  \node[anchor=north] at(0.3cm,-2.2cm) (glyreac) {{\fontfamily{cmss}\selectfont 4} xu5p + {\fontfamily{cmss}\selectfont 2} r5p $\rightarrow$ {\fontfamily{cmss}\selectfont 5} xu5p + {\fontfamily{cmss}\selectfont 5} \ce{CO2}};
    \pgfmathsetlength{\ptsierad}{\autocatalrad*0.5};
    \pgfmathsetlength{\ptsimrad}{\autocatalrad-0.5*\arcwidth};

    \assim{2*\arcwidth}{100}{-30}{\ptsimrad}{3/2}
    \arrowhead{2*\arcwidth}{-45}{\autocatalrad}{ppinit}

    \pgfmathsetlength{\ptsesrad}{\autocatalrad+1/4*\arcwidth};
    \pgfmathsetmacro{\startbranch}{-15}
    \colorgradarc{5/2*\arcwidth}{40}{\startbranch}{\ptsesrad}{autocatacyc}{ppinit}


    \draw[color=autocatacyc,line width=3*\arcwidth] (100:\autocatalrad) arc(100:40:\autocatalrad);

    \draw[color=ppinit,line width=2*\arcwidth] (\startbranch:\autocatalrad) arc(\startbranch:-45:\autocatalrad);

    \pgfmathsetlength{\ptsarcwidth}{\autocatalrad-\arcwidth/2};

    \shadedarc[2*\arcwidth]{-95}{-260}{\autocatalrad}{\ptsarcwidth}{autocatacyc}{ppinit}


    \shadedarc[\arcwidth/2]{40}{-15}{\ptsarcwidth-3/4*\arcwidth}{\ptsarcwidth-3/4*\arcwidth}{white}{autocatacyc}

    \begin{scope}[shift={(\startbranch:2*\autocatalrad+5/4*\arcwidth)}]
        \draw[color=ppinit,line width=\arcwidth/2] (\startbranch+180:\autocatalrad) arc (\startbranch+180:\startbranch+180+25:\autocatalrad);
        \revarrowhead{\arcwidth/2}{190}{\autocatalrad}{ppinit}
    \end{scope}

    \node at (-37:\autocatalrad+3.5*\arcwidth) (xu5p) {+xu5p};

    \node at(135:\autocatalrad+2.5*\arcwidth) (r5p) {{\fontfamily{cmss}\selectfont 2} r5p};

    \node at (-70:\autocatalrad+\arcwidth/3) (xu5p) {{\fontfamily{cmss}\selectfont 4} xu5p};
    \node[gray] at (-35:2*\arcwidth) (co2) {{\fontfamily{cmss}\selectfont 5} \ce{CO2}};
  \end{scope}
\end{tikzpicture}

}
    }
\caption{
    \label{fig:extrasamps1}
    An autocatalytic cycle assimilating ribose-5-phosphate using the pentose phosphate pathway.
    This cycle contains a direct input reaction (rpi, dashed line) allowing the cycle to operate with broader sets of kinetic parameters than cycles missing this feature.
    A knockout strain where rpi is eliminated, does not grow under ribose despite having the theoretical ability to do so.
}
\end{figure}
\begin{figure}[h!]
    \iftoggle{elifesubmission} { }
    {
    \centering{
        \begin{tikzpicture}
  \colorlet{fbainit}{cyan}

  \pgfmathsetlength{\assimwidth}{1.5pt};

  \begin{scope}
  \node[metaboliteStyle] (g6p) {g6p};


  \node[metaboliteStyle,below=of g6p.center] (f6p) {f6p};
  \node[metaboliteStyle,below=of f6p] (fbp) {fbp};
  \node[metaboliteStyle,shape=coordinate,below=of fbp.center](fbamid) {};
  \node[metaboliteStyle,below left=of fbamid.center,rectangle,draw=assimcol,rounded corners=2pt] (dhap) {dhap};
  \node[metaboliteStyle]at([xshift=1.4cm]dhap -|fbamid.center) (gap) {gap};
  \node[metaboliteStyle,below=of gap.center] (bpg) {bpg};
  \node[metaboliteStyle,below=of bpg.center] (3pg) {3pg};
  \node[metaboliteStyle,below=of 3pg.center] (2pg) {2pg};
  \node[metaboliteStyle,below=of 2pg.center] (pep) {pep};
  \node[metaboliteStyle,below=of pep.center] (pyr) {pyr};
  \node[metaboliteStyle,right=of g6p] (6pgi) {6pgi};
  \node[metaboliteStyle,right=of 6pgi] (6pgc) {6pgc};
  \draw[<-] (g6p) -- (f6p);
  \draw[<-] (f6p.south) -- (fbp.north);
  \draw [assimcol,line width=\assimwidth] (fbamid) [out=-90,in=45] to (dhap);
  \draw [] (fbamid) [out=-90,in=135] to (gap);
  \draw[<-] (3pg) -- (2pg);
  \draw[<-] (2pg) -- (pep);
  \draw[<-] (pep) -- (pyr);
  \draw[<-] (gap) -- (bpg);
  \draw[<-] (bpg) -- (3pg);
  \draw [<-] (fbp) [out=-90,in=90] to (fbamid);
  \draw [dashed,->] (dhap) -- (gap);
  \draw[->] (g6p) -- (6pgi);
  \draw[->] (6pgi) -- (6pgc);
  \node[metaboliteStyle,below=of 6pgc] (kdg) {kdg};
  \node[shape=coordinate,] at (fbamid -| kdg.center) (eddtop) {};
  \node[shape=coordinate,] at (2pg.center -| kdg.center) (eddbottom) {};
  \draw[->] (6pgc) -- (kdg);
  \draw[] (kdg) [out=-90,in=90] to (eddtop);
  \draw[->] (eddtop) [out=-90,in=0] to (gap);
  \draw[] (eddtop) [out=-90,in=90] to (eddbottom);
  \draw[->] (eddbottom) [out=-90,in=0] to (pyr);
  
  \draw[opacity=0.2,fill=fbainit,rounded corners=\highlightrad,visible on=<3->] ([shift={(-\highlightrad,\highlightrad)}]g6p.north west) -- ([shift={(\highlightrad,\highlightrad)}] 6pgc.north east) -- ([shift={(\highlightrad,\highlightrad)}]kdg.north east)-- ([shift={(\highlightrad,-\highlightrad)}]2pg.south -| kdg.east) -- ([shift={(\highlightrad,-\highlightrad)}]pyr.south -| kdg.east) -- ([shift={(-\highlightrad,-\highlightrad)}] pyr.south west) -- node [midway] (fbashade) {} ([xshift=-\highlightrad] gap.west) -- ([shift={(-\highlightrad,-\highlightrad)}] fbamid.south -| g6p.west) -- cycle;

%  \draw[very thick,dashed,fbainit,->,visible on=<4->] (fbashade) -- ++(-1.6cm,0cm); 
  \end{scope}

  %% FBA cycle
  \begin{scope} [shift={(-3.5cm,-9cm)},radius=2cm,visible on=<2->]
  \draw[lightgray,rounded corners=\highlightrad] (-2.4cm,-2.1cm) rectangle +(5.1,4.6);

  \node[anchor=north] at(0cm,-2.2cm) (glyreac) {gap + dhap $\rightarrow$ {\fontfamily{cmss}\selectfont 2} gap};

    \preassim{\arcwidth}{-90}{-270}{\autocatalrad}{fbainit}
    \postassim{\arcwidth}{90}{-45}{\autocatalrad}{fbainit}{2}
    \assim{\arcwidth}{90}{-30}{\autocatalrad}{2}
    \arrowhead{\arcwidth}{-45}{\autocatalrad}{fbainit}

    \node at (-37:\autocatalrad+3.5*\arcwidth) (gapp) {+gap};

    \node at(127:\autocatalrad+2.5*\arcwidth) (dhap) {dhap};

    \node at (-70:\autocatalrad) (gap) {gap};
  \end{scope}

\end{tikzpicture}

}
    }
\caption{
    \label{fig:extrasamps2}
    An autocatalytic cycle assimilating dhap while consuming gap using the fba reaction in the gluconeogenic direction.
    This cycle contains a direct input reaction (tpi, dashed line) allowing the cycle to operate with broader sets of kinetic parameters than cycles missing this feature.
    Accurding to fluxomics data this cycle does not operate in vivo as a more energy efficient alternative in growth under glycerol is to use the tpi reaction and proceed in the glycolitic direction in the lower part of glycolysis.
    A knockout strain where tpi reaction is eliminated, does not grow under glycerol despite having the theoretical ability to do so.
}
\end{figure}
