\subsection{Connecting autocatalytic theoretical results to experimental observations in the evolution of hemiautotrophy \emph{E.coli}}
In this section we analyze the connection between the theoretical model and analyses in this chapter, and the experimental observations in \cite{Antonovsky2016-jy}.
We assume the kinetic parameters of the initial strain were inappropriate to allow stable autocatalytic flux through the CBB cycle.
We therefore deduce that the initial state was not case (I) in figure \ref{fig:simplecycle}.
As all of the evolved strains had specific mutations in the vicinity of the active site of the branch reaction, prs, we assume the required change in parameters was \emph{not} limited to a change in $V_{\max,b}$, as such a change could have been achieved by increasing the expression of prs via mutations in the promoter or ribosome binding site sequences.
We also note that the proteomics measurements taken did not reveal a significant change in the abundance of prs between the different evolved strains and the initial strain.
We therefore deduce that for the initial strain, $K_{M,b}/K_{M,a}<1$, forcing a mutation in the active site of prs to increase $K_{M,b}$.
Because such a mutation could have an arbitrary effect on $K_{\text{cat},b}$, we cannot deduce where in the phase space depicted in figure \ref{fig:simplecycle}B the initial strain lay, but can only assume it was in the left half of the phase domain, reflecting the $K_{M,b}/K_{M,a}<1$ assumption.

The experimental, in vitro reaction rate measurements show that $V_{\max,b}/K_{M,b}$ decreased in the evolved strain.
This result can be interpreted visually as implying that the initial strain lay on a 45\degree isoline higher than the 45\degree isoline the evolved strains lie on.
However, as concrete information on the initial or final $K_{M,b}$ or $V_{\max,b}$ is missing, one cannot determine where along the isoline the initial and evolved strains lie.
If one assumes that the initial strain belonged to cases (II) or (III), then such a change is essential, and therefore could have been predicted.
However, theoretically, the initial strain could have been in the left, triangular part of case (IV), in which case a change in $V_{\max,b}/K_{M,b}$ was not essential to achieve stability.
In this case, a change in $V_{\max,b}/K_{M,b}$ suggests that the mandatory increase needed in $K_{M,b}$ had a side effect increasing $V_{\max,b}$ in just the right amount to produce a stable flux phenotype.

