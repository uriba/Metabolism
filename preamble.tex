\usepackage{color,amsmath,amssymb,graphicx,subcaption,geometry,mathtools,xfrac}
%\usepackage{cite}
\usepackage{mhchem}
\usepackage{tikz}
\usepackage{pgfplots}
\usepackage{lineno}
\usepackage{setspace}
\pgfplotsset{compat=1.12}
\usepackage{stackengine,ifthen}
\usetikzlibrary{arrows,positioning,calc,arrows.meta,patterns,fit}

\newcommand{\beginsupplement}{%
        \setcounter{table}{0}
        \renewcommand{\thetable}{S\arabic{table}}%
        \setcounter{figure}{0}
        \renewcommand{\thefigure}{S\arabic{figure}}%
    }

\newtoggle{article}
\newtoggle{poster}
\newtoggle{eddpathway}
\newtoggle{elifesubmission}
\togglefalse{elifesubmission}

\iftoggle{elifesubmission} {
\linenumbers
\doublespacing
}
{}

\tikzset{>=Latex}
\newcommand\influx{0.5}

\newenvironment{customlegend}[1][]{
  \begingroup
  \csname pgfplots@init@cleared@structures\endcsname
  \pgfplotsset{#1}
}{
  \csname pgfplots@createlegend\endcsname
  \endgroup
}

\def\addlegendimage{\csname pgfplots@addlegendimage\endcsname}
\setlength\abovecaptionskip{6pt}
\providecommand{\abs}[1]{\lvert#1\rvert}
\providecommand{\norm}[1]{\lVert#1\rVert}

\tikzset{>=latex}
\tikzset{metaboliteStyle/.style={}}
\definecolor{cyan}{RGB}{100,181,205}
\definecolor{blue}{RGB}{76,114,176}
\definecolor{green}{RGB}{85,168,104}
\definecolor{magenta}{RGB}{129,114,178}
\definecolor{yellow}{RGB}{204,185,116}
\definecolor{red}{RGB}{196,78,82}
\definecolor{graybg}{gray}{0.95}

\colorlet{assimcol}{green}
\colorlet{sumcolor}{yellow}

\colorlet{inputcol}{green}
\colorlet{branchout}{red}
\colorlet{branchoutfl}{red!80}
\colorlet{autocatacyc}{blue}
\colorlet{autocatacycfl}{blue!80}
\colorlet{autocataby}{cyan}

\pdfpageattr{/Group <</S /Transparency /I true /CS /DeviceRGB>>} 

\def\blendfrac{0.5}
\def\deltaang{-155}
\def\fromang{180}
\def\inputang{-40}
\def\protrude{7}
\def\arcwidth{0.3cm}
\def\highlightrad{0.2cm}
\def\autocatalrad{1.5cm}
\def\autocatalscale{1.5}

  \newcommand{\colorgradarc}[6]{%width,startang,stopang,rad,startcol,stopcol
    \pgfmathsetmacro\arcrange{#3-#2}
    \pgfmathsetmacro\progsign{\arcrange>0 ? 1 : -1}
    \pgfmathsetmacro\arcend{#3-1}
    \foreach \i in {#2,...,\arcend} {
     \pgfmathsetmacro\fracprog{\i/\arcrange-#2/\arcrange}
     \pgfmathsetmacro\col{\fracprog*100}
     \draw[color={#6!\col!#5},line width=#1] (\i-\progsign:#4)
         arc[start angle=\i-\progsign, end angle=\i+1.1*\progsign,radius=#4];
    }
  }

  \newcommand{\shadedarc}[7][\arcwidth]{%width,startang,stopang,startrad,stoprad,startcol,stopcol
    \pgfmathsetmacro\arcrange{#3-#2}
    \pgfmathsetmacro\radrange{#5-#4}
    \pgfmathsetmacro\progsign{\arcrange>0 ? 1 : -1}
    \foreach \i in {#2,...,\numexpr#3-1\relax} {
      \pgfmathsetmacro\fracprog{\i/\arcrange-#2/\arcrange}
      \pgfmathsetmacro\col{\fracprog*100}
      \draw[color={#6!\col!#7},line width=#1] (\i-\progsign:#4+\radrange*\fracprog)
  arc[start angle=\i-\progsign, end angle=\i+1.1*\progsign,radius=#4+\fracprog*\radrange];
    }
  }

  \newcommand{\preassim}[5]{%width, startang, stopang, rad, col
    \pgfmathsetmacro\halfarc{#3/2-#2/2}
    \draw[color={autocatacyc},line width=#1] (#3:#4)
        arc[start angle=#3, end angle=#3-\halfarc,radius=#4];
    \pgfmathsetmacro\startshade{#2}
    \pgfmathsetmacro\endshade{#3-\halfarc}
    \colorgradarc{#1}{\startshade}{\endshade}{#4}{#5}{autocatacyc}
  }

  \newcommand{\postassim}[6]{%width,startang,stopang,rad,col,ratio
    \pgfmathsetmacro\halfarc{#3/2-#2/2}
    \pgfmathsetmacro\progsign{\halfarc>0 ? 1 : -1}
    \pgfmathsetmacro\quarterarc{#3/4-#2/4}
    \pgfmathsetmacro\endshade{#3-\halfarc}
    \pgfmathsetmacro\startbranch{#3-\quarterarc}
    \colorgradarc{#1*#6}{#2}{\endshade}{#4+#1*#6/2-#1/2}{autocatacyc}{#5}
    \draw[color={#5},line width=#1*#6] (\endshade:#4+#1*#6/2-#1/2)
        arc[start angle=\endshade, end angle=\startbranch+\progsign,radius=#4+#1*#6/2-#1/2];
    \draw[color={#5},line width=#1] (\startbranch:#4)
        arc[start angle=\startbranch, end angle=#3,radius=#4];

    \pgfmathsetmacro\arcrange{-\quarterarc}
    \pgfmathsetmacro\outstart{\startbranch+180}
    \pgfmathsetmacro\progsign{\arcrange>0 ? 1 : -1}
    \pgfmathsetmacro\arcend{\outstart+\arcrange-\progsign}
    \pgfmathsetlength\arrowwidth{#1*#6-#1}
    \begin{scope}[shift={(\startbranch:2*#4+#1*#6/2)}]
      \draw[color={#5},line width=#1*#6-#1] (\outstart:#4)
          arc[start angle=\outstart, end angle=\arcend,radius=#4];
      \revarrowhead{\arrowwidth}{\arcend}{#4}{#5}
    \end{scope}
  }

  \newcommand{\assim}[5]{%width,startang,deltaang,rad,ratio
    \pgfmathsetmacro\assimstart{#2+180}
    \begin{scope}[shift={(#2:2*#4+#1*#5/2)}]
      \colorgradarc{#1*#5-#1}{\assimstart}{\assimstart+#3}{#4}{autocatacyc}{assimcol}
    \end{scope}
  }

  \newcommand{\arrowhead}[4]{%width,startang,rad,col
  \fill[#4] (#2+1:#3-#1/2) arc (#2+1:#2:#3-#1/2)
       -- (#2-\protrude:#3) -- (#2:#3+#1/2) arc (#2:#2+1:#3+#1/2) -- cycle;
     }
  \newcommand{\revarrowhead}[4]{%width,startang,rad,col
  \fill[#4] (#2-1:#3-#1/2) arc (#2-1:#2:#3-#1/2)
       -- (#2+\protrude:#3) -- (#2:#3+#1/2) arc (#2:#2-1:#3+#1/2) -- cycle;
     }

