\documentclass[final,10pt]{beamer}

\usepackage[orientation=portrait,size=a3,scale=1]{beamerposter}
\usepackage[absolute,overlay]{textpos}
\usepackage{tcolorbox}

\usepackage{color,amsmath,graphicx,subcaption,geometry,mathtools,xfrac}
\usepackage{cite}
\usepackage{mhchem}
\usepackage{tikz}
\usepackage{pgfplots}
\pgfplotsset{compat=1.12}
\usepackage{stackengine,ifthen}
\usetikzlibrary{arrows,positioning,calc,arrows.meta,patterns,fit}
\newtoggle{article}
\newtoggle{eddpathway}

\tikzset{>=Latex}
\newcommand\influx{0.5}

\newenvironment{customlegend}[1][]{
  \begingroup
  \csname pgfplots@init@cleared@structures\endcsname
  \pgfplotsset{#1}
}{
  \csname pgfplots@createlegend\endcsname
  \endgroup
}

\def\addlegendimage{\csname pgfplots@addlegendimage\endcsname}
\setlength\abovecaptionskip{6pt}
\providecommand{\abs}[1]{\lvert#1\rvert}
\providecommand{\norm}[1]{\lVert#1\rVert}

\tikzset{>=latex}
%\tikzset{metaboliteStyle/.style={rectangle,draw}}
\tikzset{metaboliteStyle/.style={}}
\definecolor{cyan}{RGB}{100,181,205}
\definecolor{blue}{RGB}{76,114,176}
\definecolor{green}{RGB}{85,168,104}
\definecolor{magenta}{RGB}{129,114,178}
\definecolor{yellow}{RGB}{204,185,116}
\definecolor{red}{RGB}{196,78,82}
\definecolor{graybg}{gray}{0.95}

\colorlet{assimcol}{green}
\colorlet{sumcolor}{yellow}

\colorlet{inputcol}{green}
\colorlet{branchout}{red}
\colorlet{branchoutfl}{red!80}
\colorlet{autocatacyc}{blue}
\colorlet{autocatacycfl}{blue!80}
\colorlet{autocataby}{cyan}

\pdfpageattr{/Group <</S /Transparency /I true /CS /DeviceRGB>>} 

\def\blendfrac{0.5}
\def\deltaang{-155}
\def\fromang{180}
\def\inputang{-40}
\def\protrude{7}
\def\arcwidth{0.3cm}
\def\highlightrad{0.2cm}
\def\autocatalrad{1.5cm}
\def\autocatalscale{1.5}

  \newcommand{\shadedarc}[7][\arcwidth]{%width,startang,stopang,startrad,stoprad,startcol,stopcol
    \pgfmathsetmacro\arcrange{#3-#2}
    \pgfmathsetmacro\radrange{#5-#4}
    \pgfmathsetmacro\progsign{\arcrange>0 ? 1 : -1}
    \foreach \i in {#2,...,\numexpr#3-1\relax} {
      \pgfmathsetmacro\fracprog{\i/\arcrange-#2/\arcrange}
      \pgfmathsetmacro\col{\fracprog*100}
      \draw[color={#6!\col!#7},line width=#1] (\i:#4+\radrange*\fracprog)
  arc[start angle=\i, end angle=\i+1.1*\progsign,radius=#4+\fracprog*\radrange];
    }
  }

  \newcommand{\coloredarc}[6][\arcwidth]{%width,startang,stopang,startrad,stoprad,startcol,stopcol
    \pgfmathsetmacro\arcrange{#3-#2}
    \pgfmathsetmacro\radrange{#5-#4}
    \pgfmathsetmacro\progsign{\arcrange>0 ? 1 : -1}
    \foreach \i in {#2,...,\numexpr#3-1\relax} {
      \pgfmathsetmacro\fracprog{\i/\arcrange-#2/\arcrange}
      \draw[color={#6},line width=#1] (\i:#4+\radrange*\fracprog)
        arc[start angle=\i, end angle=\i+1.1*\progsign,radius=#4+\fracprog*\radrange];
    }
  }



\tikzset{
  invisible/.style={opacity=0},
  visible on/.style={}
}

\usepackage{adjustbox}
\toggletrue{article}
\toggletrue{poster}
\togglefalse{eddpathway}

\tikzset{every picture/.style={font issue=Large},
  font issue/.style={execute at begin picture={#1\selectfont}}
}

\setbeamercolor{block title}{bg=blue!20}
\setbeamertemplate{caption}[numbered]
\setbeamertemplate{navigation symbols}{}

\setbeamertemplate{footline}{}
\setbeamertemplate{bibliography item}{\insertbiblabel}

\setbeamertemplate{headline}{
\leavevmode

\begin{beamercolorbox}[wd=\paperwidth]{headline}
    \begin{columns}[T]
        \begin{column}{.05\paperwidth}
        \end{column}
        \begin{column}{.7\paperwidth}
            \vskip16ex
            {\centering{
            \usebeamercolor{title in headline}{\color{fg}\textbf{\LARGE{\inserttitle}}\\[1ex]}
        }}
        \vskip8ex
            \usebeamercolor{author in headline}{\color{fg}\large{\insertauthor}\\[1ex]}
            \usebeamercolor{institute in headline}{\color{fg}\large{\insertinstitute}\\[1ex]}
            \normalsize{Email: uri.barenholz@weizmann.ac.il}
            \vskip1ex
        \end{column}
        \begin{column}{.25\paperwidth}
            \vskip2cm
            \begin{center}
                \includegraphics[width=\textwidth]{weizmann_logo.pdf}
            \end{center}
            \vskip-0.5cm
        \end{column}
        \begin{column}{.03\paperwidth}
        \end{column}
    \end{columns}
\end{beamercolorbox}

\begin{beamercolorbox}[wd=\paperwidth]{lower separation line head}
    \rule{0pt}{2pt}
\end{beamercolorbox}
\vskip-2cm
}

\setbeamertemplate{block begin}{
    \begin{beamercolorbox}[rounded=true,shadow=true,leftskip=1ex]{block title}
        \usebeamerfont*{block title}\insertblocktitle
    \end{beamercolorbox}
    \usebeamerfont{block body}
    \begin{beamercolorbox}[rounded=true,vmode]{block body}
    }
\setbeamertemplate{block end}{
    \end{beamercolorbox}
}

\title{\Huge Design principles of autocatalytic cycles constrain enzyme kinetics and force over-expression at flux branch points}
\author{\underline{Uri Barenholz}, Dan Davidi, Ed Reznik, Yinon Bar-On, Niv Antonovsky, Elad Noor, Ron Milo}
\institute{Department of Plant and Environmental Sciences, Weizmann Institute of Science}

\begin{document}

\newlength\gridsize
\pgfmathsetlength{\gridsize}{8cm}
\newlength\plotwidth
\pgfmathsetlength{\plotwidth}{4.6cm}
\newlength\plotheight
\pgfmathsetlength{\plotheight}{4.5cm}
\newlength\plotwidthanim
\pgfmathsetlength{\plotwidthanim}{8cm}
\newlength\plotheightanim
\pgfmathsetlength{\plotheightanim}{7cm}
\newlength\plotshift
\pgfmathsetlength{\plotshift}{-6cm}

\newlength\assimwidth

\newlength\cbbimrad
\newlength\cbbierad
\newlength\cbbesrad
\newlength\cbbemrad
\newlength\cbbeerad
\newlength\cbbwidth
\newlength\cbbtotwidth

\newlength\glyimrad
\newlength\glyierad
\newlength\glyesrad
\newlength\glyemrad
\newlength\glyeerad
\newlength\glywidth
\newlength\glytotwidth
\newlength\glyfinwidth
\newlength\glyimmrad
\newlength\glyemmrad
\newlength\glyemmmrad
\newlength\glyeamrad

\newlength\ptsierad
\newlength\ptsimrad
\newlength\ptsarcwidth
\newlength\ptsesrad
\newlength\ptsemrad




\begin{frame}{}
    \vskip9ex
    \begin{center}
    \begin{minipage}{0.9\textwidth}
       \begin{block}{\centering{Summary}}
           \begin{columns}[t]
    \begin{column}{0.66\textwidth}
            \vskip1mm
            A set of chemical reactions that require a metabolite to synthesize more of that metabolite is an autocatalytic cycle.
            Such metabolic designs must meet specific conditions to support stable fluxes, hence avoiding depletion of intermediate metabolites.
            We find that most of the reactions in central carbon metabolism are part of compact autocatalytic cycles.
            As such, they are subjected to constraints that may seem irrational: the enzymes of branch reactions out of the cycle must be overexpressed and the affinity of these enzymes to their substrates must be relatively low.
            Analysis of recent quantitative proteomics and fluxomics measurements shows that the above conditions hold for all functioning cycles in central carbon metabolism of \emph{E.coli}.
            Therefore, the topology of metabolic networks can shape kinetic parameters of enzymes and lead to apparently reduced enzyme efficiency \cite{Barenholz2016-ze}.
    \end{column}
    \begin{column}{0.25\textwidth}
        \vskip1ex
            \begin{adjustbox}{max totalsize={0.8\textwidth}{\textheight},center}
                \begin{tikzpicture}
\begin{scope} [shift={(-6.6cm,-4cm)}]
  \colorlet{genext}{assimcol}
  \colorlet{genmed}{blue}
  \colorlet{geninit}{blue}

  \newlength\imrad;
  \newlength\ierad;
  \newlength\esrad;
  \newlength\emrad;
  \newlength\eerad;
  \pgfmathsetlength{\imrad}{\autocatalrad-\blendfrac*\arcwidth};
  \pgfmathsetlength{\ierad}{\autocatalrad-0.5*\arcwidth};
  \pgfmathsetlength{\esrad}{\autocatalrad+\arcwidth};
  \pgfmathsetlength{\emrad}{\autocatalrad+\arcwidth-\blendfrac*\arcwidth};
  \pgfmathsetlength{\eerad}{\autocatalrad+0.5*\arcwidth};

  \preassim{\autocatalscale*\arcwidth}{-100}{-270}{\autocatalscale*\autocatalrad}{geninit};%width, startang, stopang, rad, col
    \postassim{\autocatalscale*\arcwidth}{90}{-45}{\autocatalscale*\autocatalrad}{geninit}{2}
    \assim{\autocatalscale*\arcwidth}{90}{-30}{\autocatalscale*\autocatalrad}{2}
    \arrowhead{\autocatalscale*\arcwidth}{-45}{\autocatalscale*\autocatalrad}{geninit}

    \node[align=center] at (-30:\autocatalscale*\autocatalrad*2.1) (int) {$\gamma \cdot$ Internal metabolite};
    \node[align=center] at (-73:\autocatalscale*\autocatalrad) (int) {Internal\\metabolite};
    \node[align=center] at (130:\autocatalscale*\autocatalrad*1.65) (ext) {External\\metabolite};
    \node [rectangle,fill=autocatacyc!30,rounded corners=3pt] at (90:\autocatalscale*\autocatalrad+1.7cm) (eq) {Internal metabolite + External metabolite $\rightarrow (1+\gamma)$ Internal metabolite};
  \end{scope}
\end{tikzpicture}


            \end{adjustbox}
    \end{column}
\end{columns}
        \end{block}
    \end{minipage}
\end{center}

\begin{columns}[t]
    \begin{column}{0.48\textwidth}
        \begin{block}{\centering{Autocatalytic cycles present in central carbon metabolism}}
            \begin{figure}
            \begin{adjustbox}{max totalsize={\textwidth}{\textheight},center}
                \begin{tikzpicture}
 \colorlet{ptsinit}{cyan}
  \colorlet{cbbinit}{yellow}
  \colorlet{glyinit}{magenta}

  \newlength\assimwidth;
  \pgfmathsetlength{\assimwidth}{1.5pt};

  \node[metaboliteStyle] (g6p) {g6p};

  %%%% upper pts
  \node[shape=coordinate,left=12mm of g6p.center] (ptsmid) {};
  \node[metaboliteStyle,left=7mm of ptsmid,rectangle,draw=assimcol,rounded corners=2pt] (gluc) {gluc};
  \node[metaboliteStyle,shift={(-7mm,-7mm)},gray] at (g6p.center) (pyr1) {pyr};
  \draw[assimcol,line width=\assimwidth] (gluc) -- (ptsmid);
  \draw[->] (ptsmid) [out=0,in=90] to (pyr1);

  \node[metaboliteStyle,below=of g6p.center] (f6p) {f6p};
  \node[metaboliteStyle,below=of f6p] (fbp) {fbp};
  \node[metaboliteStyle,shape=coordinate,below=of fbp.center](fbamid) {};
  \node[metaboliteStyle,below left=of fbamid.center] (dhap) {dhap};
  \node[metaboliteStyle,below right=of fbamid] (gap) {gap};
  \node[metaboliteStyle,below=of gap.center] (bpg) {bpg};
  \node[metaboliteStyle,below=of bpg.center] (3pg) {3pg};
  \node[metaboliteStyle,below=of 3pg.center] (2pg) {2pg};
  \node[metaboliteStyle,below=of 2pg.center] (pep) {pep};
  \node[metaboliteStyle,below=of pep.center] (pyr) {pyr};
  \node[metaboliteStyle,below=of pyr.center,rectangle,draw=assimcol,rounded corners=2pt] (aca) {accoa};
  \node[shape=coordinate,below=of aca] (dummyglta) {};
  \node[metaboliteStyle,left=of dummyglta] (oaa) {oaa};
  \node[metaboliteStyle,right=of dummyglta] (cit) {cit};
  \node[metaboliteStyle,right=of cit] (icit) {icit};
  \node[metaboliteStyle,below=of icit.center] (akg) {akg};
  \node[metaboliteStyle,below=of akg.center] (sca) {sca};
  \node[metaboliteStyle,below=of oaa.center] (mal) {mal};
  \node[metaboliteStyle,below=of mal.center] (fum) {fum};
  \node[metaboliteStyle,right=of mal] (glx) {glx};
  \node[metaboliteStyle,right=of fum] (suc) {suc};
  \node[metaboliteStyle,right=of g6p] (6pgi) {6pgi};
  \node[metaboliteStyle,shape=coordinate,right=of f6p] (s7pspace) {};
  \node[metaboliteStyle,right=of s7pspace] (s7p) {s7p};
  \node[metaboliteStyle,right=of s7p] (r5p) {r5p};
  \node[metaboliteStyle,right=of r5p] (ru5p) {ru5p};
  \node[metaboliteStyle,above=of ru5p.center] (6pgc) {6pgc};
  \node[metaboliteStyle,] at (fbp.center -| s7p.center) (e4p) {e4p};
  \node[metaboliteStyle,] at(e4p.center -| r5p.center) (xu5p) {xu5p};
  \node[metaboliteStyle,] at(3pg.center -| ru5p.center) (rub) {rubp};
  \node[metaboliteStyle,rectangle,draw=assimcol,rounded corners=2pt] at(2pg -| xu5p.center) (co2) {\ce{CO2}};
  \draw[->] (g6p) -- (f6p);
  \draw[->] ([xshift=0.1cm]f6p.south) -- ([xshift=0.1cm]fbp.north);
  \draw[<-] ([xshift=-0.1cm]f6p.south) -- ([xshift=-0.1cm]fbp.north);
  \draw [<-] (fbp) [out=-90,in=90] to (fbamid);
  \draw [->] (fbamid) [out=-90,in=45] to (dhap);
  \draw [->] (fbamid) [out=-90,in=135] to (gap);
  \draw [<->] (dhap) -- (gap);
  \draw[<->] (gap) -- (bpg);
  \draw[<->] (bpg) -- (3pg);
  \draw[->] (3pg) -- (2pg);
  \draw[->] (2pg) -- (pep);
  \draw[->] (pep) -- (pyr);
  \draw[->] (pyr) -- (aca);
  \draw[->] (oaa) -- (cit) node [pos=0.9] (midglta) {};
  \draw [assimcol,line width=\assimwidth] (aca) [out=-70,in=180] to (midglta);
  \draw[->] (cit) -- (icit);
  \draw[->] (icit) -- (suc) node [pos=0.3] (midacea) {};
  \draw[->] (midacea) [out=220,in=0] to (glx);
  \draw[->] (icit) -- (akg);
  \draw[->] (akg) -- (sca);
  \draw[->] (sca) -- (suc);
  \draw[->] (suc) -- (fum);
  \draw[->] (fum) -- (mal);
  \draw[->] (glx) -- (mal) node [pos=0.9] (midaceb) {};
  \draw[->] (mal) -- (oaa);
  \draw[assimcol,line width=\assimwidth] (aca) [out=-90,in=0] to (midaceb);
  \draw[->] (g6p) -- (6pgi);
  \draw[->] (6pgi) -- (6pgc);
  \draw[->] (6pgc) -- (ru5p);
  \draw[<-] (ru5p) -- (xu5p);
  \draw[<-] (ru5p) -- (r5p);
  \path[] (r5p) -- (gap) coordinate [pos=0.2] (midtkt1) {};
  \draw[<-] (xu5p) [out=180,in=-90] to (midtkt1);
  \draw[] (midtkt1) [out=90,in=0] to (s7p);
  \draw[<-] (r5p) [out=180,in=90] to (midtkt1);
  \draw[] (midtkt1) [out=270,in=0] to (gap);
  \path[] (e4p) -- (gap) coordinate [pos=0.4] (midtkt2) {};
  \draw[<-] (e4p) [out=-60,in=0] to (midtkt2);
  \draw[] (midtkt2) [out=180,in=90] to (gap);
  \draw[<-] (xu5p) [out=245,in=0] to (midtkt2);
  \draw[] (midtkt2) [out=180,in=-30] to (f6p);
  \path[] (e4p) -- (s7pspace) coordinate [pos=0.5] (midtal) {};
  \draw[] (midtal) [out=90,in=0] to (f6p);
  \draw[<-] (s7p) [out=180,in=90] to (midtal);
  \draw[] (midtal) [out=-90,in=180] to (e4p);
  \draw[<-] (gap) [out=55,in=-90] to (midtal);
  \node[shape=coordinate,left=2.5cm of pep.center] (pts3) {};
  \draw[] (pep) [out=180,in=0] to (pts3);
  \node[shape=coordinate,left=2.5cm of pyr.center] (pts5) {};
  \node[shape=coordinate,left=2.1cm of f6p.center] (ptstop) {};
  \node[shape=coordinate] at(ptstop |- 2pg.center) (ptsbottom) {};
  \draw[] (pts3) [in=-90,out=180] to (ptsbottom);
  \draw[] (ptsbottom) [in=-90,out=90] to (ptstop);
  \draw[] (ptstop) [in=180,out=90] to (ptsmid);
  \draw[->] (ptsmid) -- (g6p);
  \draw[->] (ru5p) [out=-90,in=90] to (rub);
  \draw[->] (rub) -- (3pg) coordinate [pos=0.9] (rubisco); 
  \draw[assimcol,line width=\assimwidth] (co2) [out=90,in=0] to (rubisco);
  \node[shape=coordinate,shift={(-\highlightrad,-\highlightrad)}] at (pep.south -| ptsbottom) (ptsbottomlimit) {};
  \node[shape=coordinate,shift={(-\highlightrad,\highlightrad)}] at (g6p.north -| ptstop) (ptstoplimit) {};

  \draw[opacity=0.2,fill=ptsinit,rounded corners=\highlightrad] ([shift={(\highlightrad,\highlightrad)}]g6p.north east) -- ([xshift=\highlightrad] fbp.east) -- ([shift={(\highlightrad,\highlightrad)}]gap.north east)--([shift={(\highlightrad,-\highlightrad)}]pep.south east) -- (ptsbottomlimit) -- node[midway] (ptsshademid) {} ([yshift=-1.2cm]ptstoplimit) -- ([shift={(-1mm,\highlightrad)}]g6p.north -| ptsmid) -- cycle;

  \draw[very thick,dashed,cyan,->] (ptsshademid) -- ++(-1.5cm,0cm); 

  \draw[opacity=0.2,fill=glyinit,rounded corners=\highlightrad] ([shift={(-\highlightrad,2*\highlightrad)}]oaa.west) -- ([shift={(\highlightrad,2*\highlightrad)}]icit.east) -- node[midway] (glyshadedmid) {}([shift={(\highlightrad,-0.5*\highlightrad)}]icit.south east) -- ([shift={(0.5*\highlightrad,-2*\highlightrad)}]suc.east) -- ([shift={(-\highlightrad,-2*\highlightrad)}]fum.west) -- cycle;

  \draw[very thick,dashed,magenta,->] (glyshadedmid) -- ++(1.5cm,0cm); 

  \node[shape=coordinate] at (dhap.south -| gap.west) (cbbmid) {};
  \draw[opacity=0.2,fill=cbbinit,rounded corners=\highlightrad] ([shift={(-\highlightrad,2.2*\highlightrad)}]f6p.west) -- ([shift={(3*\highlightrad,2.2*\highlightrad)}]ru5p.center) -- node[midway] (cbbshadedmid) {} ([shift={(3*\highlightrad,-2*\highlightrad)}]rub.center) -- ([shift={(-\highlightrad,-2*\highlightrad)}]3pg.west) -- ([shift={(-\highlightrad,-\highlightrad)}]cbbmid) -- ([shift={(-0.5*\highlightrad,-\highlightrad)}]dhap.south west) -- ([shift={(-0.5*\highlightrad,0.5*\highlightrad)}]dhap.north west) -- ([xshift=-\highlightrad]fbp.west) -- cycle;

  \draw[very thick,dashed,cbbinit,->] (cbbshadedmid) -- ++(1.5cm,0cm); 

  %% CBB cycle
  \begin{scope} [shift={(11.2cm,-4.5cm)},radius=2cm]
    \draw[lightgray,rounded corners=\highlightrad] (-2.7,-2.1) rectangle +(4.6,4.5);
    \node at (-2.2cm,2cm) (I) {\large  \textbf{I}};
    \node[anchor=north] at(-0.4cm,-2.2cm) (cbbreac) {{\fontfamily{cmss}\selectfont 5}  3pg + {\fontfamily{cmss}\selectfont 3} \ce{CO2} $\rightarrow$ {\fontfamily{cmss}\selectfont 6} 3pg};
    \colorlet{cbbmed}{blue}
    \colorlet{cbbext}{assimcol}
  

    \newlength\cbbimrad;
    \newlength\cbbierad;
    \newlength\cbbesrad;
    \newlength\cbbemrad;
    \newlength\cbbeerad;
    \newlength\cbbwidth;
    \newlength\cbbtotwidth;
    \pgfmathsetlength{\cbbwidth}{\arcwidth*0.2};
    \pgfmathsetlength{\cbbtotwidth}{\cbbwidth+\arcwidth};
    \pgfmathsetlength{\cbbimrad}{\autocatalrad-\blendfrac*0.5*\cbbtotwidth};
    \pgfmathsetlength{\cbbierad}{\autocatalrad-0.5*\cbbwidth};
    \pgfmathsetlength{\cbbesrad}{\autocatalrad+0.5*\cbbtotwidth};
    \pgfmathsetlength{\cbbemrad}{\autocatalrad+0.5*\cbbtotwidth-\blendfrac*0.5*\cbbtotwidth};
    \pgfmathsetlength{\cbbeerad}{\autocatalrad+0.5*\arcwidth};

    \shadedarc{-20}{-180}{\autocatalrad}{\autocatalrad}{cbbmed}{cbbinit};
    \shadedarc{100}{180}{\cbbimrad}{\autocatalrad}{cbbmed}{cbbinit};
    \coloredarc{25}{100}{\cbbierad}{\cbbimrad}{cbbinit};
    \shadedarc[\cbbwidth]{100}{180}{\cbbemrad}{\cbbesrad}{cbbext}{cbbinit};
    \coloredarc[\cbbwidth]{25}{100}{\cbbeerad}{\cbbemrad}{cbbinit};

%% \assimilatedcol input arc
        \draw[color=cbbext,line width=\cbbwidth]
        (\fromang:\autocatalrad+0.5*\arcwidth+0.5*\cbbwidth)
        arc (0:\inputang:2cm)
        node [pos=0.5,color=black,anchor=east] (co2c) {\ce{CO2}};
%% arrowhead
    \fill[cbbinit]
      (\fromang+\deltaang+1:\autocatalrad-0.5*\arcwidth-0.5*\cbbwidth)
      arc (\fromang+\deltaang+1:\fromang+\deltaang-1:\autocatalrad-0.5*\arcwidth-0.5*\cbbwidth)
      -- (\fromang+\deltaang-1-\protrude:\autocatalrad)
      -- (\fromang+\deltaang-1:\autocatalrad+0.5*\arcwidth+0.5*\cbbwidth)
      arc (\fromang+\deltaang-1:\fromang+\deltaang+1:\autocatalrad+0.5*\arcwidth+0.5*\cbbwidth)
      -- cycle;

%% metabolites
        \node at (0:\autocatalrad) (3pgc) {3pg};
        \node at (220:\autocatalrad-4.7mm) (rubc) {rubp};

  \end{scope}

  %glyoxilate cycle
\begin{scope} [shift={(9cm,-13.5cm)},radius=2cm]
  \draw[lightgray,rounded corners=\highlightrad] (-2.4cm,-2.6cm) rectangle +(4.5,5.1);
  \node at (-2cm,2.2cm) (II) {\large  \textbf {II}};
  \node[anchor=north] at(-0.15cm,-2.7cm) (glyreac) {mal + {\fontfamily{cmss}\selectfont 2} accoa $\rightarrow$ {\fontfamily{cmss}\selectfont 2} mal};
  \colorlet{glymed}{blue}
  \colorlet{glyext}{assimcol}
  \colorlet{glyinter}{blue}
  
    \newlength\glyimrad;
    \newlength\glyierad;
    \newlength\glyesrad;
    \newlength\glyemrad;
    \newlength\glyeerad;
    \newlength\glywidth;
    \newlength\glytotwidth;
    \newlength\glyfinwidth;
    \newlength\glyimmrad;
    \newlength\glyemmrad;
    \newlength\glyemmmrad;
    \newlength\glyeamrad;
    \pgfmathsetlength{\glywidth}{\arcwidth*0.5};
    \pgfmathsetlength{\glytotwidth}{\glywidth+\arcwidth};
    \pgfmathsetlength{\glyfinwidth}{\glytotwidth+\glywidth};
    \pgfmathsetlength{\glyimrad}{\autocatalrad-\blendfrac*0.5*\glywidth};
    \pgfmathsetlength{\glyimmrad}{\autocatalrad-0.5*\glywidth};
    \pgfmathsetlength{\glyemmrad}{\glyimmrad+0.5*\glytotwidth};
    \pgfmathsetlength{\glyemmmrad}{\glyemmrad+0.5*\glywidth};
    \pgfmathsetlength{\glyeamrad}{\autocatalrad+0.5*\glytotwidth};
    \pgfmathsetlength{\glyierad}{\autocatalrad-0.5*\glywidth};
    \pgfmathsetlength{\glyesrad}{\autocatalrad+0.5*\glytotwidth};
    \pgfmathsetlength{\glyemrad}{\autocatalrad+0.5*\glyfinwidth-\blendfrac*0.5*\glywidth};
    \pgfmathsetlength{\glyeerad}{\autocatalrad+0.5*\glytotwidth};

    \shadedarc{-20}{-90}{\autocatalrad}{\autocatalrad}{glymed}{glyinit};
    \shadedarc{-150}{-90}{\glyimmrad}{\autocatalrad}{glymed}{glyinter};
    \shadedarc[\glywidth]{-150}{-90}{\glyemmrad}{\glyeamrad}{glyext}{glyinter};

    \draw[color=glyext,line width=\glywidth] (-90:\autocatalrad+0.5*\glytotwidth) arc(90:50:2cm) node [pos=0.5,color=black,anchor=north,xshift=-1.5mm] (acac) {accoa};

    \coloredarc[\glytotwidth]{160}{90}{\glyimmrad}{\glyimmrad}{glyinter};

    \draw[color=glyext,line width=\glywidth] (90:\glyemmmrad) arc(-90:-130:2cm) node [pos=0.5,color=black,anchor=south,xshift=2mm] (acac) {accoa};

    \shadedarc[\glytotwidth]{50}{90}{\glyimrad}{\glyimmrad}{glyinter}{glyinit};
    \coloredarc[\glytotwidth]{50}{25}{\glyimrad}{\glyierad}{glyinit};

    \shadedarc[\glywidth]{90}{50}{\glyemmmrad}{\glyemrad}{glyinit}{glyext};
    \coloredarc[\glywidth]{50}{25}{\glyemrad}{\glyeerad}{glyinit};


    \node at (0:\autocatalrad) (malc) {mal};
    \node at (180:\autocatalrad) (glxc) {glx+fum};
    \node at (-50:\autocatalrad-4.5mm) (oaa) {oaa};

  \fill[glyinit] (\fromang+\deltaang+1:\autocatalrad-\arcwidth) arc (\fromang+\deltaang+1:\fromang+\deltaang-1:\autocatalrad-\arcwidth)
       -- (\fromang+\deltaang-1-\protrude:\autocatalrad) -- (\fromang+\deltaang-1:\autocatalrad+\arcwidth) arc (\fromang+\deltaang-1:\fromang+\deltaang+1:\autocatalrad+\arcwidth)
       -- cycle;

  \fill[glyinter] (-149:\autocatalrad-\arcwidth+0.5*\glywidth) arc (-149:-161:\autocatalrad-\arcwidth+0.5*\glywidth)
       -- (-161-\protrude:\autocatalrad) -- (-161:\autocatalrad+\arcwidth-0.5*\glywidth) arc (-161:-149:\autocatalrad+\arcwidth-0.5*\glywidth)
       -- cycle;
  \end{scope}


  %pts cycle
\begin{scope} [shift={(-5.8cm,-5.5cm)},radius=2cm]
  \draw[lightgray,rounded corners=\highlightrad] (-2.5,-3) rectangle +(4.5,5.5);
  \node at (-2cm,2.2cm) (III) {\large  \textbf {III}};
  \node[anchor=north] at(-0.25cm,-3.1cm) (ptsreac) {gap + gluc $\rightarrow$ {\fontfamily{cmss}\selectfont 2} gap + pyr};
    \colorlet{ptsmed}{blue}
    \colorlet{ptsext}{assimcol}

    \newlength\ptsierad;
    \newlength\ptsimrad;
    \newlength\ptsarcwidth;
    \newlength\ptsesrad;
    \newlength\ptsemrad;
    \pgfmathsetlength{\ptsierad}{\autocatalrad*0.5};
    \pgfmathsetlength{\ptsimrad}{\autocatalrad-0.5*\arcwidth};
    \pgfmathsetlength{\ptsarcwidth}{2*\arcwidth};
    \pgfmathsetlength{\ptsesrad}{\autocatalrad+0.5*\arcwidth+0.5*\ptsarcwidth};
    \pgfmathsetlength{\ptsemrad}{\autocatalrad+0.5*\ptsarcwidth};

    \shadedarc{-20}{-120}{\autocatalrad}{\autocatalrad}{ptsmed}{ptsinit};
    \shadedarc{160}{240}{\ptsimrad}{\autocatalrad}{ptsmed}{ptsinit};
    \shadedarc{70}{160}{\ptsierad}{\ptsimrad}{ptsinit}{white};
    \shadedarc[\ptsarcwidth]{160}{240}{\ptsemrad}{\ptsesrad}{ptsext}{ptsinit};
    \coloredarc[\ptsarcwidth]{25}{160}{\autocatalrad}{\ptsemrad}{ptsinit};

  \fill[ptsinit] (\fromang+\deltaang+1:\autocatalrad-\arcwidth) arc (\fromang+\deltaang+1:\fromang+\deltaang-1:\autocatalrad-\arcwidth)
       -- (\fromang+\deltaang-1-\protrude:\autocatalrad) -- (\fromang+\deltaang-1:\autocatalrad+\arcwidth) arc (\fromang+\deltaang-1:\fromang+\deltaang+1:\autocatalrad+\arcwidth)
       -- cycle;

       %% \assimilatedcol input arc
       \draw[color=ptsext,line width=\ptsarcwidth] (240:\ptsesrad) arc(60:20:2cm) node [pos=0.5,color=black,left,shift={(-1mm,-2mm)}] (glucc) {gluc};
    \node at (0:\autocatalrad) (gapc) {gap};
    \node at (270:\autocatalrad-4.5mm) (pepc) {pep};
    \node at (70:\ptsierad) (pyrc) {pyr};
  \end{scope}


\end{tikzpicture}


            \end{adjustbox}
        \caption{
            \label{fig:realautocatal}
        Three representative autocatalytic cycles in central carbon metabolism: (I) The Calvin-Benson-Bassham cycle (yellow); (II) The glyoxylate cycle (magenta); (III) A cycle using the PTS system to assimilate glucose (cyan).
        Assimilation reactions are indicated in green.
        Arrow width in panels represent relative carbon mass.
        Two other potential autocatalytic cycles identified are: a cycle using the Pentose-Phosphate pathway to assimilate ribose, and a cycle using fba in the gluconeogenic direction with the Entner-Doudoroff pathway to assimilate glycerol.
}
\end{figure}
    \end{block}
         \begin{block}{\centering{A simple autocatalytic cycle model reveals constraints on the kinetic parameters of catalyzing enzymes}}
             \begin{figure}
            \begin{adjustbox}{max totalsize={\textwidth}{\textheight},center}
                  \begin{tikzpicture}[>=latex',node distance = 2cm]
    \tikzset{
        vstyle/.style={opacity=0.3,pattern=north west lines,cyan,visible on=<7->}}
    \tikzset{
        kstyle/.style={opacity=0.3,pattern=north east lines,magenta,visible on=<7->}}
  \begin{scope}[shift={(-4cm,4.3cm)}]
        \node at (-60:1cm) (X) {$X$};
        \node[shape=coordinate] (orig) {};
        \draw [-,line width=1pt,autocatacyc] (X.south west) arc (285:0:1cm) node [pos=0.65,above] (fa) {$f_a:$\small{$A+X\rightarrow2X$}} node [pos=0.45,shape=coordinate] (midauto) {} node [pos=1,shape=coordinate] (endcommon) {};
        \draw [->,line width=1pt,autocatacyc] (endcommon) arc (-25:-44:2cm);
        \draw [->,line width=1pt,autocatacyc] (endcommon) arc (-5:-32:1.5cm);
        \draw [line width=1pt,assimcol] (midauto) arc (-60:-90:1cm) node [pos=1,left] (e) {$A$};
        \draw [->,line width=1pt,branchout] (X.south east) arc (225:270:1cm) node [pos=0.75,above] {$f_b$};
        \iftoggle{article} {
            \node at (-2.4cm,1.3cm) (A) {(A)};
        }{}
  \end{scope}
  \begin{scope}[shift={(-1.5cm,-\gridsize/2)}]
    \begin{axis}[name=phase,clip=false,xmin=0,ymin=0,xmax=2,ymax=2,ylabel={\Large{$\sfrac{V_{\max,b}}{V_{\max,a}}$}},xlabel={\Large{$\sfrac{K_{M,b}}{K_{M,a}}$}},samples=6,width=\gridsize,height=\gridsize,ytick={0,1,2},xtick={0,1,2},visible on=<7->]
        \addplot[domain=0:2,dotted,black,thick] {x};
        \addplot[dotted,black,thick] coordinates {(0,1) (2,1)};
        \draw[kstyle] (axis cs:0,0) -- (axis cs:2,2) -- (axis cs:2,0) --cycle;
        \draw[vstyle] (axis cs:0,1) -- (axis cs:2,1) -- (axis cs:2,2) -- (axis cs:0,2) --cycle;
        \draw[->,black!50,dashed] (axis cs:0.25,1.4) -- +(-1.9cm,0cm);
        \draw[->,black!50,dashed] (axis cs:1.75,1.4) -- +(1.1cm,0cm);
        \draw[->,black!50,dashed] (axis cs:0.25,0.6) -- +(-1.9cm,0cm);
        \draw[->,black!50,dashed] (axis cs:1.75,0.6) -- +(1.1cm,0cm);
        \node[align=left,anchor=east] at (axis cs:1.75,1.4) (I) {I};
        \node[align=right,anchor=west] at (axis cs:0.25,1.4) (II) {II};
        \node[align=right,anchor=west] at (axis cs:0.25,0.6) (III) {III};
        \node[align=left,anchor=east] at (axis cs:1.75,0.6) (IV) {IV};
      \end{axis}

\iftoggle{article} {
        \pgfmathsetlength{\plotwidthanim}{\plotwidth}
        \pgfmathsetlength{\plotheightanim}{\plotheight}
        \pgfmathsetlength{\plotshift}{1mm}
}{
    \only<5-> {
        \pgfmathsetlength{\plotwidthanim}{\plotwidth}
        \pgfmathsetlength{\plotheightanim}{\plotheight}
        \pgfmathsetlength{\plotshift}{1mm}
    }
}

      \begin{axis}[name=plot1,axis x line=middle,axis y line=left,xlabel near ticks,ylabel near ticks,xmin=0,ymin=-2.5,xmax=2.9,ymax=5.9,xlabel={[$X$]},ylabel={flux},samples=60,width=\plotwidthanim,height=\plotheightanim,clip=false,yticklabels={,,},xticklabels={,,},tick label style={major tick length=0pt},at=(phase.right of north east),anchor=left of north west,ylabel style={name=ylabel1},xshift=\plotshift,visible on=<2->]%,axis background/.style={fill=cyan!50!magenta,opacity=0.3}]
        \addplot[domain=0:2.9,autocatacyc,thick] {3*x/(0.1+x)};
        \addplot[domain=0:2.9,branchout,thick,visible on=<3->] {5*x/(1+x)};
        \addplot[domain=0:2.9,sumcolor,thick,visible on=<4->] {3*x/(0.1+x)-5*x/(1+x)};
        \addplot[dashed,gray,thick,visible on=<4->] coordinates {(1.25,0) (1.25,2.77)};
        \node[right,align=left,visible on=<6->] (onetext) at (axis cs:0.05,4.7) {\scriptsize \textbf{stable non-zero}\\[-0.4em]\scriptsize \textbf{steady state}};
      \end{axis}
     \iftoggle{elifesubmission} {}
     {
      \iftoggle{article} {}
      {
        \node[visible on=<2-4>,color=blue,at=(plot1.left of north west),anchor=north east,scale=1.5,xshift=-1cm,yshift=-0.5cm] (fa){$f_a=\frac{V_{\max,a}X}{K_{M,a}+X}$};
        \node[visible on=<3-4>,color=red,below=of fa,scale=1.5,yshift=0.7cm] (fb) {$f_b=\frac{V_{\max,b}X}{K_{M,b}+X}$};
      }
    }
      \node[draw,fit=(plot1) (ylabel1),line width=2pt, fill=none,rounded corners=3pt,cyan!50!magenta,opacity=0.6,visible on=<7->]{};

        \begin{customlegend}[legend entries={$f_a$,$V_{\max,b}>V_{\max,a}$,$f_b$,$\sfrac{V_{\max,b}}{V_{\max,a}}<\sfrac{K_{M,b}}{K_{M,a}}$,$\dot{X}=f_a-f_b$},legend style={above=1cm of plot1.north east,anchor=south east,name=legend1,visible on=<4->},legend columns=2]
          \addlegendimage{autocatacyc,fill=black!50!red,sharp plot,line width=1pt}
          \addlegendimage{vstyle,area legend,visible on=<5->}
          \addlegendimage{branchout,fill=black!50!red,sharp plot,line width=1pt}
          \addlegendimage{kstyle,area legend,visible on=<5->}
          \addlegendimage{sumcolor,fill=black!50!red,sharp plot,line width=1pt}
        \end{customlegend}

      \begin{axis}[name=plot2,axis x line=middle,axis y line=left,xlabel near ticks,ylabel near ticks,xmin=0,ymin=-2.5,xmax=2.9,ymax=5.9,xlabel={[$X$]},ylabel={flux},samples=60,at=(phase.left of north west),anchor=right of north east,width=\plotwidth,height=\plotheight,yticklabels={,,},xticklabels={,,},tick label style={major tick length=0pt},ylabel style={name=ylabel2},xshift=-1mm,visible on=<5->]%,axis background/.style=vstyle]
        \addplot[domain=0:4,autocatacyc,thick] {4*x/(1+x)};
        \addplot[domain=0:4,branchout,thick] {5*x/(0.2+x)};
        \addplot[domain=0:4,sumcolor,thick,visible on=<6->] {4*x/(1+x)-5*x/(0.2+x)};
        \node[right,align=left,visible on=<6->] (twotext) at (axis cs:0.0,5) {\scriptsize stable zero steady state};
      \end{axis}
     \iftoggle{article} {}
     {
      \node[draw,fit=(plot2) (ylabel2),line width=2pt, vstyle,fill=none,rounded corners=3pt]{};
  }

      \begin{axis}[name=plot3,axis x line=middle,axis y line=left,xlabel near ticks,ylabel near ticks,xmin=0,ymin=-2.5,xmax=2.9,ymax=5.9,xlabel={[$X$]},ylabel={flux},samples=60,width=\plotwidth,height=\plotheight,yticklabels={,,},xticklabels={,,},tick label style={major tick length=0pt},at=(phase.left of south west),anchor=right of south east,ylabel style={name=ylabel3},xshift=-1mm,visible on=<5->]
        \addplot[domain=0:4,autocatacyc,thick] {5*x/(1+x)};
        \addplot[domain=0:4,branchout,thick] {3*x/(0.1+x)};
        \addplot[domain=0:4,sumcolor,thick,visible on=<6->] {5*x/(1+x)-3*x/(0.1+x)};
        \addplot[dashed,gray,thick] coordinates {(1.25,0) (1.25,2.77)};
        \node[right,align=left,visible on=<6->] (threetext) at (axis cs:0.05,4.7) {\scriptsize unstable non-zero\\[-0.4em]\scriptsize steady state};
      \end{axis}
     \iftoggle{article} {}
     {
      \node[draw,fit=(plot3) (ylabel3),line width=2pt, fill=none,rounded corners=3pt,opacity=0.2, black!40,visible on=<7->]{};
  }

      \begin{axis}[name=plot4,axis x line=middle,axis y line=left,xlabel near ticks,ylabel near ticks,xmin=0,ymin=-2.5,xmax=2.9,ymax=5.9,xlabel={[$X$]},ylabel={flux},samples=60,at=(phase.right of south east),anchor=left of south west,width=\plotwidth,height=\plotheight,yticklabels={,,},xticklabels={,,},tick label style={major tick length=0pt},ylabel style={name=ylabel4},xshift=1mm,visible on=<5->]%,axis background/.style=kstyle]
        \addplot[domain=0:2.9,autocatacyc,thick] {5*x/(0.2+x)};
        \addplot[domain=0:2.9,branchout,thick] {4*x/(1+x)};
        \addplot[domain=0:2.9,sumcolor,thick,visible on=<6->] {5*x/(0.2+x)-4*x/(1+x)};
        \node[right,align=left,visible on=<6->] (fourtext) at (axis cs:0.05,5) {\scriptsize  no stable steady state};
     \end{axis}
     \iftoggle{article} {}
     {
      \node[draw,fit=(plot4) (ylabel4),line width=2pt, kstyle,fill=none,rounded corners=3pt]{};
     }

        \iftoggle{article} {
          \node [at=(plot2.north west),xshift=-0.6cm,yshift=0.35cm] (B) {(B)};
        }{}
    \end{scope}
  \end{tikzpicture}


            \end{adjustbox}
       \caption{\label{fig:simplecycle}
        (A) A simple autocatalytic cycle induces two fluxes, $f_a$ and $f_b$ as a function of the concentration of $X$.
        These fluxes follow simple Michaelis Menten kinetics.
        A steady state occurs when $f_a=f_b$, implying that $\dot{X}=0$.
        The cycle always has a steady state at $X=0$.
        The slope of each reaction at $X=0$ is $\sfrac{V_{\max}}{K_m}$.
        A steady state is stable if at the steady state concentration $\frac{d\dot{X}}{dX}<0$.
        (B) Each set of kinetic parameters, $V_{\max,a},V_{\max,b},K_{M,a},K_{M,b}$ determines two characteristics of the system: 
        If $V_{\max,b}>V_{\max,a}$, then a stable steady state concentration must exist as for high concentrations of $X$ the branching reaction will reduce its concentration (cyan domain, cases (I) and (II)).
        If $\sfrac{V_{\max,b}}{K_{M,b}}<\sfrac{V_{\max,a}}{K_{M,a}}$, implying that $\sfrac{V_{\max,b}}{V_{\max,a}}<\sfrac{K_{M,b}}{K_{M,a}}$, then zero is a non-stable steady state concentration as if $X$ is slightly larger than zero, the autocatalytic reaction will carry higher flux, further increasing the concentration of $X$ (magenta domain, cases (I) and (IV)).
    At the intersection of these two domains a non-zero, stable steady state concentration exists (I).}
    \end{figure}
    \textbf{Conclusion:} A steady state can only exist if $K_{M,b}>K_{M,a}$ and at the steady state, $f_b$ is not saturated.
        \end{block}
    \end{column}
    \hspace{-.75cm}
    \vrule
    \hspace{.3cm}
    \begin{column}{0.48\textwidth}
        \begin{block}{\centering{Input fluxes into autocatalytic cycles stabilize them}}
            \begin{figure}
            \begin{adjustbox}{max totalsize={\textwidth}{\textheight},center}
                \begin{tikzpicture}[>=latex',node distance = 2cm]
  \begin{scope}[]
        \node at (-60:1cm) (X) {$X$};
        \node[shape=coordinate] (orig) {};
        \draw [-,line width=1pt,autocatacyc] (X.south west) arc (285:0:1cm) node [pos=0.65,above] (fa) {$f_a:$\small{$E+X\rightarrow2X$}} node [pos=0.45,shape=coordinate] (midauto) {} node [pos=1,shape=coordinate] (endcommon) {};
        \draw [->,line width=1pt,autocatacyc] (endcommon) arc (-25:-44:2cm);
        \draw [->,line width=1pt,autocatacyc] (endcommon) arc (-5:-32:1.5cm);
        \draw [line width=1pt,assimcol] (midauto) arc (-60:-90:1cm) node [pos=1,left] (e) {$E$};
        \draw [->,line width=1pt,branchout] (X.south east) arc (225:270:1cm) node [pos=0.75,above] {$f_b$};
    \draw [<-,line width=1pt,magenta] (X.east) arc (-90:-30:1cm) node [pos=0.4,above] (fi) {$f_i$};
        \iftoggle{article} {
            \node at (-2.4cm,1.3cm) (A) {(A)};
        }{}
  \end{scope}
    \begin{customlegend}[legend entries={$f_a+f_i$,$f_b$,$\dot{X}=f_a+f_i-f_b$},legend style={right=3cm of orig,anchor=west,name=legend1}]
      \addlegendimage{autocatacyc,fill=black!50!red,sharp plot,line width=1pt}
      \addlegendimage{branchout,fill=black!50!red,sharp plot,line width=1pt}
      \addlegendimage{sumcolor,fill=black!50!red,sharp plot,dashed,line width=1pt}
    \end{customlegend}
    \begin{scope}[shift={(-2.5cm,-6cm)},node distance = 1cm]

        \begin{axis}[name=plot1,axis x line=middle,axis y line=left,xlabel near ticks,ylabel near ticks,xmin=0,ymin=-2.5,xmax=2.9,ymax=5.9,xlabel={[$X$]},ylabel={flux},samples=60,width=5cm,height=6cm,yticklabels={,,},xticklabels={,,},tick label style={major tick length=0pt}]
    \addplot[domain=0:4,autocatacyc,thick] {3*x/(0.5+x)+\influx};
    \addplot[domain=0:4,branchout,thick] {5*x/(1+x)};
    \addplot[domain=0:4,sumcolor,thick,dashed] {3*x/(0.5+x)-5*x/(1+x)+\influx};
    \addplot[dashed,gray,thick] coordinates {(1,0) (1,2.5)};
    \node[right] (one) at (axis cs:0.1,5.5) {I};
    \node[right,align=left] (onetext) at (axis cs:0.05,4.7) {\scriptsize stable non-zero\\[-0.4em]\scriptsize steady state};
  \end{axis}

  \begin{axis}[name=plot2,axis x line=middle,axis y line=left,xlabel near ticks,ylabel near ticks,xmin=0,ymin=-2.5,xmax=2.9,ymax=5.9,xlabel={[$X$]},ylabel={flux},samples=60,at=(plot1.right of south east),anchor=left of south west,width=5cm,height=6cm,yticklabels={,,},xticklabels={,,},tick label style={major tick length=0pt}]
    \addplot[domain=0:4,autocatacyc,thick] {5*x/(1+x)+\influx};
    \addplot[domain=0:4,branchout,thick] {4*x/(1+x)};
    \addplot[domain=0:4,sumcolor,thick,dashed] {5*x/(1+x)-4*x/(1+x)+\influx};
    \node[right] (two) at (axis cs:0.1,5.5) {II};
    \node[right,align=left] (twotext) at (axis cs:0.05,5) {\scriptsize no steady state};
  \end{axis}

  \begin{axis}[name=plot3,axis x line=middle,axis y line=left,xlabel near ticks,ylabel near ticks,xmin=0,ymin=-2.5,xmax=2.9,ymax=5.9,xlabel={[$X$]},ylabel={flux},samples=60,at=(plot2.right of south east),anchor=left of south west,width=5cm,height=6cm,yticklabels={,,},xticklabels={,,},tick label style={major tick length=0pt}]
    \addplot[domain=0:4,autocatacyc,thick] {6*x/(2+x)+1.5*\influx};
    \addplot[domain=0:4,branchout,thick] {4*x/(0.4+x)};
    \addplot[domain=0:4,sumcolor,thick,dashed] {6*x/(2+x)-4*x/(0.4+x)+1.5*\influx};
    \addplot[dashed,gray,thick] coordinates {(1.2,0) (1.2,3)};
    \addplot[dashed,gray,thick] coordinates {(0.182,0) (0.182,1.25)};
    \node[right] (three) at (axis cs:0.1,5.5) {III};
    \node[right,align=left] (threetext) at (axis cs:0.05,4.7) {\scriptsize two non-zero\\[-0.4em]\scriptsize steady states};
  \end{axis}
    \end{scope}
\end{tikzpicture}


            \end{adjustbox}
      \caption{\label{fig:inputcycle}
        (A) The effect of a fixed input flux, $f_i$, on the possible steady states of a simple autocatalytic cycle.
        A steady state occurs when $f_a+f_i=f_b$.
        If $V_{\max,b}>V_{\max,a}+f_i$ then there is always a single stable steady state (I).
        If $V_{\max,b}<V_{\max,a}+f_i$ then there can either be no steady states (II), or two steady states where the smaller one is stable (III).
        Therefore, an input flux relaxes the constraints on the kinetic parameters of $f_b$.
        We find reactions catalyzing such fluxes in the form of alternative glucose transport mechanisms for the PTS autocatalytic cycle and in seemingly redundant reactions in the Pentose-Phosphate autocatalytic cycle.
      }
    \end{figure}

    \end{block}
    \begin{block}{\centering{Stability of multiple-reactions autocatalytic cycles also imposes constraints on kinetic parameters of branch enzymes}}
        \begin{figure}
            \begin{columns}
            \column{0.66\linewidth}
                \caption{
             Steady state stability of multiple-reaction autocatalytic cycles with a single assimilating reaction is analytically tractable.
             The system is at steady state when the total consumption of intermediate metabolites by the branch reactions is equal to the flux through the autocatalytic reaction.
             A sufficient condition for the stability of a steady state in these systems is that the derivative of at least one branch reaction W.R.T. the substrate concentration is larger than the derivative of the equivalent autocatalytic reaction at the steady state concentration.
                \label{fig:multiple}}
            \column{0.25\linewidth}
                \begin{adjustbox}{max totalsize={0.7\textwidth}{\textheight},center}
                  \begin{tikzpicture}[>=latex']
    \iftoggle{poster} {}
    {
    \iftoggle{article} {
    \begin{scope}[shift={(-5cm,0cm)},node distance = 2cm]
        \node (X1) {$X_1$};
        \node[left=of X1]  (X2) {$X_2$};
        \draw [->,line width=1pt,autocatacyc] (X1.south) [out=-90,in=-90] to  node [pos=0.5,above] (fa1) {$f_{a_1}$} (X2.south);
        \draw [->,line width=1pt,autocatacyc] (X2.north) [out=90,in=90] to node [pos=0.5,shape=coordinate,yshift=0.5pt] (assimpt) {} node [pos=0.6,above] (fa2) {$f_{a_2}$} (X1.north);
        \draw [line width=1pt,assimcol] (assimpt) arc (-90:-140:0.7cm) node [pos=1,above] (e) {$A$};
        \draw [->,line width=1pt,branchout] (X1.south) arc (190:270:1cm) node [pos=0.3,right,xshift=1mm] {$f_{b_1}$};
        \draw [->,line width=1pt,branchout] (X2.north) arc (10:90:1cm) node [pos=0.3,left] {$f_{b_2}$};
    \end{scope}}
    {}
}
    \begin{scope}[shift={(0cm,0cm)},node distance = 1cm]
        \node (X1) {$X_1$};
        \node[below left=of X1]  (X2) {$X_2$};
        \node[above left=of X1]  (Xn) {$X_n$};
        \draw [->,line width=1pt,autocatacyc] (X1.south) [out=-90,in=0] to  node [pos=0.7,right,xshift=1mm] (fa1) {$f_{a_1}$} (X2.east);
        \draw [->,line width=1pt,autocatacyc] (Xn.east) [out=0,in=90] to node [pos=0.4,shape=coordinate] (assimpt) {} node [pos=0.4,right,xshift=1mm] (fan) {$f_{a_n}$}(X1.north);
        \draw [line width=1pt,assimcol] (assimpt) arc (-120:-170:0.7cm) node [pos=1,above] (e) {$A$};
        \draw [->,line width=1pt,branchout] (X1.south) arc (190:270:1cm) node [pos=0.3,right,xshift=1mm] {$f_{b_1}$};
        \draw [->,line width=1pt,branchout] (X2.south) arc (-10:-90:1cm) node [pos=0.3,left] {$f_{b_2}$};
        \draw [->,line width=1pt,branchout] (Xn.north) arc (10:90:1cm) node [pos=0.3,left] {$f_{b_n}$};
        \draw [->,line width=1pt,autocatacyc,dashed] (X2.west) [out=180,in=-90] to ($(X1)+(-3,0)$) node [right] (fmid) {$f_{a_2}\dots f_{a_{n-1}}$} to [out=90,in=180] (Xn.west);
    \end{scope}
    \iftoggle{poster} {}
    {
    \iftoggle{article} {
        \node [shift={(-8cm,3cm)}] (A) {(A)};
        \node [right of=A,xshift=4cm] (B) {(B)};
    }{}
}
\end{tikzpicture}


                \end{adjustbox}
            \end{columns}
        \end{figure}
\textbf{Conclusion:} As the derivatives of fluxes and saturation levels of reactions are inter-related, the stability condition implies that at a stable steady state the saturation level of at least one branch reactions is smaller than the saturation level of the corresponding autocatalytic reaction.

    \vskip2ex
        \end{block}
        \begin{block}{\centering{Experimental data reveals operating autocatalytic cycles obay predicted restrictions}}
\begin{figure}
            \begin{adjustbox}{max totalsize={\textwidth}{\textheight},center}
              \begin{tikzpicture}
  \tikzset{
    figArrowStyle/.style={arrows={-{Stealth[inset=0pt,scale=#1,angle'=60]}}},
    figArrowStyle/.default=0.25
  }
  \tikzset{
    capArrowStyle/.style={arrows={-{Stealth[inset=0pt,scale=0.25,angle'=60,color=#1]}}},
    capArrowStyle/.default=autocatacycfl
    }

  \tikzset{
    ratioRect/.style={rectangle,fill=graybg,rounded corners=2pt}}

  \newcommand{\coloredRatio}[2]}{\mathbf{\color{autocatacyc}#2\%}}$}}
  }

\iftoggle{article} {
    \pgfmathsetlength{\nodedist}{1cm}
    \renewcommand{\fontsizedef}{\normalsize}
    \renewcommand{\ratiosizedef}{\Large}
    \renewcommand{\coloredRatio}[2]}{\mathbf{\color{autocatacyc}#2\%}}$}}
  }
}{
    \only<3-> {
        \pgfmathsetlength{\nodedist}{1cm}
        \renewcommand{\fontsizedef}{\normalsize}
        \renewcommand{\ratiosizedef}{\Large}
        \renewcommand{\coloredRatio}[2]}{\mathbf{\color{autocatacyc}##2\%}}$}}
      }
    }
}

    \pgfmathsetlength{\headlinedist}{0.3cm}
    %prediction
  \begin{scope}[shift={(-6cm,0cm)}]
      \node[ratioRect] (prediction) {\textbf{Prediction:} $\mathbf{\color{branchout}XX\%}\leq\mathbf{\color{autocatacyc}YY\%}$};
  \end{scope}
  %Galactose
  \begin{scope}[shift={(-6cm,-7.5cm)},visible on=<3->,font=\fontsizedef]
    \def\galmaxflux{0.5mm}
    \def\galglt{1.52*\galmaxflux}
    \def\galcapvalglt{20}
    \def\galcapglt{\galglt/\galcapvalglt*100}
    \def\galacn{1.52*\galmaxflux*1.5}
    \def\galacea{1.02*\galmaxflux}
    \def\galcapvalacea{30}
    \def\galcapacea{\galacea/\galcapvalacea*100}
    \def\galaceb{1.02*\galmaxflux}
    \def\galsdh{1.26*\galmaxflux}
    \def\galfum{1.26*\galmaxflux}
    \def\galmdh{2.28*\galmaxflux}
    \def\galpck{0.85*\galmaxflux}
    \def\galcapvalpck{25}
    \def\galcappck{\galpck/\galcapvalpck*100}
    \def\galicd{0.5*\galmaxflux}
    \def\galcapvalicd{10}
    \def\galcapicd{\galicd/\galcapvalicd*100}

    \node[] (galactose) {\textbf{Galactose input}};
    \node[metaboliteStyle,inputcol,below=\headlinedist of galactose] (aca) {aca};
    \node[shape=coordinate,below=of aca] (dummyglta) {};
    \node[metaboliteStyle,left=of dummyglta] (oaa) {oaa};
    \node[metaboliteStyle] at (aca -| oaa) (pep) {pep};
    \node[metaboliteStyle,right=of dummyglta] (cit) {cit};
    \node[metaboliteStyle,right=of cit] (icit) {icit};
    \node[metaboliteStyle,below=of icit.center] (akg) {akg};
    \node[metaboliteStyle,below=of akg.center] (sca) {sca};
    \node[metaboliteStyle,below=of oaa.center] (mal) {mal};
    \node[metaboliteStyle,below=of mal.center] (fum) {fum};
    \node[metaboliteStyle,right=of mal] (glx) {glx};
    \node[metaboliteStyle,right=of fum] (suc) {suc};
    \path[] (oaa) -- (cit) node [pos=0.85,shape=coordinate] (midglta) {};
    \draw[line width=\galcapglt,autocatacyc!40] ([yshift=-0.35*\galglt]oaa.east) -- ([yshift=-0.35*\galglt]midglta) ;

    \node[ratioRect,anchor=east] at (oaa.west) (galb1) {\coloredRatio{\galcapvalpck}{\galcapvalglt}};
    \node[ratioRect,anchor=south] at (icit.north) (galb2) {\coloredRatio{\galcapvalicd}{\galcapvalacea}};

    \draw[line width=\galglt,autocatacycfl] ([yshift=-0.35*\galglt]oaa.east) -- ([yshift=-0.35*\galglt]midglta);
    \draw[figArrowStyle,line width=\galglt*1.5,autocatacycfl] (midglta) -- (cit);
    \draw[capArrowStyle=branchoutfl,line width=\galcappck,branchout!40] (oaa) -- (pep);
    \draw[figArrowStyle,line width=\galpck,branchoutfl] (oaa) -- (pep);
    \draw [inputcol,line width=0.5*\galglt] (aca) [out=-70,in=180] to ([yshift=0.35*\galglt]midglta);
    \draw[figArrowStyle,line width=\galacn,autocatacycfl] (cit) -- (icit);
    \path[] (icit.south west) -- (suc) node [pos=0.3,shape=coordinate] (midacea) {};
    \draw[line width=\galcapacea*1.5,autocatacyc!40] (icit.south west) -- (midacea);
    \draw[capArrowStyle,line width=\galcapacea,autocatacyc!40] ([xshift=\galcapacea*0.35]midacea) -- ([xshift=\galcapacea*0.4]suc);
    \draw[capArrowStyle,line width=\galcapacea,autocatacyc!40] ([xshift=\galcapacea*0.38]midacea) -- ([xshift=\galcapacea*0.4]suc);
    \draw[capArrowStyle,line width=\galcapacea*0.5,autocatacyc!40] ([shift={(-\galcapacea*0.35,\galcapacea*0.35)}]midacea) [out=220,in=0] to (glx);
    \draw[capArrowStyle,line width=\galcapacea*0.5,autocatacyc!40] ([shift={(-\galacea*0.35,\galacea*0.35)}]midacea) [out=220,in=0] to (glx);
    \draw[line width=\galacea*1.5,autocatacycfl] (icit.south west) -- (midacea);
    \draw[figArrowStyle,line width=\galacea,autocatacycfl] ([xshift=\galacea*0.4]midacea) -- ([xshift=\galacea*0.4]suc);
    \draw[figArrowStyle,line width=\galacea*0.5,autocatacycfl] ([shift={(-\galacea*0.35,\galacea*0.35)}]midacea) [out=220,in=0] to (glx);
    \draw[capArrowStyle=branchoutfl,line width=\galcapicd*1.5,branchout!40] (icit) -- (akg);
    \draw[figArrowStyle,line width=\galicd*1.5,branchoutfl] (icit) -- (akg);
    \draw[->] (akg) -- (sca);
    \draw[->] (sca) -- (suc);
    \draw[figArrowStyle,line width=\galsdh,autocatacycfl] (suc) -- (fum);
    \draw[figArrowStyle,line width=\galfum,autocatacycfl](fum) -- (mal);
    \path[] (glx) -- (mal) node [pos=0.85,shape=coordinate] (midaceb) {};
    \draw[line width=\galaceb*0.5,autocatacycfl] ([yshift=-0.25*\galaceb]glx) -- ([yshift=-0.25*\galaceb]midaceb);
    \draw[inputcol,line width=0.5*\galaceb] ([xshift=-0.5mm]aca.south) [out=-90,in=0] to ([yshift=0.25*\galaceb]midaceb);
    \draw[figArrowStyle,line width=\galaceb,autocatacycfl] (midaceb) -- (mal);
    \draw[figArrowStyle,line width=\galmdh,autocatacycfl] (mal) -- (oaa);
  \end{scope}

  %acetate
  \begin{scope}[shift={(-6cm,-1.5cm)},node distance=\nodedist,font=\fontsizedef]
    \def\acemaxflux{0.2mm}
    \def\aceglt{8.83*\acemaxflux}
    \def\acecapvalglt{70}
    \def\acecapglt{\aceglt/\acecapvalglt*100}
    \def\aceacn{8.83*\acemaxflux*1.5}
    \def\aceacea{4.14*\acemaxflux}
    \def\acecapvalacea{100}
    \def\acecapacea{\aceacea}
    \def\aceaceb{4.14*\acemaxflux}
    \def\acesdh{8.4*\acemaxflux}
    \def\acefum{8.4*\acemaxflux}
    \def\acemdh{10.67*\acemaxflux}
    \def\acecapvalmdh{100}
    \def\acemae{1.87*\acemaxflux}
    \def\acecapvalmae{15}
    \def\acecapmae{\acemae/\acecapvalmae*100}
    \def\acepck{3.11*\acemaxflux}
    \def\acecapvalpck{75}
    \def\acecappck{\acepck/\acecapvalpck*100}
    \def\aceicd{4.7*\acemaxflux}
    \def\acecapvalicd{65}
    \def\acecapicd{\aceicd/\acecapvalicd*100}

    \node[] (acetate) {\textbf{Acetate input}};
    \node[metaboliteStyle,inputcol,below=\headlinedist of acetate] (aca) {aca};
    \node[shape=coordinate,below=of aca] (dummyglta) {};
    \node[metaboliteStyle,left=of dummyglta] (oaa) {oaa};
    \node[metaboliteStyle] at (aca -| oaa) (pep) {pep};
    \node[metaboliteStyle,right=of dummyglta] (cit) {cit};
    \node[metaboliteStyle,right=of cit] (icit) {icit};
    \node[metaboliteStyle,below=of icit.center] (akg) {akg};
    \node[metaboliteStyle,below=of akg.center] (sca) {sca};
    \node[metaboliteStyle,below=of oaa.center] (mal) {mal};
    \node[metaboliteStyle,below=of mal.center] (fum) {fum};
    \node[metaboliteStyle,right=of mal] (glx) {glx};
    \node[metaboliteStyle,left=of mal] (pyr) {pyr};
    \node[metaboliteStyle,right=of fum] (suc) {suc};
    \path[] (oaa) -- (cit) node [pos=0.85,shape=coordinate] (midglta) {};
    \draw[line width=\acecapglt,autocatacyc!40] ([yshift=-0.25*\aceglt]oaa.east) -- ([yshift=-0.25*\aceglt]midglta);
    \node[ratioRect,anchor=south east,visible on=<2->] at (oaa.north west) (aceb1) {\coloredRatio{\acecapvalpck}{\acecapvalglt}};
    \draw[line width=\aceglt,autocatacycfl] ([yshift=-0.25*\aceglt]oaa.east) -- ([yshift=-0.25*\aceglt]midglta);
    \draw[figArrowStyle,line width=\aceglt*1.5,autocatacycfl] (midglta) -- (cit);
    \draw[capArrowStyle=branchoutfl,line width=\acecappck,branchout!40] (oaa) -- (pep);
    \draw[figArrowStyle,line width=\acepck,branchoutfl] (oaa) -- (pep);
    \draw[capArrowStyle=branchoutfl,line width=\acecapmae,branchout!40] (mal) -- (pyr);
    \node[ratioRect,anchor=south east,visible on=<2->] at (mal.north west) (aceb2) {\coloredRatio{\acecapvalmae}{\acecapvalmdh}};
    \draw[figArrowStyle,line width=\acemae,branchoutfl] (mal) -- (pyr);
    \draw [inputcol,line width=0.5*\aceglt] (aca) [out=-70,in=180] to ([yshift=0.4*\aceglt]midglta);
    \draw[figArrowStyle,line width=\aceacn,autocatacycfl] (cit) -- (icit);
    \path[] (icit.south west) -- (suc) node [pos=0.3,shape=coordinate] (midacea) {};
    \draw[line width=\acecapacea*1.5,autocatacyc!40] (icit.south west) -- (midacea);
    \draw[capArrowStyle,line width=\acecapacea,autocatacycfl!40] ([xshift=\acecapacea*0.35]midacea) -- ([xshift=\acecapacea*0.4]suc);
    \node[ratioRect,anchor=south,visible on=<2->] at (icit.north) (aceb3) {\coloredRatio{\acecapvalicd}{\acecapvalacea}};
    \draw[capArrowStyle,line width=\acecapacea*0.5,autocatacyc!40] ([shift={(-\acecapacea*0.35,\acecapacea*0.35)}]midacea) [out=220,in=0] to (glx);
    \draw[line width=\aceacea*1.5,autocatacycfl] (icit.south west) -- (midacea);
    \draw[figArrowStyle,line width=\aceacea,autocatacycfl] ([xshift=\aceacea*0.4]midacea) -- ([xshift=\aceacea*0.4]suc);
    \draw[figArrowStyle,line width=\aceacea*0.5,autocatacycfl] ([shift={(-\aceacea*0.35,\aceacea*0.35)}]midacea) [out=220,in=0] to (glx);
    \draw[capArrowStyle=branchoutfl,line width=\acecapicd*1.5,branchout!40] (icit) -- (akg);
    \draw[figArrowStyle,line width=\aceicd*1.5,branchoutfl] (icit) -- (akg);
    \draw[->] (akg) -- (sca);
    \draw[->] (sca) -- (suc);
    \draw[figArrowStyle,line width=\acesdh,autocatacycfl] (suc) -- (fum);
    \draw[figArrowStyle,line width=\acefum,autocatacycfl](fum) -- (mal);
    \path[] (glx) -- (mal) node [pos=0.85,shape=coordinate] (midaceb) {};
    \draw[line width=\aceaceb*0.5,autocatacycfl] ([yshift=-0.25*\aceaceb]glx) -- ([yshift=-0.25*\aceaceb]midaceb);
    \draw[inputcol,line width=0.5*\aceaceb] ([xshift=-0.5mm]aca.south) [out=-90,in=0] to ([yshift=0.25*\aceaceb]midaceb);
    \draw[figArrowStyle,line width=\aceaceb,autocatacycfl] (midaceb) -- (mal);
    \draw[figArrowStyle,line width=\acemdh,autocatacycfl] (mal) -- (oaa);
 
  \end{scope}

  %Glucose
  \begin{scope}[shift={(2cm,0cm)},visible on=<3->,font=\fontsizedef]
    \def\glucmaxflux{0.35mm}
    \def\glucpgi{5.7*\glucmaxflux}
    \def\gluccapvalpgi{100}
    \def\gluccappgi{\glucpgi}
    \def\glucpfk{7.06*\glucmaxflux}
    \def\glucfba{7.06*\glucmaxflux}
    \def\gluctpi{7.06/2*\glucmaxflux}
    \def\glucgap{15.71/2*\glucmaxflux}
    \def\glucpgk{15.71/2*\glucmaxflux}
    \def\glucgpm{14.56/2*\glucmaxflux}
    \def\gluceno{14.56/2*\glucmaxflux}
    \def\glucpyk{2.49/2*\glucmaxflux}
    \def\gluccapvalpyk{35}
    \def\gluccappyk{\glucpyk/\gluccapvalpyk*100}
    \def\glucppc{2.45/2*\glucmaxflux}
    \def\gluccapvalppc{45}
    \def\gluccapppc{\glucppc/\gluccapvalppc*100}
    \def\gluczwf{3.92*\glucmaxflux}
    \def\gluccapvalzwf{55}
    \def\gluccapzwf{\gluczwf/\gluccapvalzwf*100}
    \def\glucpts{9.65*\glucmaxflux}
    \def\gluccapvalpts{100}
    \def\gluccapvalcomp{\pgfmathparse{round((\glucppc+\glucpyk)/(\gluccapppc+\gluccappyk)*20)*5}\pgfmathprintnumber{\pgfmathresult}}
    \def\gluccappts{\glucpts}

    \node[] (glucose) {\textbf{Glucose input}};
    \node[metaboliteStyle,below=\headlinedist of glucose] (g6p) {g6p};
    \node[metaboliteStyle,below=of g6p.center] (f6p) {f6p};
    %%% ptstop
    \node[shape=coordinate,left=1.1cm of g6p] (ptsmid) {};
    \node[metaboliteStyle,shift={(-11mm,2mm)},gray] at (f6p) (pyr1) {pyr};
    \node[metaboliteStyle,inputcol,left=of ptsmid] (gluc) {gluc};

    \node[metaboliteStyle,below=of f6p] (fbp) {fbp};
    \node[shape=coordinate,below=of fbp.center](fbamid) {};
    \node[metaboliteStyle,below left=of fbamid.center] (dhap) {dhap};
    \node[metaboliteStyle,below right=of fbamid] (gap) {gap};
    \node[metaboliteStyle,below=of gap.center] (bpg) {bpg};
    \node[metaboliteStyle,below=of bpg.center] (3pg) {3pg};
    \node[metaboliteStyle,below=of 3pg.center] (2pg) {2pg};
    \node[metaboliteStyle,below=of 2pg.center] (pep) {pep};
    \node[metaboliteStyle,right=of pep] (oaa) {oaa};
    \node[metaboliteStyle,below=of pep.center] (pyr) {pyr};
    \node[metaboliteStyle,right=of g6p] (6pgi) {6pgi};
    \draw[figArrowStyle,line width=\glucpgi,autocatacycfl] (g6p) -- (f6p);
    \draw[figArrowStyle,line width=\glucpfk,autocatacycfl] (f6p.south) -- (fbp.north);
    \draw [line width=\glucfba,autocatacycfl] (fbp) [out=-90,in=90] to (fbamid);
    \draw [figArrowStyle,line width=\glucfba/2,autocatacycfl] ([xshift=-\glucfba/4]fbamid) [out=-90,in=45] to (dhap);
    \draw [figArrowStyle,line width=\glucfba/2,autocatacycfl] ([xshift=\glucfba/4]fbamid) [out=-90,in=135] to (gap);
    \draw [figArrowStyle,line width=\gluctpi,autocatacycfl] (dhap) -- (gap);

    \draw[figArrowStyle,line width=\glucgap,autocatacycfl] (gap) -- (bpg);
    \draw[figArrowStyle,line width=\glucpgk,autocatacycfl] (bpg) -- (3pg);
    \draw[figArrowStyle,line width=\glucgpm,autocatacycfl] (3pg) -- (2pg);
    \draw[figArrowStyle,line width=\gluceno,autocatacycfl] (2pg) -- (pep);
    \draw[capArrowStyle=branchoutfl,line width=\gluccappyk,branchout!40] (pep) -- (pyr);
    \draw[figArrowStyle,line width=\glucpyk,branchoutfl] (pep) -- (pyr);
    \draw[capArrowStyle=branchoutfl,line width=\gluccapppc,branchout!40] (pep) -- (oaa);
    \node[ratioRect,anchor=south east] at (pep.north west) (glucb2) {\coloredRatio{\gluccapvalcomp}{\gluccapvalpts}};
    \draw[figArrowStyle,line width=\glucppc,branchoutfl] (pep) -- (oaa);
    \draw[capArrowStyle=branchoutfl,line width=\gluccapzwf,branchout!40] (g6p) -- (6pgi);
    \draw[figArrowStyle,line width=\gluczwf,branchoutfl] (g6p) -- (6pgi);
    \node[ratioRect,anchor=north west] at (g6p.south east) (glucb1) {\coloredRatio{\gluccapvalzwf}{\gluccapvalpgi}};
    \node[shape=coordinate,left=2.5cm of pep.center] (pts3) {};
    \draw[line width=\glucpts/2,autocatacycfl] (pep.west) -- (pts3);
    \draw[inputcol,line width=\glucpts] (gluc) -- (ptsmid);
    \node[shape=coordinate,left=2.5cm of pyr.center] (pts5) {};
    \draw[figArrowStyle,line width=\glucpts/2,autocataby] ([yshift=-3/4*\glucpts]ptsmid) [out=0,in=90] to (pyr1);
    \node[shape=coordinate,xshift=-2mm] at (f6p.center -| dhap.west) (ptstop) {};
    \node[shape=coordinate] at(ptstop |- 2pg.center) (ptsbottom) {};
    \draw[line width=\glucpts/2,autocatacycfl] (pts3) [in=-90,out=180] to (ptsbottom);
    \draw[line width=\glucpts/2,autocatacycfl] (ptsbottom) -- (ptstop);
    \draw[line width=\glucpts/2,autocatacycfl] (ptstop) [in=180,out=90] to ([yshift=-3/4*\glucpts]ptsmid);
    \draw[figArrowStyle,line width=\glucpts,autocatacycfl] (ptsmid) -- (g6p);
  \end{scope}

  %Fructose
  \begin{scope}[shift={(8.5cm,0cm)},visible on=<3->,font=\fontsizedef]
    \def\frucmaxflux{0.35mm}
    \def\frucfbp{2.46*\frucmaxflux}
    \def\fruccapvalfbp{100}
    \def\fruccapfbp{\frucfbp}
    \def\frucfba{5.87*\frucmaxflux}
    \def\fruccapvalfba{70}
    \def\fruccapfba{\frucfba/\fruccapvalfba*100}
    \def\fructpi{5.87/2*\frucmaxflux}
    \def\frucgap{13.46/2*\frucmaxflux}
    \def\frucpgk{13.46/2*\frucmaxflux}
    \def\frucgpm{12.6/2*\frucmaxflux}
    \def\fruceno{12.6/2*\frucmaxflux}
    \def\frucpyk{0.67/2*\frucmaxflux}
    \def\fruccapvalpyk{5}
    \def\fruccappyk{\frucpyk/\fruccapvalpyk*100}
    \def\frucppc{3.55/2*\frucmaxflux}
    \def\fruccapvalppc{50}
    \def\fruccapppc{\frucppc/\fruccapvalppc*100}
    \def\frucpts{8.33*\frucmaxflux}
    \def\fruccapvalpts{100}
    \def\fruccappts{\frucpts}
    \def\fruccapvalcomp{\pgfmathparse{round((\frucppc+\frucpyk)/(\fruccapppc+\fruccappyk)*20)*5}\pgfmathprintnumber{\pgfmathresult}}


    \node[] (fructose) {\textbf{Fructose input}};
    \node[metaboliteStyle,below=\headlinedist of fructose] (f6p) {f6p};
    \node[metaboliteStyle,below=of f6p] (fbp) {fbp};
    \node[shape=coordinate,below=of fbp.center](fbamid) {};
    %%% ptstop
    \node[shape=coordinate,left=1.1cm of fbp] (ptsmid) {};
    \node[metaboliteStyle,shift={(-11mm,0mm)},gray] at (fbamid) (pyr1) {pyr};
    \node[metaboliteStyle,inputcol,left=of ptsmid] (fruc) {fruc};

    \node[metaboliteStyle,below left=of fbamid.center] (dhap) {dhap};
    \node[metaboliteStyle,below right=of fbamid] (gap) {gap};
    \node[metaboliteStyle,below=of gap.center] (bpg) {bpg};
    \node[metaboliteStyle,below=of bpg.center] (3pg) {3pg};
    \node[metaboliteStyle,below=of 3pg.center] (2pg) {2pg};
    \node[metaboliteStyle,below=of 2pg.center] (pep) {pep};
    \node[metaboliteStyle,right=of pep] (oaa) {oaa};
    \node[metaboliteStyle,below=of pep.center] (pyr) {pyr};
    \draw[figArrowStyle,line width=\frucfbp,branchoutfl] (fbp)-- (f6p) ;
    \node[ratioRect,anchor=west] at (fbp.east) (frucb1) {\coloredRatio{\fruccapvalfbp}{\fruccapvalfba}};
    \node[at=(frucb1.north east)] (star) {*};
    \draw [line width=\fruccapfba,autocatacyc!40] (fbp) -- (fbamid);
    \draw [capArrowStyle,line width=\fruccapfba/2,autocatacyc!40] ([xshift=-\fruccapfba/4]fbamid) [out=-90,in=45] to (dhap);
    \draw [capArrowStyle,line width=\fruccapfba/2,autocatacyc!40] ([xshift=\fruccapfba/4]fbamid) [out=-90,in=135] to (gap);
    \draw [line width=\frucfba,autocatacycfl] (fbp) [out=-90,in=90] to (fbamid);
    \draw [figArrowStyle,line width=\frucfba/2,autocatacycfl] ([xshift=-\frucfba/4]fbamid) [out=-90,in=45] to (dhap);
    \draw [figArrowStyle,line width=\frucfba/2,autocatacycfl] ([xshift=\frucfba/4]fbamid) [out=-90,in=135] to (gap);
    \draw [figArrowStyle,line width=\fructpi,autocatacycfl] (dhap) -- (gap);

    \draw[figArrowStyle,line width=\frucgap,autocatacycfl] (gap) -- (bpg);
    \draw[figArrowStyle,line width=\frucpgk,autocatacycfl] (bpg) -- (3pg);
    \draw[figArrowStyle,line width=\frucgpm,autocatacycfl] (3pg) -- (2pg);
    \draw[figArrowStyle,line width=\fruceno,autocatacycfl] (2pg) -- (pep);
    \draw[capArrowStyle=branchoutfl,line width=\fruccappyk,branchout!40] (pep) -- (pyr);
    \draw[figArrowStyle,line width=\frucpyk,branchoutfl] (pep) -- (pyr);
    \draw[capArrowStyle=branchoutfl,line width=\fruccapppc,branchout!40] (pep) -- (oaa);
    \node[ratioRect,anchor=south east] at (pep.north west) (frucb2) {\coloredRatio{\fruccapvalcomp}{\fruccapvalpts}};
    \draw[figArrowStyle,line width=\frucppc,branchoutfl] (pep) -- (oaa);
    \node[shape=coordinate,left=2.5cm of pep.center] (pts3) {};
    \draw[line width=\frucpts/2,autocatacycfl] ([yshift=\frucpts/4]pep.west) -- (pts3);
    \node[shape=coordinate,left=2.5cm of pyr.center] (pts5) {};
    \draw[figArrowStyle,line width=\frucpts/2,autocataby] ([yshift=-3/4*\frucpts]ptsmid) [out=0,in=90] to (pyr1);
    \node[shape=coordinate,xshift=-2mm] at (fbamid -| dhap.west) (ptstop) {};
    \node[shape=coordinate] at(ptstop |- 2pg.center) (ptsbottom) {};
    \draw[] (pts3) [in=-90,out=180] to (ptsbottom);
    \draw[] (ptsbottom) [in=-90,out=90] to (ptstop);
    \draw[inputcol,line width=\frucpts] (fruc) -- (ptsmid);
    \draw[line width=\frucpts/2,autocatacycfl] (pts3) [in=-90,out=180] to (ptsbottom);
    \draw[line width=\frucpts/2,autocatacycfl] (ptsbottom) -- (ptstop);
    \draw[line width=\frucpts/2,autocatacycfl] (ptstop) [in=180,out=90] to ([yshift=-3/4*\frucpts]ptsmid);
    \draw[figArrowStyle,line width=\frucpts,autocatacycfl] (ptsmid) -- (fbp);
  \end{scope}
  \end{tikzpicture}


            \end{adjustbox}
\caption{
  Major branch points and relative enzyme saturation in operating autocatalytic cycles.
  Solid arrow width represents carbon flux per unit time.
  Shaded arrow width represents maximal carbon flux capacity per unit time, given the expression level of the catalyzing enzyme.
  In all cases there is enough excess capacity in the branching reactions to prevent the cycle from overflowing.
  Only in one out of the 9 branch points observed (the branch point at fbp in growth under fructose, marked with *), the outgoing reaction is significantly more saturated than the autocatalytic reaction, which may result from neglecting fructose transport directly as f6p when deriving fluxes.
}
    \label{fig:branch}
\end{figure}

        \end{block}
    \end{column}
\end{columns}
\begin{block}{References}
    \vskip1ex
    \footnotesize
\bibliographystyle{ieeetr}
\bibliography{library}
\end{block}
\end{frame}

\end{document}
