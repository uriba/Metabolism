\section{Introduction}
    An essential trait of a living system is its ability to reproduce.
    This fundamental ability makes all living systems autocatalytic by definition.
    It then follows that any essential component of a living system is autocatalytic, if it is required for reproduction, because  such a component is being used in the reproduction process that leads to the duplication of it.
    Primitive autocatalytic metabolism is therefore considered to be one of the essential components of life \cite{Ganti2003-ch}.
    The metabolic system of any modern living organism, composed of the intermediate metabolites and enzymes catalyzing its reactions, is autocatalytic as a whole and must enable both the excess production of each of its metabolites, as well as the production of all the precursor molecules needed in order for the translation-transcription machinery to synthesize new copies of its enzymes.

    Even though the metabolic system of an organism is part of the autocatalytic process of reproduction, this does not imply that it is composed of smaller autocatalytic components.
    The metabolic system is composed of two classes of components, enzymes and metabolites.
    Because the metabolic system lacks the transcription-translation machinery, it does not produce enzymes.
    Therefore, while enzymes are an essential part of the metabolic system, they are not an autocatalytic component of this system.
    Regarding intermediate metabolites, the metabolic system evidently produces them, but it is not a priori evident that specific intermediate metabolites must exist in order for the metabolic system to metabolize external nutrients.
    Theoretically, all the intermediate metabolites of a metabolic system could be generated directly from external nutrients during catabolism, not requiring existing internal metabolite pools.
    However, and despite the essentiality of functioning metabolic networks, such networks often do require the availability of certain intermediate metabolites, in addition to the nutrients consumed, in order to function.
    The most common examples of obligatorily autocatalytic internal metabolites are ATP and co-factors \cite{Kun2008-xg}.
    The requirement for availability of certain metabolites in order to consume nutrients (also referred to as turbo design \cite{Teusink1998-he}) implies metabolic processes must be finely controlled to prevent such essential metabolites from running out, as in such cases metabolism will stop and the organism will die.
    Therefore, from a robustness point of view, autocatalytic cycles seem to be a weakness in metabolism as they confer constraints on metabolite levels that would not have existed in the absence of autocatalysis.
    The weaknesses and constraints autocatalysis imposes have been identified and modeled in the context of ATP activation in glycolysis \cite{Teusink1998-he}.
    On the other hand, autocatalytic cycles are, in specific cases, optimal network topologies that minimize the number of reactions needed for the production of precursor molecules from different nutrients \cite{Riehl2010-yh}.

    In this work we focus on autocatalytic cycles in chemical reaction systems, and specifically in metabolic networks.
    The components we consider are the metabolites of the system, with autocatalytic cycles being formed using the reactions of the metabolic network.
    An illustrative example for a metabolic autocatalytic cycle is glycolysis.
    In glycolysis, 2 ATP molecules are consumed in the priming phase, in order to produce 4 ATP molecules in the pay off phase.
    Therefore, in order to produce ATP in glycolysis, ATP must already be present in the cell.
    We show that stable operation of autocatalytic cycles dictates constraints on kinetic parameters of participating enzymes.

    Metabolic autocatalytic cycles present control challenges as their inherent feed-back nature makes them susceptible to instabilities such as divergence or drainage of their intermediate metabolites \cite{Fell1999,Reznik2010-te,Teusink1998-he}.
    The stability criteria typically represent one constraint among the parameters of the cycle enzymes.
    For large cycles, such as the whole metabolic network, one such constraint adds little information.
    For compact autocatalytic cycles embedded within metabolism, one such constraint is much more meaningful.
    We thus focus our efforts at analyzing small autocatalytic cycles.
    Understanding the unique constraints that metabolic autocatalytic cycles impose is essential for realizing limitations of existing metabolic networks, as well as for modifying them in synthetic biology and metabolic engineering applications.

    The importance of autocatalytic cycles in metabolism is demonstrated by one of its most prominent examples, the Calvin-Benson-Bassham cycle (CBB) \cite{Benson1950-cl}.
    The carbon fixation CBB cycle, fixing \ce{CO2} while transforming 5-carbon compounds to two 3-carbon compounds, serves as the main gateway for converting inorganic carbon to organic compounds in nature \cite{Raven2012-le}.
    The autocatalytic nature of the CBB cycle stems from the fact that for every five 5-carbon compounds the cycle consumes, six 5-carbon compounds are produced.
    Beyond the CBB cycle, we show that most of the reactions and metabolites in central carbon metabolism are part of compact (i.e. consisting of around 10 reactions or less) metabolic autocatalytic cycles.
    Some of the autocatalytic cycles we find are not usually considered as such.
    The wide span of autocatalytic cycles in central carbon metabolism suggests that the constraints underlying their stable operation have network-wide biological consequences.

    In this study we present the specific requirements metabolic autocatalytic cycles must meet in order to achieve at least one, non-zero, steady state and be stable in respect to fluctuations of metabolites or levels of enzymes close to the steady state point.
    The mathematical tools we use are part of dynamical systems theory \cite{Strogatz2014-hp}.
    We identify the kinetic parameters of enzymes at metabolic branch points out of an autocatalytic cycle as critical values that determine whether the cycle can stably operate.
    We show that the affinity of enzymes consuming intermediate metabolites of autocatalytic cycles must be limited to prevent drainage of these metabolites.
    Moreover, we show that stable operation of such cycles requires low saturation, and thus excess expression, of these enzymes.
    Such low saturation of enzymes has previously been suggested to represent requirements for minimal flux in reversible reactions, for example in glycolysis \cite{Staples1997-fq,Eanes2006-gh,Flamholz2013-bb}, safety factors in face of varying nutrient availability, for example in the brush-border of mouse intestine \cite{Weiss1998-gp}, or rapid shifts in demands from the metabolic network in low flux muscles \cite{Suarez1997-jo}.
    Low saturation has also been previously shown to contribute to fast response to pulses of oxidative load in erythrocytes, resulting from their adherence to phagocytes \cite{Salvador2003-fc}.
    We point that for enzymes consuming intermediate metabolites of autocatalytic cycles there is a reverse relationship between the affinity of the enzyme and the flux it carries.
    Namely, for these enzymes, the lower the affinity to the substrate, the higher the flux carried by the reaction, which is opposite to the normal relationship between affinities and fluxes.
    We use recent fluxomics and proteomics data to test these design principles.
    We find them to hold in all cases tested, where autocatalytic cycles support flux.
    Our analysis demonstrates how the requirement for stable operation of autocatalytic cycles results in design principles that are followed by autocatalytic cycles in-vivo.
    The results and design principles presented here can further be used in synthetic metabolic engineering applications that require proper functioning of autocatalytic cycles.

\section{Results}
\subsection{Autocatalytic cycles play important roles in central carbon metabolism}
Different definitions exist for autocatalytic sets in the context of chemical reaction networks \cite{Hordijk2004-xe, Eigen2012-ti, Kun2008-xg}.
Here we define an autocatalytic cycle as a set of reactions and metabolites that form a cycle, and that, when the reactions are applied to the substrates at appropriate ratios, increase the amount of the intermediate metabolites.
A simple example of a metabolic autocatalytic cycle is shown in Figure \ref{fig:autocataldef}, where an internal metabolite joins with an external assimilated metabolite to give rise to $1+\delta$ copies of the internal metabolite, representing an increase by $\delta$ copies.
For stable operation, $\delta$ copies have to branch out of the cycle, and this consumption must be robust to small fluctuations in enzyme levels and metabolite concentrations.
For a formal, mathematical definition, see the methods section.

\begin{figure}[!htb]
    \centering{
        \begin{tikzpicture}
\begin{scope} [shift={(-6.6cm,-4cm)}]
  \colorlet{genext}{assimcol}
  \colorlet{genmed}{blue}
  \colorlet{geninit}{blue}

  \newlength\imrad;
  \newlength\ierad;
  \newlength\esrad;
  \newlength\emrad;
  \newlength\eerad;
  \pgfmathsetlength{\imrad}{\autocatalrad-\blendfrac*\arcwidth};
  \pgfmathsetlength{\ierad}{\autocatalrad-0.5*\arcwidth};
  \pgfmathsetlength{\esrad}{\autocatalrad+\arcwidth};
  \pgfmathsetlength{\emrad}{\autocatalrad+\arcwidth-\blendfrac*\arcwidth};
  \pgfmathsetlength{\eerad}{\autocatalrad+0.5*\arcwidth};

  \preassim{\autocatalscale*\arcwidth}{-100}{-270}{\autocatalscale*\autocatalrad}{geninit};%width, startang, stopang, rad, col
    \postassim{\autocatalscale*\arcwidth}{90}{-45}{\autocatalscale*\autocatalrad}{geninit}{2}
    \assim{\autocatalscale*\arcwidth}{90}{-30}{\autocatalscale*\autocatalrad}{2}
    \arrowhead{\autocatalscale*\arcwidth}{-45}{\autocatalscale*\autocatalrad}{geninit}

    \node[align=center] at (-30:\autocatalscale*\autocatalrad*2.1) (int) {$\gamma \cdot$ Internal metabolite};
    \node[align=center] at (-73:\autocatalscale*\autocatalrad) (int) {Internal\\metabolite};
    \node[align=center] at (130:\autocatalscale*\autocatalrad*1.65) (ext) {External\\metabolite};
    \node [rectangle,fill=autocatacyc!30,rounded corners=3pt] at (90:\autocatalscale*\autocatalrad+1.7cm) (eq) {Internal metabolite + External metabolite $\rightarrow (1+\gamma)$ Internal metabolite};
  \end{scope}
\end{tikzpicture}

}
\caption{
    \label{fig:autocataldef}
  A basic autocatalytic cycle requires an internal metabolite to be present in order to assimilate the external metabolite into the cycle, increasing the amount of the internal metabolite by some amount, $\delta$.}
\end{figure}
 
While rarely discussed as such, a systematic search in central carbon metabolism of \emph{E.coli} (see methods) shows the ubiquity of autocatalytic cycles.
On top of the previously discussed CBB cycle (Figure \ref{fig:realautocatal}, example I), we show two other prominent examples:
\begin{itemize}
    \item The glyoxylate cycle within the TCA cycle, which turns an internal malate and two external acetyl-CoA's into two malate molecules.
        This is achieved by transforming malate to isocitrate, while assimilating acetyl-CoA, and then splitting the isocitrate to produce two malate molecules, assimilating another acetyl-CoA \cite{Kornberg1966-lh} (Figure \ref{fig:realautocatal}, example II).
    \item A widely known feature in bacteria that import a glucose molecule using phosphoenolpyruvate (PEP) as a co-factor, and then produce two PEP molecules via glycolysis.
        This autocatalytic cycle therefore consumes PEP to transport sugar substrates using the phosphotransferase system (PTS), then converts the imported substrate into two PEP molecules (Figure \ref{fig:realautocatal}, example III).

\end{itemize}
    \iftoggle{elifesubmission} {
        Two additional examples are presented in Figure \ref{fig:realautocatal}-figure supplements 1 and 2 and discussed below.
    } {
        Two additional examples are presented in Figures \ref{fig:extrasamps1} and \ref{fig:extrasamps2} and discussed below.
    }

The ubiquity of compact autocatalytic cycles in central carbon metabolism suggests that unique features of autocatalytic cycles, as derived below, may constrain and shape the kinetic parameters of a broad set of enzymes at the heart of metabolism.

\begin{figure}[!htb]
\resizebox{1\linewidth}{!}{
  \begin{tikzpicture}
 \colorlet{ptsinit}{cyan}
  \colorlet{cbbinit}{yellow}
  \colorlet{glyinit}{magenta}

  \newlength\assimwidth;
  \pgfmathsetlength{\assimwidth}{1.5pt};

  \node[metaboliteStyle] (g6p) {g6p};

  %%%% upper pts
  \node[shape=coordinate,left=12mm of g6p.center] (ptsmid) {};
  \node[metaboliteStyle,left=7mm of ptsmid,rectangle,draw=assimcol,rounded corners=2pt] (gluc) {gluc};
  \node[metaboliteStyle,shift={(-7mm,-7mm)},gray] at (g6p.center) (pyr1) {pyr};
  \draw[assimcol,line width=\assimwidth] (gluc) -- (ptsmid);
  \draw[->] (ptsmid) [out=0,in=90] to (pyr1);

  \node[metaboliteStyle,below=of g6p.center] (f6p) {f6p};
  \node[metaboliteStyle,below=of f6p] (fbp) {fbp};
  \node[metaboliteStyle,shape=coordinate,below=of fbp.center](fbamid) {};
  \node[metaboliteStyle,below left=of fbamid.center] (dhap) {dhap};
  \node[metaboliteStyle,below right=of fbamid] (gap) {gap};
  \node[metaboliteStyle,below=of gap.center] (bpg) {bpg};
  \node[metaboliteStyle,below=of bpg.center] (3pg) {3pg};
  \node[metaboliteStyle,below=of 3pg.center] (2pg) {2pg};
  \node[metaboliteStyle,below=of 2pg.center] (pep) {pep};
  \node[metaboliteStyle,below=of pep.center] (pyr) {pyr};
  \node[metaboliteStyle,below=of pyr.center,rectangle,draw=assimcol,rounded corners=2pt] (aca) {accoa};
  \node[shape=coordinate,below=of aca] (dummyglta) {};
  \node[metaboliteStyle,left=of dummyglta] (oaa) {oaa};
  \node[metaboliteStyle,right=of dummyglta] (cit) {cit};
  \node[metaboliteStyle,right=of cit] (icit) {icit};
  \node[metaboliteStyle,below=of icit.center] (akg) {akg};
  \node[metaboliteStyle,below=of akg.center] (sca) {sca};
  \node[metaboliteStyle,below=of oaa.center] (mal) {mal};
  \node[metaboliteStyle,below=of mal.center] (fum) {fum};
  \node[metaboliteStyle,right=of mal] (glx) {glx};
  \node[metaboliteStyle,right=of fum] (suc) {suc};
  \node[metaboliteStyle,right=of g6p] (6pgi) {6pgi};
  \node[metaboliteStyle,shape=coordinate,right=of f6p] (s7pspace) {};
  \node[metaboliteStyle,right=of s7pspace] (s7p) {s7p};
  \node[metaboliteStyle,right=of s7p] (r5p) {r5p};
  \node[metaboliteStyle,right=of r5p] (ru5p) {ru5p};
  \node[metaboliteStyle,above=of ru5p.center] (6pgc) {6pgc};
  \node[metaboliteStyle,] at (fbp.center -| s7p.center) (e4p) {e4p};
  \node[metaboliteStyle,] at(e4p.center -| r5p.center) (xu5p) {xu5p};
  \node[metaboliteStyle,] at(3pg.center -| ru5p.center) (rub) {rubp};
  \node[metaboliteStyle,rectangle,draw=assimcol,rounded corners=2pt] at(2pg -| xu5p.center) (co2) {\ce{CO2}};
  \draw[->] (g6p) -- (f6p);
  \draw[->] ([xshift=0.1cm]f6p.south) -- ([xshift=0.1cm]fbp.north);
  \draw[<-] ([xshift=-0.1cm]f6p.south) -- ([xshift=-0.1cm]fbp.north);
  \draw [<-] (fbp) [out=-90,in=90] to (fbamid);
  \draw [->] (fbamid) [out=-90,in=45] to (dhap);
  \draw [->] (fbamid) [out=-90,in=135] to (gap);
  \draw [<->] (dhap) -- (gap);
  \draw[<->] (gap) -- (bpg);
  \draw[<->] (bpg) -- (3pg);
  \draw[->] (3pg) -- (2pg);
  \draw[->] (2pg) -- (pep);
  \draw[->] (pep) -- (pyr);
  \draw[->] (pyr) -- (aca);
  \draw[->] (oaa) -- (cit) node [pos=0.9] (midglta) {};
  \draw [assimcol,line width=\assimwidth] (aca) [out=-70,in=180] to (midglta);
  \draw[->] (cit) -- (icit);
  \draw[->] (icit) -- (suc) node [pos=0.3] (midacea) {};
  \draw[->] (midacea) [out=220,in=0] to (glx);
  \draw[->] (icit) -- (akg);
  \draw[->] (akg) -- (sca);
  \draw[->] (sca) -- (suc);
  \draw[->] (suc) -- (fum);
  \draw[->] (fum) -- (mal);
  \draw[->] (glx) -- (mal) node [pos=0.9] (midaceb) {};
  \draw[->] (mal) -- (oaa);
  \draw[assimcol,line width=\assimwidth] (aca) [out=-90,in=0] to (midaceb);
  \draw[->] (g6p) -- (6pgi);
  \draw[->] (6pgi) -- (6pgc);
  \draw[->] (6pgc) -- (ru5p);
  \draw[<-] (ru5p) -- (xu5p);
  \draw[<-] (ru5p) -- (r5p);
  \path[] (r5p) -- (gap) coordinate [pos=0.2] (midtkt1) {};
  \draw[<-] (xu5p) [out=180,in=-90] to (midtkt1);
  \draw[] (midtkt1) [out=90,in=0] to (s7p);
  \draw[<-] (r5p) [out=180,in=90] to (midtkt1);
  \draw[] (midtkt1) [out=270,in=0] to (gap);
  \path[] (e4p) -- (gap) coordinate [pos=0.4] (midtkt2) {};
  \draw[<-] (e4p) [out=-60,in=0] to (midtkt2);
  \draw[] (midtkt2) [out=180,in=90] to (gap);
  \draw[<-] (xu5p) [out=245,in=0] to (midtkt2);
  \draw[] (midtkt2) [out=180,in=-30] to (f6p);
  \path[] (e4p) -- (s7pspace) coordinate [pos=0.5] (midtal) {};
  \draw[] (midtal) [out=90,in=0] to (f6p);
  \draw[<-] (s7p) [out=180,in=90] to (midtal);
  \draw[] (midtal) [out=-90,in=180] to (e4p);
  \draw[<-] (gap) [out=55,in=-90] to (midtal);
  \node[shape=coordinate,left=2.5cm of pep.center] (pts3) {};
  \draw[] (pep) [out=180,in=0] to (pts3);
  \node[shape=coordinate,left=2.5cm of pyr.center] (pts5) {};
  \node[shape=coordinate,left=2.1cm of f6p.center] (ptstop) {};
  \node[shape=coordinate] at(ptstop |- 2pg.center) (ptsbottom) {};
  \draw[] (pts3) [in=-90,out=180] to (ptsbottom);
  \draw[] (ptsbottom) [in=-90,out=90] to (ptstop);
  \draw[] (ptstop) [in=180,out=90] to (ptsmid);
  \draw[->] (ptsmid) -- (g6p);
  \draw[->] (ru5p) [out=-90,in=90] to (rub);
  \draw[->] (rub) -- (3pg) coordinate [pos=0.9] (rubisco); 
  \draw[assimcol,line width=\assimwidth] (co2) [out=90,in=0] to (rubisco);
  \node[shape=coordinate,shift={(-\highlightrad,-\highlightrad)}] at (pep.south -| ptsbottom) (ptsbottomlimit) {};
  \node[shape=coordinate,shift={(-\highlightrad,\highlightrad)}] at (g6p.north -| ptstop) (ptstoplimit) {};

  \draw[opacity=0.2,fill=ptsinit,rounded corners=\highlightrad] ([shift={(\highlightrad,\highlightrad)}]g6p.north east) -- ([xshift=\highlightrad] fbp.east) -- ([shift={(\highlightrad,\highlightrad)}]gap.north east)--([shift={(\highlightrad,-\highlightrad)}]pep.south east) -- (ptsbottomlimit) -- node[midway] (ptsshademid) {} ([yshift=-1.2cm]ptstoplimit) -- ([shift={(-1mm,\highlightrad)}]g6p.north -| ptsmid) -- cycle;

  \draw[very thick,dashed,cyan,->] (ptsshademid) -- ++(-1.5cm,0cm); 

  \draw[opacity=0.2,fill=glyinit,rounded corners=\highlightrad] ([shift={(-\highlightrad,2*\highlightrad)}]oaa.west) -- ([shift={(\highlightrad,2*\highlightrad)}]icit.east) -- node[midway] (glyshadedmid) {}([shift={(\highlightrad,-0.5*\highlightrad)}]icit.south east) -- ([shift={(0.5*\highlightrad,-2*\highlightrad)}]suc.east) -- ([shift={(-\highlightrad,-2*\highlightrad)}]fum.west) -- cycle;

  \draw[very thick,dashed,magenta,->] (glyshadedmid) -- ++(1.5cm,0cm); 

  \node[shape=coordinate] at (dhap.south -| gap.west) (cbbmid) {};
  \draw[opacity=0.2,fill=cbbinit,rounded corners=\highlightrad] ([shift={(-\highlightrad,2.2*\highlightrad)}]f6p.west) -- ([shift={(3*\highlightrad,2.2*\highlightrad)}]ru5p.center) -- node[midway] (cbbshadedmid) {} ([shift={(3*\highlightrad,-2*\highlightrad)}]rub.center) -- ([shift={(-\highlightrad,-2*\highlightrad)}]3pg.west) -- ([shift={(-\highlightrad,-\highlightrad)}]cbbmid) -- ([shift={(-0.5*\highlightrad,-\highlightrad)}]dhap.south west) -- ([shift={(-0.5*\highlightrad,0.5*\highlightrad)}]dhap.north west) -- ([xshift=-\highlightrad]fbp.west) -- cycle;

  \draw[very thick,dashed,cbbinit,->] (cbbshadedmid) -- ++(1.5cm,0cm); 

  %% CBB cycle
  \begin{scope} [shift={(11.2cm,-4.5cm)},radius=2cm]
    \draw[lightgray,rounded corners=\highlightrad] (-2.7,-2.1) rectangle +(4.6,4.5);
    \node at (-2.2cm,2cm) (I) {\large  \textbf{I}};
    \node[anchor=north] at(-0.4cm,-2.2cm) (cbbreac) {{\fontfamily{cmss}\selectfont 5}  3pg + {\fontfamily{cmss}\selectfont 3} \ce{CO2} $\rightarrow$ {\fontfamily{cmss}\selectfont 6} 3pg};
    \colorlet{cbbmed}{blue}
    \colorlet{cbbext}{assimcol}
  

    \newlength\cbbimrad;
    \newlength\cbbierad;
    \newlength\cbbesrad;
    \newlength\cbbemrad;
    \newlength\cbbeerad;
    \newlength\cbbwidth;
    \newlength\cbbtotwidth;
    \pgfmathsetlength{\cbbwidth}{\arcwidth*0.2};
    \pgfmathsetlength{\cbbtotwidth}{\cbbwidth+\arcwidth};
    \pgfmathsetlength{\cbbimrad}{\autocatalrad-\blendfrac*0.5*\cbbtotwidth};
    \pgfmathsetlength{\cbbierad}{\autocatalrad-0.5*\cbbwidth};
    \pgfmathsetlength{\cbbesrad}{\autocatalrad+0.5*\cbbtotwidth};
    \pgfmathsetlength{\cbbemrad}{\autocatalrad+0.5*\cbbtotwidth-\blendfrac*0.5*\cbbtotwidth};
    \pgfmathsetlength{\cbbeerad}{\autocatalrad+0.5*\arcwidth};

    \shadedarc{-20}{-180}{\autocatalrad}{\autocatalrad}{cbbmed}{cbbinit};
    \shadedarc{100}{180}{\cbbimrad}{\autocatalrad}{cbbmed}{cbbinit};
    \coloredarc{25}{100}{\cbbierad}{\cbbimrad}{cbbinit};
    \shadedarc[\cbbwidth]{100}{180}{\cbbemrad}{\cbbesrad}{cbbext}{cbbinit};
    \coloredarc[\cbbwidth]{25}{100}{\cbbeerad}{\cbbemrad}{cbbinit};

%% \assimilatedcol input arc
        \draw[color=cbbext,line width=\cbbwidth]
        (\fromang:\autocatalrad+0.5*\arcwidth+0.5*\cbbwidth)
        arc (0:\inputang:2cm)
        node [pos=0.5,color=black,anchor=east] (co2c) {\ce{CO2}};
%% arrowhead
    \fill[cbbinit]
      (\fromang+\deltaang+1:\autocatalrad-0.5*\arcwidth-0.5*\cbbwidth)
      arc (\fromang+\deltaang+1:\fromang+\deltaang-1:\autocatalrad-0.5*\arcwidth-0.5*\cbbwidth)
      -- (\fromang+\deltaang-1-\protrude:\autocatalrad)
      -- (\fromang+\deltaang-1:\autocatalrad+0.5*\arcwidth+0.5*\cbbwidth)
      arc (\fromang+\deltaang-1:\fromang+\deltaang+1:\autocatalrad+0.5*\arcwidth+0.5*\cbbwidth)
      -- cycle;

%% metabolites
        \node at (0:\autocatalrad) (3pgc) {3pg};
        \node at (220:\autocatalrad-4.7mm) (rubc) {rubp};

  \end{scope}

  %glyoxilate cycle
\begin{scope} [shift={(9cm,-13.5cm)},radius=2cm]
  \draw[lightgray,rounded corners=\highlightrad] (-2.4cm,-2.6cm) rectangle +(4.5,5.1);
  \node at (-2cm,2.2cm) (II) {\large  \textbf {II}};
  \node[anchor=north] at(-0.15cm,-2.7cm) (glyreac) {mal + {\fontfamily{cmss}\selectfont 2} accoa $\rightarrow$ {\fontfamily{cmss}\selectfont 2} mal};
  \colorlet{glymed}{blue}
  \colorlet{glyext}{assimcol}
  \colorlet{glyinter}{blue}
  
    \newlength\glyimrad;
    \newlength\glyierad;
    \newlength\glyesrad;
    \newlength\glyemrad;
    \newlength\glyeerad;
    \newlength\glywidth;
    \newlength\glytotwidth;
    \newlength\glyfinwidth;
    \newlength\glyimmrad;
    \newlength\glyemmrad;
    \newlength\glyemmmrad;
    \newlength\glyeamrad;
    \pgfmathsetlength{\glywidth}{\arcwidth*0.5};
    \pgfmathsetlength{\glytotwidth}{\glywidth+\arcwidth};
    \pgfmathsetlength{\glyfinwidth}{\glytotwidth+\glywidth};
    \pgfmathsetlength{\glyimrad}{\autocatalrad-\blendfrac*0.5*\glywidth};
    \pgfmathsetlength{\glyimmrad}{\autocatalrad-0.5*\glywidth};
    \pgfmathsetlength{\glyemmrad}{\glyimmrad+0.5*\glytotwidth};
    \pgfmathsetlength{\glyemmmrad}{\glyemmrad+0.5*\glywidth};
    \pgfmathsetlength{\glyeamrad}{\autocatalrad+0.5*\glytotwidth};
    \pgfmathsetlength{\glyierad}{\autocatalrad-0.5*\glywidth};
    \pgfmathsetlength{\glyesrad}{\autocatalrad+0.5*\glytotwidth};
    \pgfmathsetlength{\glyemrad}{\autocatalrad+0.5*\glyfinwidth-\blendfrac*0.5*\glywidth};
    \pgfmathsetlength{\glyeerad}{\autocatalrad+0.5*\glytotwidth};

    \shadedarc{-20}{-90}{\autocatalrad}{\autocatalrad}{glymed}{glyinit};
    \shadedarc{-150}{-90}{\glyimmrad}{\autocatalrad}{glymed}{glyinter};
    \shadedarc[\glywidth]{-150}{-90}{\glyemmrad}{\glyeamrad}{glyext}{glyinter};

    \draw[color=glyext,line width=\glywidth] (-90:\autocatalrad+0.5*\glytotwidth) arc(90:50:2cm) node [pos=0.5,color=black,anchor=north,xshift=-1.5mm] (acac) {accoa};

    \coloredarc[\glytotwidth]{160}{90}{\glyimmrad}{\glyimmrad}{glyinter};

    \draw[color=glyext,line width=\glywidth] (90:\glyemmmrad) arc(-90:-130:2cm) node [pos=0.5,color=black,anchor=south,xshift=2mm] (acac) {accoa};

    \shadedarc[\glytotwidth]{50}{90}{\glyimrad}{\glyimmrad}{glyinter}{glyinit};
    \coloredarc[\glytotwidth]{50}{25}{\glyimrad}{\glyierad}{glyinit};

    \shadedarc[\glywidth]{90}{50}{\glyemmmrad}{\glyemrad}{glyinit}{glyext};
    \coloredarc[\glywidth]{50}{25}{\glyemrad}{\glyeerad}{glyinit};


    \node at (0:\autocatalrad) (malc) {mal};
    \node at (180:\autocatalrad) (glxc) {glx+fum};
    \node at (-50:\autocatalrad-4.5mm) (oaa) {oaa};

  \fill[glyinit] (\fromang+\deltaang+1:\autocatalrad-\arcwidth) arc (\fromang+\deltaang+1:\fromang+\deltaang-1:\autocatalrad-\arcwidth)
       -- (\fromang+\deltaang-1-\protrude:\autocatalrad) -- (\fromang+\deltaang-1:\autocatalrad+\arcwidth) arc (\fromang+\deltaang-1:\fromang+\deltaang+1:\autocatalrad+\arcwidth)
       -- cycle;

  \fill[glyinter] (-149:\autocatalrad-\arcwidth+0.5*\glywidth) arc (-149:-161:\autocatalrad-\arcwidth+0.5*\glywidth)
       -- (-161-\protrude:\autocatalrad) -- (-161:\autocatalrad+\arcwidth-0.5*\glywidth) arc (-161:-149:\autocatalrad+\arcwidth-0.5*\glywidth)
       -- cycle;
  \end{scope}


  %pts cycle
\begin{scope} [shift={(-5.8cm,-5.5cm)},radius=2cm]
  \draw[lightgray,rounded corners=\highlightrad] (-2.5,-3) rectangle +(4.5,5.5);
  \node at (-2cm,2.2cm) (III) {\large  \textbf {III}};
  \node[anchor=north] at(-0.25cm,-3.1cm) (ptsreac) {gap + gluc $\rightarrow$ {\fontfamily{cmss}\selectfont 2} gap + pyr};
    \colorlet{ptsmed}{blue}
    \colorlet{ptsext}{assimcol}

    \newlength\ptsierad;
    \newlength\ptsimrad;
    \newlength\ptsarcwidth;
    \newlength\ptsesrad;
    \newlength\ptsemrad;
    \pgfmathsetlength{\ptsierad}{\autocatalrad*0.5};
    \pgfmathsetlength{\ptsimrad}{\autocatalrad-0.5*\arcwidth};
    \pgfmathsetlength{\ptsarcwidth}{2*\arcwidth};
    \pgfmathsetlength{\ptsesrad}{\autocatalrad+0.5*\arcwidth+0.5*\ptsarcwidth};
    \pgfmathsetlength{\ptsemrad}{\autocatalrad+0.5*\ptsarcwidth};

    \shadedarc{-20}{-120}{\autocatalrad}{\autocatalrad}{ptsmed}{ptsinit};
    \shadedarc{160}{240}{\ptsimrad}{\autocatalrad}{ptsmed}{ptsinit};
    \shadedarc{70}{160}{\ptsierad}{\ptsimrad}{ptsinit}{white};
    \shadedarc[\ptsarcwidth]{160}{240}{\ptsemrad}{\ptsesrad}{ptsext}{ptsinit};
    \coloredarc[\ptsarcwidth]{25}{160}{\autocatalrad}{\ptsemrad}{ptsinit};

  \fill[ptsinit] (\fromang+\deltaang+1:\autocatalrad-\arcwidth) arc (\fromang+\deltaang+1:\fromang+\deltaang-1:\autocatalrad-\arcwidth)
       -- (\fromang+\deltaang-1-\protrude:\autocatalrad) -- (\fromang+\deltaang-1:\autocatalrad+\arcwidth) arc (\fromang+\deltaang-1:\fromang+\deltaang+1:\autocatalrad+\arcwidth)
       -- cycle;

       %% \assimilatedcol input arc
       \draw[color=ptsext,line width=\ptsarcwidth] (240:\ptsesrad) arc(60:20:2cm) node [pos=0.5,color=black,left,shift={(-1mm,-2mm)}] (glucc) {gluc};
    \node at (0:\autocatalrad) (gapc) {gap};
    \node at (270:\autocatalrad-4.5mm) (pepc) {pep};
    \node at (70:\ptsierad) (pyrc) {pyr};
  \end{scope}


\end{tikzpicture}


}
\caption{
    \label{fig:realautocatal}
Three representative autocatalytic cycles in central carbon metabolism: (I) The Calvin-Benson-Bassham cycle (yellow); (II) The glyoxylate cycle (magenta); (III) A cycle using the PTS system to assimilate glucose (cyan).
Assimilation reactions are indicated in green.
Arrow width in panels represent relative carbon mass.
}
\end{figure}

\subsection{Steady state existence and stability analysis of a simple autocatalytic cycle}
To explore general principles governing the dynamic behavior of autocatalytic cycles, we consider the simple autocatalytic cycle depicted in Figure \ref{fig:simplecycle}A.
This cycle has a single intermediate metabolite, $X$.
We denote the flux through the autocatalytic reaction of the cycle by $f_a$, such that for any unit of $X$ consumed, it produces two units of $X$.
The autocatalytic reaction assimilates an external metabolite (denoted $A$), which we assume to be at a constant concentration.
We denote the flux through the reaction branching out of the cycle by $f_b$.
Biologically, $f_b$ represents the consumption of the intermediate metabolite $X$.
While not obligatorily such, in the cycles we find the branch reactions provide precursors that support growth through subsequent reactions.
We therefore also consider $f_b$ to represent biomass generation.

For simplicity in the derivation, we assume irreversible Michaelis-Menten kinetics for the two reactions.
We note that under the assumption of constant concentration of $A$, the bisubstrate velocity equation for $f_a$ can be reduced to a simple Michaelis-Menten equation the apparent kinetic constants of which depend on the constant value of $A$ assumed (see methods).
We therefore assume that:
    \begin{eqnarray*}
      f_a = \frac{V_{\max,a}X}{K_{M,a}+X} \\
      f_b = \frac{V_{\max,b}X}{K_{M,b}+X}
    \end{eqnarray*}
    where $V_{\max}$ is the maximal flux each reaction can carry and $K_M$ is the substrate concentration at which half the maximal flux is attained.
    Physiologically, these kinetic parameters must be positive.
    Using these simple forms allows us to obtain an analytic solution.
    We discuss more general cases below.
    \begin{figure}[!htb]
        \begin{tikzpicture}[>=latex',node distance = 2cm]
    \tikzset{
        vstyle/.style={opacity=0.3,pattern=north west lines,cyan,visible on=<7->}}
    \tikzset{
        kstyle/.style={opacity=0.3,pattern=north east lines,magenta,visible on=<7->}}
  \begin{scope}[shift={(-4cm,4.3cm)}]
        \node at (-60:1cm) (X) {$X$};
        \node[shape=coordinate] (orig) {};
        \draw [-,line width=1pt,autocatacyc] (X.south west) arc (285:0:1cm) node [pos=0.65,above] (fa) {$f_a:$\small{$A+X\rightarrow2X$}} node [pos=0.45,shape=coordinate] (midauto) {} node [pos=1,shape=coordinate] (endcommon) {};
        \draw [->,line width=1pt,autocatacyc] (endcommon) arc (-25:-44:2cm);
        \draw [->,line width=1pt,autocatacyc] (endcommon) arc (-5:-32:1.5cm);
        \draw [line width=1pt,assimcol] (midauto) arc (-60:-90:1cm) node [pos=1,left] (e) {$A$};
        \draw [->,line width=1pt,branchout] (X.south east) arc (225:270:1cm) node [pos=0.75,above] {$f_b$};
        \iftoggle{article} {
            \node at (-2.4cm,1.3cm) (A) {(A)};
        }{}
  \end{scope}
  \begin{scope}[shift={(-1.5cm,-\gridsize/2)}]
    \begin{axis}[name=phase,clip=false,xmin=0,ymin=0,xmax=2,ymax=2,ylabel={\Large{$\sfrac{V_{\max,b}}{V_{\max,a}}$}},xlabel={\Large{$\sfrac{K_{M,b}}{K_{M,a}}$}},samples=6,width=\gridsize,height=\gridsize,ytick={0,1,2},xtick={0,1,2},visible on=<7->]
        \addplot[domain=0:2,dotted,black,thick] {x};
        \addplot[dotted,black,thick] coordinates {(0,1) (2,1)};
        \draw[kstyle] (axis cs:0,0) -- (axis cs:2,2) -- (axis cs:2,0) --cycle;
        \draw[vstyle] (axis cs:0,1) -- (axis cs:2,1) -- (axis cs:2,2) -- (axis cs:0,2) --cycle;
        \draw[->,black!50,dashed] (axis cs:0.25,1.4) -- +(-1.9cm,0cm);
        \draw[->,black!50,dashed] (axis cs:1.75,1.4) -- +(1.1cm,0cm);
        \draw[->,black!50,dashed] (axis cs:0.25,0.6) -- +(-1.9cm,0cm);
        \draw[->,black!50,dashed] (axis cs:1.75,0.6) -- +(1.1cm,0cm);
        \node[align=left,anchor=east] at (axis cs:1.75,1.4) (I) {I};
        \node[align=right,anchor=west] at (axis cs:0.25,1.4) (II) {II};
        \node[align=right,anchor=west] at (axis cs:0.25,0.6) (III) {III};
        \node[align=left,anchor=east] at (axis cs:1.75,0.6) (IV) {IV};
      \end{axis}

\iftoggle{article} {
        \pgfmathsetlength{\plotwidthanim}{\plotwidth}
        \pgfmathsetlength{\plotheightanim}{\plotheight}
        \pgfmathsetlength{\plotshift}{1mm}
}{
    \only<5-> {
        \pgfmathsetlength{\plotwidthanim}{\plotwidth}
        \pgfmathsetlength{\plotheightanim}{\plotheight}
        \pgfmathsetlength{\plotshift}{1mm}
    }
}

      \begin{axis}[name=plot1,axis x line=middle,axis y line=left,xlabel near ticks,ylabel near ticks,xmin=0,ymin=-2.5,xmax=2.9,ymax=5.9,xlabel={[$X$]},ylabel={flux},samples=60,width=\plotwidthanim,height=\plotheightanim,clip=false,yticklabels={,,},xticklabels={,,},tick label style={major tick length=0pt},at=(phase.right of north east),anchor=left of north west,ylabel style={name=ylabel1},xshift=\plotshift,visible on=<2->]%,axis background/.style={fill=cyan!50!magenta,opacity=0.3}]
        \addplot[domain=0:2.9,autocatacyc,thick] {3*x/(0.1+x)};
        \addplot[domain=0:2.9,branchout,thick,visible on=<3->] {5*x/(1+x)};
        \addplot[domain=0:2.9,sumcolor,thick,visible on=<4->] {3*x/(0.1+x)-5*x/(1+x)};
        \addplot[dashed,gray,thick,visible on=<4->] coordinates {(1.25,0) (1.25,2.77)};
        \node[right,align=left,visible on=<6->] (onetext) at (axis cs:0.05,4.7) {\scriptsize \textbf{stable non-zero}\\[-0.4em]\scriptsize \textbf{steady state}};
      \end{axis}
     \iftoggle{elifesubmission} {}
     {
      \iftoggle{article} {}
      {
        \node[visible on=<2-4>,color=blue,at=(plot1.left of north west),anchor=north east,scale=1.5,xshift=-1cm,yshift=-0.5cm] (fa){$f_a=\frac{V_{\max,a}X}{K_{M,a}+X}$};
        \node[visible on=<3-4>,color=red,below=of fa,scale=1.5,yshift=0.7cm] (fb) {$f_b=\frac{V_{\max,b}X}{K_{M,b}+X}$};
      }
    }
      \node[draw,fit=(plot1) (ylabel1),line width=2pt, fill=none,rounded corners=3pt,cyan!50!magenta,opacity=0.6,visible on=<7->]{};

        \begin{customlegend}[legend entries={$f_a$,$V_{\max,b}>V_{\max,a}$,$f_b$,$\sfrac{V_{\max,b}}{V_{\max,a}}<\sfrac{K_{M,b}}{K_{M,a}}$,$\dot{X}=f_a-f_b$},legend style={above=1cm of plot1.north east,anchor=south east,name=legend1,visible on=<4->},legend columns=2]
          \addlegendimage{autocatacyc,fill=black!50!red,sharp plot,line width=1pt}
          \addlegendimage{vstyle,area legend,visible on=<5->}
          \addlegendimage{branchout,fill=black!50!red,sharp plot,line width=1pt}
          \addlegendimage{kstyle,area legend,visible on=<5->}
          \addlegendimage{sumcolor,fill=black!50!red,sharp plot,line width=1pt}
        \end{customlegend}

      \begin{axis}[name=plot2,axis x line=middle,axis y line=left,xlabel near ticks,ylabel near ticks,xmin=0,ymin=-2.5,xmax=2.9,ymax=5.9,xlabel={[$X$]},ylabel={flux},samples=60,at=(phase.left of north west),anchor=right of north east,width=\plotwidth,height=\plotheight,yticklabels={,,},xticklabels={,,},tick label style={major tick length=0pt},ylabel style={name=ylabel2},xshift=-1mm,visible on=<5->]%,axis background/.style=vstyle]
        \addplot[domain=0:4,autocatacyc,thick] {4*x/(1+x)};
        \addplot[domain=0:4,branchout,thick] {5*x/(0.2+x)};
        \addplot[domain=0:4,sumcolor,thick,visible on=<6->] {4*x/(1+x)-5*x/(0.2+x)};
        \node[right,align=left,visible on=<6->] (twotext) at (axis cs:0.0,5) {\scriptsize stable zero steady state};
      \end{axis}
     \iftoggle{article} {}
     {
      \node[draw,fit=(plot2) (ylabel2),line width=2pt, vstyle,fill=none,rounded corners=3pt]{};
  }

      \begin{axis}[name=plot3,axis x line=middle,axis y line=left,xlabel near ticks,ylabel near ticks,xmin=0,ymin=-2.5,xmax=2.9,ymax=5.9,xlabel={[$X$]},ylabel={flux},samples=60,width=\plotwidth,height=\plotheight,yticklabels={,,},xticklabels={,,},tick label style={major tick length=0pt},at=(phase.left of south west),anchor=right of south east,ylabel style={name=ylabel3},xshift=-1mm,visible on=<5->]
        \addplot[domain=0:4,autocatacyc,thick] {5*x/(1+x)};
        \addplot[domain=0:4,branchout,thick] {3*x/(0.1+x)};
        \addplot[domain=0:4,sumcolor,thick,visible on=<6->] {5*x/(1+x)-3*x/(0.1+x)};
        \addplot[dashed,gray,thick] coordinates {(1.25,0) (1.25,2.77)};
        \node[right,align=left,visible on=<6->] (threetext) at (axis cs:0.05,4.7) {\scriptsize unstable non-zero\\[-0.4em]\scriptsize steady state};
      \end{axis}
     \iftoggle{article} {}
     {
      \node[draw,fit=(plot3) (ylabel3),line width=2pt, fill=none,rounded corners=3pt,opacity=0.2, black!40,visible on=<7->]{};
  }

      \begin{axis}[name=plot4,axis x line=middle,axis y line=left,xlabel near ticks,ylabel near ticks,xmin=0,ymin=-2.5,xmax=2.9,ymax=5.9,xlabel={[$X$]},ylabel={flux},samples=60,at=(phase.right of south east),anchor=left of south west,width=\plotwidth,height=\plotheight,yticklabels={,,},xticklabels={,,},tick label style={major tick length=0pt},ylabel style={name=ylabel4},xshift=1mm,visible on=<5->]%,axis background/.style=kstyle]
        \addplot[domain=0:2.9,autocatacyc,thick] {5*x/(0.2+x)};
        \addplot[domain=0:2.9,branchout,thick] {4*x/(1+x)};
        \addplot[domain=0:2.9,sumcolor,thick,visible on=<6->] {5*x/(0.2+x)-4*x/(1+x)};
        \node[right,align=left,visible on=<6->] (fourtext) at (axis cs:0.05,5) {\scriptsize  no stable steady state};
     \end{axis}
     \iftoggle{article} {}
     {
      \node[draw,fit=(plot4) (ylabel4),line width=2pt, kstyle,fill=none,rounded corners=3pt]{};
     }

        \iftoggle{article} {
          \node [at=(plot2.north west),xshift=-0.6cm,yshift=0.35cm] (B) {(B)};
        }{}
    \end{scope}
  \end{tikzpicture}


      \caption{\label{fig:simplecycle}
        (A) A simple autocatalytic cycle induces two fluxes, $f_a$ and $f_b$ as a function of the concentration of $X$.
        These fluxes follow simple Michaelis Menten kinetics.
        A steady state occurs when $f_a=f_b$, implying that $\dot{X}=0$.
        The cycle always has a steady state at $X=0$.
        The slope of each reaction at $X=0$ is $\sfrac{V_{\max}}{K_m}$.
        A steady state is stable if at the steady state concentration $\frac{d\dot{X}}{dX}<0$.
        (B) Each set of kinetic parameters, $V_{\max,a},V_{\max,b},K_{M,a},K_{M,b}$ determines two characteristics of the system: 
        If $V_{\max,b}>V_{\max,a}$, then a stable steady state concentration must exist as for high concentrations of $X$ the branching reaction will reduce its concentration (cyan domain, cases (I) and (II)).
        If $\sfrac{V_{\max,b}}{K_{M,b}}<\sfrac{V_{\max,a}}{K_{M,a}}$, implying that $\sfrac{V_{\max,b}}{V_{\max,a}}<\sfrac{K_{M,b}}{K_{M,a}}$, then zero is a non-stable steady state concentration as if $X$ is slightly higher than zero, the autocatalytic reaction will carry higher flux, further increasing the concentration of $X$ (magenta domain, cases (I) and (IV)).
    At the intersection of these two domains a non-zero, stable steady state concentration exists (I).}
    \end{figure}

    We characterize the metabolic state of this system by the concentration of the metabolite $X$.
    We note that knowing the concentration of $X$ suffices in order to calculate the fluxes originating from it, $f_a$ and $f_b$, thus fully defining the state of the system.
    A steady state of the system is defined as a concentration, $X^*$, which induces fluxes that keep the concentration constant, such that the total in-flux to $X$ is exactly equal to the total out-flux from it.
    In our example, the outgoing flux from $X$ is $f_a+f_b$ and the incoming flux to $X$ is $2f_a$, so at steady state it holds that:

    \begin{equation}
      \label{eq:xdyna}
      \dot X = \frac{dX}{dt} = 2f_a - (f_a + f_b) = 0
    \end{equation}

    Intuitively, at steady state, the branch reaction must consume all the excess intermediate metabolite that is produced by the autocatalytic reaction.
    Indeed, expanding the condition above gives:

    \begin{equation*}
      f_a = f_b \Rightarrow \frac{V_{\max,a}X^*}{K_{M,a}+X^*}=\frac{V_{\max,b}X^*}{K_{M,b}+X^*}
    \end{equation*}
    which is satisfied either if $X^*=0$ or if:

    \begin{equation}
      \label{eq:xstst}
      X^*=\frac{V_{\max,b}K_{M,a}-V_{\max,a}K_{M,b}}{V_{\max,a}-V_{\max,b}}
    \end{equation}
    implying that:
    \begin{equation}
      \label{eq:xststalt}
      \frac{X^*}{K_{M,a}}=\frac{\frac{V_{\max,b}}{V_{\max,a}}-\frac{K_{M,b}}{K_{M,a}}}{1-\frac{V_{\max,b}}{V_{\max,a}}}
    \end{equation}
    The concentration of $X$ cannot be negative, and thus we get a constraint on the kinetic parameters for which a positive steady state exists, either both the numerator and the denominator of equation \ref{eq:xststalt} are negative, such that:
    \begin{equation*}
    1<\frac{V_{\max,b}}{V_{\max,a}}<\frac{K_{M,b}}{K_{M,a}}
    \end{equation*}
    , or both are positive, such that:
    \begin{equation*}
    1>\frac{V_{\max,b}}{V_{\max,a}}>\frac{K_{M,b}}{K_{M,a}}
    \end{equation*}
    This constraint is graphically illustrated in Figure \ref{fig:simplecycle}B, cases (I) and (III).

    In order to gain intuition for this relationship we note that $\frac{V_{\max}}{K_m}$ is the slope of the Michaelis Menten function at $X=0$.
    The constraint therefore implies that in order for a positive steady state to exist, the reaction with higher maximal flux must have a shallower slope at $X=0$.
    Mathematically, the constraint states that if $V_{\max,b}<V_{\max,a}$ then $\frac{V_{\max,b}}{K_{M,b}}>\frac{V_{\max,a}}{K_{M,a}}$.
    Alternatively, if $V_{\max,b}>V_{\max,a}$ then $\frac{V_{\max,b}}{K_{M,b}}<\frac{V_{\max,a}}{K_{M,a}}$.
    This condition can be intuitively understood, as the reaction with shallower slope at $X=0$ has smaller fluxes for small values of $X$, compared with the other reaction, so unless it has higher fluxes than the other reaction for large values of $X$ (meaning that its maximal flux is higher), the two will never intersect, which is the condition for a steady state.

    While having a positive concentration steady state is an essential condition to sustain flux, it is not sufficient in terms of biological function.
    The positive concentration steady state must also be stable to small perturbations.
    Stability with respect to small perturbations is determined by the response of the system to small deviations from the steady state, $X^*$.
    Mathematically, stability dictates that at $X=X^*$ it holds that $\frac{d\dot X}{dX} <0$, as this  implies that upon a small deviation from the steady state: $X = X^*+\Delta X$ the net flux $\dot X$ will oppose the direction of the deviation.
    If $\Delta X$ is positive then $\dot X$ will be negative at $X^*+\Delta X$, reducing $X$ back to $X^*$, and if $\Delta X$ is negative, $\dot X$ will be positive, increasing $X$ back to $X^*$.

    For the simple kinetics we chose, the stability condition dictates that:
    \begin{equation}
        \label{eq:stable}
      \frac{d\dot X}{dX}\Big\vert_{X=X^*} = \frac{V_{\max,a}K_{M,a}}{(K_{M,a}+X^*)^2}-\frac{V_{\max,b}K_{M,b}}{(K_{M,b}+X^*)^2}<0
    \end{equation}
    The analysis is straightforward for the case of $X^*=0$, yielding that $0$ is a stable steady state concentration if $\frac{V_{\max,b}}{K_{M,b}}>\frac{V_{\max,a}}{K_{M,a}}$, corresponding to the area above the diagonal in figure \ref{fig:simplecycle}B, where $\frac{V_{\max,b}}{V_{\max,a}}>\frac{K_{M,b}}{K_{M,a}}$. These cases are denoted as cases (II) and (III).
    If the relation is reversed (i.e. $\frac{V_{\max,b}}{K_{M,b}}<\frac{V_{\max,a}}{K_{M,a}}$), then $0$ is an unstable steady state.
    The criterion that is of interest, however, is the criterion for stability of the non-zero steady state, $X^*=\frac{V_{\max,b}K_{M,a}-V_{\max,a}K_{M,b}}{V_{\max,a}-V_{\max,b}}$.
    In this case, substituting $X^*$ in equation \ref{eq:stable} gives the opposite condition to that of $X^*=0$.
    This steady state is thus stable if $\frac{V_{\max,b}}{K_{M,b}}<\frac{V_{\max,a}}{K_{M,a}}$, corresponding to the magenta domain in figure \ref{fig:simplecycle}B, containing cases (I) and (IV), and unstable otherwise.

    The stability criterion can be generally stated in metabolic control terms \cite{Fell1997-bp} using the notion of elasticity coefficients of reactions, defined as:

    \begin{equation*}
    \varepsilon^f_X=\frac{\partial f}{\partial X}\frac{X}{f}
    \end{equation*}
    In these terms, stability is obtained if and only if the elasticity of the branch reaction at the positive steady state concentration is greater than the elasticity of the autocatalytic reaction:

    \begin{equation*}
        \frac{df_b}{dX}\Big\vert_{X=X^*}>\frac{df_a}{dX}\Big\vert_{X=X^*} \Rightarrow \varepsilon^{f_b}_X>\varepsilon^{f_a}_X
    \end{equation*}
    
    The complete analysis is summarized up in Figure \ref{fig:simplecycle}B.
    Domain (I) is the only domain where a positive, stable steady state exists.
    Domains (I) and (III) are the domains at which a positive steady state concentration exists, but in domain (III) that steady state is not stable.
    The domains below the diagonal (cases (I) and (IV)) are the domains where $X^*=0$ is an unstable steady state concentration, so that if another steady state exists, it is stable, but in domain (IV) no positive steady state exists.
    The domains above the diagonal (cases (II) and (III)) are the domains where $X^*=0$ is a stable steady state concentration, so that the other steady state, if it exists, is unstable.

    Aside from existence and stability, a quantitative relationship between the affinity of the biomass generating, branching reaction and the flux it carries can be obtained.
    This relationship is opposite to the standard one, meaning that unlike the common case where $f\propto\frac{1}{K_M}$, in this case, noting that the steady state concentration increases with $K_{M,b}$ (equation \ref{eq:ststdep}), we get that $f_b \propto K_{M,b}$.

    To conclude, for this simple cycle, we get that in order for a positive-concentration stable steady state to exist (case (I)), two conditions must be satisfied:
    \begin{equation}
    \label{eq:stabconds}
    \begin{dcases}
      & V_{\max,b}>V_{\max,a} \\
      & \frac{V_{\max,b}}{K_{M,b}}<\frac{V_{\max,a}}{K_{M,a}}
    \end{dcases}
    \end{equation}
    The first requirement states that the maximal flux of the biomass generating, branching reaction should be higher than the maximal flux of the autocatalytic reaction.
    This requirement ensures a stable solution exists, as large concentrations of $X$ will result in its reduction by the branching reaction.
    The second requirement states that at low concentrations of $X$, the autocatalytic reaction starts higher than the branch reaction (as can be inferred from the slopes).
    This requirement implies that the two fluxes will be equal for some positive concentration of $X$, ensuring a positive steady state exists.
    As this requirement further implies that below the positive steady state the branch reaction will carry less flux than the autocatalytic reaction, it follows that small deviations of the concentration of $X$ below the steady state will result in an increase in its concentration by the autocatalytic reaction, driving it back to the steady state.
    Meeting the second constraint has another consequence.
    Unlike the common case where higher affinity of an enzyme increases the flux through the reaction it catalyzes, in autocatalytic cycles, higher affinity of the branch reaction reduces the flux it carries due to reduction in the steady state concentration of the intermediate metabolite of the cycle.

    Interestingly, these conditions imply that if $K_{M,b}<K_{M,a}$ then no positive stable steady state can be achieved.
    Specifically, changes to the expression levels of the enzymes catalyzing $f_a$ and $f_b$ only affect $V_{\max,a}$ and $V_{\max,b}$, and therefore do not suffice to attain a stable positive steady state.
    This indicates that stability of autocatalytic cycles depends on inherent kinetic properties of the enzymes involved and cannot always be achieved by modulating expression levels, a design principle that may be critical in metabolic engineering context.

    \subsection{Integrating the bisubstrate nature of the autocatalytic reaction into the simple model}
    In the section above, to keep the analysis concise, we neglected the bisubstrate nature of the autocatalytic reaction.
    There are generally two kinds of bisubstrate reaction schemes; random binding order scheme, at which the two substrates can bind the relevant enzyme at any order, and ordered binding, at which one substrate must bind the enzyme first and only then can the second substrate bind and the reaction proceeds \cite{Leskovac2003-li}.
    Because in the ordered binding case, the equation that describes the reaction flux in asymmetric is the concentrations of the two substrates, two possible sub-cases exist; the assimilated metabolite can be the first-binding substrate, or the second-binding substrate.

    We note that, as is shown in the methods, all three bisubstrate reaction schemes take a Michaelis-Menten like form once the concentration of the assimilated metabolite is kept constant.
    Moreover, we find that for any set of kinetic parameters, under all three schemes, a lower bound on the concentration of $A$ exists, below which the conditions for existence and stability of a positive steady state cannot be satisfied.
    The exact value of the minimal concentration of $A$ depends on the specific bisubstrate reaction scheme and the kinetic parameters of it, as is shown in the methods.
    In the random binding case and in the ordered binding case when the assimilated metabolite binds first, we further find that a lower bound on $K_{M,b}$ exists, as in the simplified model analyzed above.
    On the other hand, if the bisubstrate reaction is ordered with the internal metabolite binding first, then no strict constraints exist on $K_{M,b}$ and a stable steady state solution can always be achieved by setting appropriate values to $V_{\max,b}$ and $V_{\max}'$, the maximal flux of the bisubstrate autocatalytic reaction.
    
    Finally, regarding the dynamic behavior of the system when the concentration of $A$ varies, we note that in all three cases, as the concentration of $A$ approaches its lower bound, the steady state concentration of $X$ approaches $0$, reducing both the autocatalytic and the branch fluxes (see methods).
    In the specific case that the maximal flux of the autocatalytic reaction is higher than the maximal flux of the branch reaction, an upper bound on the concentration of $A$ may also exist, to satisfy the condition that $V_{\max,a}<V_{\max,b}$.
    However, this bound can be removed by increasing $V_{\max,b}$ or reducing $V_{\max}'$, the maximal flux of the bisubstrate autocatalytic reaction.
\subsection{Extensions of the simple autocatalytic cycle model}
    \subsubsection{Generalizing for different autocatalytic stoichiometries}
    Our didactic exemplary analysis considered an autocatalytic reaction with $1:2$ stoichiometry, such that for every substrate molecule consumed, two are produced.
    Real-world autocatalytic cycles may have different stoichiometries.
    For example, the CBB cycle has a stoichiometry of $5:6$ so that for every five molecules of 5-carbon sugars that the autocatalytic reaction consumes, six 5-carbon sugars are produced.
    We can generalize our analysis by defining a positive $\delta$ such that for every molecule of $X$ that $f_a$ consumes, it produces $1+\delta$ molecules of $X$, where $\delta$ may be a fraction.
    This extension implies that equation \eqref{eq:xdyna} becomes:

    \begin{equation*}
      \dot X = \frac{dX}{dt} = (1+\delta)f_a - (f_a + f_b) = 0 \Rightarrow \delta f_a = f_b \Rightarrow \frac{\delta V_{\max,a}X}{K_{M,a}+X}=\frac{V_{\max,b}X}{K_{M,b}+X}
    \end{equation*}

    Therefore, all of the results above can be extended to different stoichiometries by replacing $V_{\max,a}$ with $\delta V_{\max,a}$.
    As a result, the qualitative conditions and observations from the $1:2$ stoichiometry case remain valid but with a constant factor that changes the quantitative relations according to the relevant stoichiometry.
\subsubsection{Input flux increases the range of parameters for which a stable steady state solution exists}
    Autocatalytic cycles are embedded within a larger metabolic network.
    Specifically, such cycles may have independent input fluxes to some of their intermediate metabolites, not requiring the use of other intermediate metabolites of the cycle.
    For example, in the glucose based, PTS-dependent autocatalytic cycle, the existence of alternative transporters can generate flux of glucose 6-phosphate into the cycle without the use of PEP \cite{Ferenci1996-nv}.

    When adding a constant input flux, $f_i$ to our simple system (Figure \ref{fig:inputcycle}A) the steady state condition changes to include this flux, giving:

    \begin{equation*}
      \dot X = \frac{dX}{dt} = f_i + f_a - f_b = 0
    \end{equation*}
    In this situation, at $X=0$, $\dot X=f_i>0$ so the concentration of $X$ increases and there is no steady state at zero.
    If $V_{\max,b}>f_i+V_{\max,a}$ then at a large enough value of $X$, $\dot X$ will be negative, implying that at some value of $X$ between these two extremes, $\dot{X}$ attains the value of zero, such that under this condition a positive stable steady state concentration exists (Figure \ref{fig:inputcycle} (I)).
    This case therefore differs from the case with no input flux analyzed above, as now a positive stable steady state can always be achieved by modifying only $V_{\max,a}$ and/or $V_{\max,b}$.
    In this setup, cells can therefore tune the expression levels of enzymes to meet the needs of a stable steady state flux.


    \begin{figure}[!htb]
        \centering{
            \begin{tikzpicture}[>=latex',node distance = 2cm]
  \begin{scope}[]
        \node at (-60:1cm) (X) {$X$};
        \node[shape=coordinate] (orig) {};
        \draw [-,line width=1pt,autocatacyc] (X.south west) arc (285:0:1cm) node [pos=0.65,above] (fa) {$f_a:$\small{$E+X\rightarrow2X$}} node [pos=0.45,shape=coordinate] (midauto) {} node [pos=1,shape=coordinate] (endcommon) {};
        \draw [->,line width=1pt,autocatacyc] (endcommon) arc (-25:-44:2cm);
        \draw [->,line width=1pt,autocatacyc] (endcommon) arc (-5:-32:1.5cm);
        \draw [line width=1pt,assimcol] (midauto) arc (-60:-90:1cm) node [pos=1,left] (e) {$E$};
        \draw [->,line width=1pt,branchout] (X.south east) arc (225:270:1cm) node [pos=0.75,above] {$f_b$};
    \draw [<-,line width=1pt,magenta] (X.east) arc (-90:-30:1cm) node [pos=0.4,above] (fi) {$f_i$};
        \iftoggle{article} {
            \node at (-2.4cm,1.3cm) (A) {(A)};
        }{}
  \end{scope}
    \begin{customlegend}[legend entries={$f_a+f_i$,$f_b$,$\dot{X}=f_a+f_i-f_b$},legend style={right=3cm of orig,anchor=west,name=legend1}]
      \addlegendimage{autocatacyc,fill=black!50!red,sharp plot,line width=1pt}
      \addlegendimage{branchout,fill=black!50!red,sharp plot,line width=1pt}
      \addlegendimage{sumcolor,fill=black!50!red,sharp plot,dashed,line width=1pt}
    \end{customlegend}
    \begin{scope}[shift={(-2.5cm,-6cm)},node distance = 1cm]

        \begin{axis}[name=plot1,axis x line=middle,axis y line=left,xlabel near ticks,ylabel near ticks,xmin=0,ymin=-2.5,xmax=2.9,ymax=5.9,xlabel={[$X$]},ylabel={flux},samples=60,width=5cm,height=6cm,yticklabels={,,},xticklabels={,,},tick label style={major tick length=0pt}]
    \addplot[domain=0:4,autocatacyc,thick] {3*x/(0.5+x)+\influx};
    \addplot[domain=0:4,branchout,thick] {5*x/(1+x)};
    \addplot[domain=0:4,sumcolor,thick,dashed] {3*x/(0.5+x)-5*x/(1+x)+\influx};
    \addplot[dashed,gray,thick] coordinates {(1,0) (1,2.5)};
    \node[right] (one) at (axis cs:0.1,5.5) {I};
    \node[right,align=left] (onetext) at (axis cs:0.05,4.7) {\scriptsize stable non-zero\\[-0.4em]\scriptsize steady state};
  \end{axis}

  \begin{axis}[name=plot2,axis x line=middle,axis y line=left,xlabel near ticks,ylabel near ticks,xmin=0,ymin=-2.5,xmax=2.9,ymax=5.9,xlabel={[$X$]},ylabel={flux},samples=60,at=(plot1.right of south east),anchor=left of south west,width=5cm,height=6cm,yticklabels={,,},xticklabels={,,},tick label style={major tick length=0pt}]
    \addplot[domain=0:4,autocatacyc,thick] {5*x/(1+x)+\influx};
    \addplot[domain=0:4,branchout,thick] {4*x/(1+x)};
    \addplot[domain=0:4,sumcolor,thick,dashed] {5*x/(1+x)-4*x/(1+x)+\influx};
    \node[right] (two) at (axis cs:0.1,5.5) {II};
    \node[right,align=left] (twotext) at (axis cs:0.05,5) {\scriptsize no steady state};
  \end{axis}

  \begin{axis}[name=plot3,axis x line=middle,axis y line=left,xlabel near ticks,ylabel near ticks,xmin=0,ymin=-2.5,xmax=2.9,ymax=5.9,xlabel={[$X$]},ylabel={flux},samples=60,at=(plot2.right of south east),anchor=left of south west,width=5cm,height=6cm,yticklabels={,,},xticklabels={,,},tick label style={major tick length=0pt}]
    \addplot[domain=0:4,autocatacyc,thick] {6*x/(2+x)+1.5*\influx};
    \addplot[domain=0:4,branchout,thick] {4*x/(0.4+x)};
    \addplot[domain=0:4,sumcolor,thick,dashed] {6*x/(2+x)-4*x/(0.4+x)+1.5*\influx};
    \addplot[dashed,gray,thick] coordinates {(1.2,0) (1.2,3)};
    \addplot[dashed,gray,thick] coordinates {(0.182,0) (0.182,1.25)};
    \node[right] (three) at (axis cs:0.1,5.5) {III};
    \node[right,align=left] (threetext) at (axis cs:0.05,4.7) {\scriptsize two non-zero\\[-0.4em]\scriptsize steady states};
  \end{axis}
    \end{scope}
\end{tikzpicture}

}
      \caption{\label{fig:inputcycle}
        (A) The effect of a fixed input flux, $f_i$, on the possible steady states of a simple autocatalytic cycle.
        A steady state occurs when $f_a+f_i=f_b$.
        If $V_{\max,b}>V_{\max,a}+f_i$ then there is always a single stable steady state (I).
        If $V_{\max,b}<V_{\max,a}+f_i$ then there can either be no steady states (II), or two steady states where the smaller one is stable (III).
      }
    \end{figure}

    In cases where $V_{\max,b}<f_i+V_{\max,a}$ either no steady states exist (Figure \ref{fig:inputcycle} (II)), or two positive steady states exist (Figure \ref{fig:inputcycle} (III)).
    The latter case implies that there exists a positive concentration $X$ that satisfies:

    \begin{equation*}
        \dot X = 0 \Rightarrow f_i + f_a(X) - f_b(X) = 0 \Rightarrow f_i+\frac{V_{\max,a}X}{K_{M,a}+X} = \frac{V_{\max,b}X}{K_{M,b}+X}
    \end{equation*}
  In this case, the lower concentration steady state will be stable.

  To conclude, input fluxes change the steady state(s) of autocatalytic cycles.
  When an input flux is present, an autocatalytic cycle can always achieve a non zero, stable steady state by tuning the expression levels of the enzymes forming the cycle.

    \iftoggle{elifesubmission} {
  Interestingly, we find that in the two autocatalytic cycles shown in Figure \ref{fig:realautocatal}-figure supplements 1 and 2, reactions that generate direct input flux into the cycle exist.
  In the ribose-5P assimilating autocatalytic cycle (Figure \ref{fig:realautocatal}-figure supplement 1), the rpi reaction serves as a shortcut, allowing input flux directly from ribose-5P into the cycle.
  In the glycerone-phosphate assimilating cycle (Figure \ref{fig:realautocatal}-figure supplement 2), the tpi reaction similarly serves as such a shortcut.} {
  Interestingly, we find that in the two autocatalytic cycles shown in figures \ref{fig:extrasamps1} and \ref{fig:extrasamps2}, reactions that generate direct input flux into the cycle exist.
  In the ribose-5P assimilating autocatalytic cycle (Figure \ref{fig:extrasamps1}), the rpi reaction serves as a shortcut, allowing input flux directly from ribose-5P into the cycle.
  In the glycerone-phosphate assimilating cycle (Figure \ref{fig:extrasamps2}), the tpi reaction similarly serves as such a shortcut.}
  In these two cases, these shortcuts relax the constraints imposed by strict use of the corresponding autocatalytic cycles as they prevent zero from being a stable steady state concentration.
    Another example of the effects of addition of an input flux to an autocatalytic cycle is the input flux of fructose-6-phosphate from the catabolism of starch into the CBB cycle.
    This input flux can be used to 'kick start' the cycle even without using the intermediate metabolites of the cycle.

    \subsubsection{Reversible autocatalytic reaction allows enzyme expression level tuning to always achieve positive stable steady state}
    The simple model assumed both the autocatalytic and the branch reactions are irreversible.
    If the autocatalytic reaction is reversible, then it must be driven by the displacement from thermodynamic equilibrium of the concentration of $A$ versus the concentration of $X$.
    Therefore, for any concentration of $A$, a concentration of $X$ exists such that $f_a(X)=0<f_b(X)$.
    It then follows that a sufficient condition for a positive steady state to exist is that at $X=0$, $\dot{X}>0$, which implies that 
    \begin{equation*}
    \frac{\partial f_a}{\partial X}\Big\vert_{X=0}>\frac{V_{\max,b}}{K_{M,b}}
    \end{equation*}
    which can always be satisfied by high expression of the autocatalytic enzyme, increasing $V_{\max,a}$.
    For this case it therefore follows that for any concentration of $A$, a minimal value for $V_{\max,a}$ exists, above which a positive steady state is achieved.

    \subsubsection{Ignore - Reversible branch reaction - incomplete section}
    The simple model assumed both the autocatalytic and the branch reactions are irreversible.
    Here we assume the branch reaction is reversible, and let $Y$ denote its product.
    For simplicity, we further assume that $K_{\text{eq}}=1$, noting that this assumption can always be satisfied by measuring the concentration of $Y$ in units of $K_{\text{eq}}X$.
    We recall that the reversible Michaelis-Menten equation states that:
   \begin{equation*}
       f_b=\frac{V_{\max,b}(X-Y)}{K_X+X+\frac{K_X}{K_Y}Y}
   \end{equation*}

   We first note that if $Y$ is removed rapidly, such that $Y << X$, making the reversible reaction proceed far from equilibrium, then the system is identical to the irreversible case analyzed above, resulting in the same conditions on existence and stability of non-zero steady state.
   Another observation is that for any positive constant concentration of $Y$, $X=0$ is not a steady state as the branch reaction will replenish $X$.
   In this case, the system will always reach a steady state if $V_{\max,b}>V_{\max,a}$ as the concentration of $X$ will increase until $f_b=f_a$.
   However, this case has limited biological relevance, as $Y$ most probably serves as the substrate of subsequent reactions.

   The most biologically relevant case is therefore when a third reaction, $f_c$ irreversibly consumes $Y$.
   While assuming $f_c$ follows irreversible Michaelis-Menten kinetics is analytically tractable, the analysis is much simpler and as informative under the assumption that $f_c=CY$ for some constant $C$.
   This case is equivalent to assuming that $V_{\max,c} >> V_{\max,a},V_{\max,b}$.

    For this case we get that at steady state $f_a=f_c$ so that:
    \begin{equation}
        \label{eq:branchrevy}
        \frac{V_{\max,a}X}{K_{M,a}+X}=CY \Rightarrow Y=\frac{\frac{V_{\max,a}}{C}X}{K_{M,a}+X}
    \end{equation}
    Furthermore, as at the steady state $f_a=f_b$, we get that:
    \begin{equation*}
        \frac{V_{\max,a}X}{K_{M,a}+X}=\frac{V_{\max,b}(X-Y)}{K_X+X+\frac{K_X}{K_Y}Y}
    \end{equation*}
    Substituting $Y$ from equation \ref{eq:branchrevy} gives:
    \begin{align*}
        &\frac{V_{\max,a}X}{K_{M,a}+X}=\frac{V_{\max,b}(X-\frac{\frac{V_{\max,a}}{C}X}{K_{M,a}+X})}{K_X+X+\frac{K_X}{K_Y}\frac{\frac{V_{\max,a}}{C}X}{K_{M,a}+X}} \Rightarrow \\
        &V_{\max,a}=\frac{V_{\max,b}(K_{M,a}+X-\frac{V_{\max,a}}{C})}{K_X+X+\frac{K_X}{K_Y}\frac{\frac{V_{\max,a}}{C}X}{K_{M,a}+X}} \Rightarrow \\
        &\frac{K_X}{K_Y}\frac{\frac{V_{\max,a}^2}{C}X}{K_{M,a}+X}=V_{\max,b}X-V_{\max,a}X+V_{\max,b}K_{M,a}-V_{\max,a}K_X-\frac{V_{\max,a}V_{\max,b}}{C} \Rightarrow \\
        &0=X^2+\frac{2K_{M,a}V_{\max,b}-(K_{M,a}+K_X)V_{\max,a}-\frac{K_XV_{\max,a}^2}{K_YC}-\frac{V_{\max,a}V_{\max,b}}{C}}{V_{\max,b}-V_{\max,a}}X+\\
        &\frac{K_{M,a}(V_{\max,b}K_{M,a}-V_{\max,a}K_X-\frac{V_{\max,a}V_{\max,b}}{C})}{V_{\max,b}-V_{\max,a}}
    \end{align*}

    Albeit intimidating, this quadratic equation can be used to derive the conditions for existence of a positive steady state.
    Only if both of the roots of this equation are negative, no positive steady state exists.
    We recall that the two roots of a quadratic equation with $a=1$ are negative iff $b>0$ and $c>0$.
    We can therefore divide the existence conditions into cases:
    \begin{itemize}
        \item If $V_{\max,b}>V_{\max,a}$: If $K_{M,a}>V_{\max,a}/C$ (implying $C>V_{\max,a}/K_{M,a}$, qualitatively suggesting rapid removal of $Y$) then an upper bound on $V_{\max,b}$ exists, above which the two solutions are negative, implying no positive steady state exists.
            If $C<V_{\max,a}/K_{M,a}$ then for any $V_{\max,b}>V_{\max,a}$, $c<0$ implying a positive steady state exists.
        \item If $V_{\max,b}<V_{\max,a}$ then an upper bound on $V_{\max,a}$ always exists, such that if $V_{\max,a}$ exceeds this upper bound, no positive steady state solution is possible.
            Moreover, if $K_X>K_A$ then the upper bound on $V_{\max,a}$ is negative, meaning no positive steady state is attainable under these conditions.
            Also in this case, if $c>0$ then $b>0$ meaning at most one positive steady state is attainable.
            As we show below, such a steady state violates the stability conditions we derive.
    \end{itemize}
    To sum up, a sufficient condition for a positive steady state is to tune expression levels such that: $V_{\max,b}>V_{\max,a}>K_{M,a}C$.
    
    We now turn to analyze the stability of the steady state.
    For a steady state to be stable, the eigenvalues of the Jacobian matrix must have negative real values.
    In our system it holds that
    \begin{align*}
        \dot{X} & =f_a-f_b \\
        \dot{Y} & =f_b-f_c
    \end{align*}

 We use the following notation:
\begin{align*}
  \alpha & =\frac{df_a}{dX}=\frac{V_{\max,a}K_{M,a}}{(K_{M,a}+X)^2} \\
  \beta_x & =\frac{\partial f_b}{\partial X}= \frac{V_{\max,b}(K_X+Y(1+\frac{K_X}{K_Y}))}{(K_X+X+\frac{K_XY}{K_Y})^2}
\\
  \beta_y & =\frac{\partial f_b}{\partial Y}= \frac{-V_{\max,b}(K_X+X(1+\frac{K_X}{K_Y}))}{(K_X+X+\frac{K_XY}{K_Y})^2} \\
  \frac{df_c}{dY}&=C
\end{align*}

We note that for MM kinetics and positive concentrations it holds that:
    \begin{align*}
        \alpha & >0 \\
        \beta_x & >0 \\
        -\beta_y & >\beta_x \\
    \end{align*}

We can use this notation to write the Jacobian matrix as:
   \begin{equation*}
        J=
        \begin{pmatrix}
            \alpha-\beta_x & -\beta_y \\
            \beta_x & \beta_y-C
        \end{pmatrix}
    \end{equation*}
which gives a characteristic polynomial of:
\begin{equation*}
    (\alpha-\beta_x-\lambda)(\beta_y-C-\lambda)+\beta_y\beta_x=0
\end{equation*}

 In order for the real values of the roots of the characteristic polynomial to be negative it must hold that $b>0$ and $c>0$, where $b$ and $c$ are now the coefficients of the quadratic equation $a\lambda^2+b\lambda+c=0$.
We therefore get that:
\begin{align*}
b & =\beta_x-\alpha-\beta_y+C>0\\
c & =(\alpha-\beta_x)(\beta_y-C)+\beta_y\beta_x>0
\end{align*}
We note that as $\beta_y\beta_x<0$ and $\beta_y-C<0$, it is a necessary, but not sufficient, condition that $\alpha<\beta_x$ for the second inequality to hold.
This is, however, a sufficient condition for the first inequality to hold.
The sufficient condition for the second inequality to hold (that also satisfies the first inequality) is that:
\begin{equation*}
  \beta_x-\alpha>\beta_y\beta_x/(\beta_y-C)=\frac{\beta_x}{1-\frac{C}{\beta_y}}\Rightarrow (1-\frac{\alpha}{\beta_x})(1-\frac{C}{\beta_y})>1
\end{equation*}
implying that the smaller the ratio between $\beta_x$ and $\alpha$ is, the larger $C$ has to be to ensure stability.
Moreover, as $-\beta_y >\beta_x$ it follows that $C>\alpha$.
We can use the last inequality to get that:
\begin{equation*}
    C>\alpha \Rightarrow C>\frac{V_{\max,a}K_{M,a}}{(K_{M,a}+X)^2}
\end{equation*}
which can always be satisfied if $X$ is large enough.

Our conclusions are therefore as follows:
As in the irreversible case, $V_{\max,b}>V_{\max,a}$.
For large values of $C$, conditions equivalent to those of the irreversible case apply, that is $V_{\max,b}/K_X$ is upper bounded by a term that approaches $V_{\max,a}/K_{M,a}$ as $C$ increases.
Stability conditions are also met in this case, as $\beta_x>\alpha$ implying that $C>-\alpha\beta_y/(\beta_x-\alpha)$ suffices to satisfy stability.

For small values of $C$, a steady state point always exists.
From the fact that $V_{\max,b}>V_{\max,a}$ we can deduce that at the steady state point $\beta_x>\alpha$.
However, as $C$ is small, the steady state point must be substituted into the derivatives to find out if the steady state is stable.
   \subsubsection{Stability analysis for multiple-reaction cycles}
    It is useful to extend the simple criteria we derived to more complex autocatalytic cycles.
    In such cycles the criteria for existence of a steady state become complex and detail specific.
    We therefore focus on evaluating stability of such cycles, under the assumption that a non-zero steady state exists, which is often known from experimental measurements.
    The most straightforward extension is for cycles with more than one intermediate metabolite.

    We start by writing the relevant equations for the two-metabolite autocatalytic cycle depicted in figure \ref{fig:multiple}A.
    In this system there are two intermediate metabolites, $X_1$ and $X_2$, two reactions that form the cycle, $f_{a_1}$ and $f_{a_2}$, and two branch reactions, $f_{b_1}$ and $f_{b_2}$.
    We assume, without loss of generality, that the autocatalytic reaction (the reaction that increases the amount of carbon in the cycle) is $f_{a_2}$ and that the autocatalysis is in a $1:2$ ratio.
    The equations describing the dynamics of the system are thus:

    \begin{align*}
        \dot{X_1} & =2f_{a_2}-f_{a_1}-f_{b_1} \\
        \dot{X_2} & =f_{a_1}-f_{a_2}-f_{b_2}
    \end{align*}
    We note that in steady state, where $\dot{X_1}=\dot{X_2}=0$, it must hold that $f_{b_1}+f_{b_2}=f_{a_2}$, as the total outgoing flux must balance the total increase of intermediate metabolites due to autocatalysis.
    Given that a steady state of the system exists for some value $(X_1^*,X_2^*)$, we can evaluate the condition for stability.
    In multi-variable systems, stability dictates that the real part of the eigenvalues of the Jacobian matrix must all be negative.
    We define $\alpha_i=\frac{\partial f_{a_i}}{\partial X_i}$ and $\beta_i=\frac{\partial f_{b_i}}{\partial X_i}$ for $i=1,2$.
    We note that as we assume Michaelis Menten kinetics, $\alpha_i>0$ and $\beta_i \geq 0$, where $\beta_i=0$ is the case where there is no flux branching out at $i$.
    We then get that the Jacobian matrix is:

    \begin{equation*}
        J=
        \begin{pmatrix}
            -(\alpha_1+\beta_1) & 2\alpha_2 \\
            \alpha_1 & -(\alpha_2+\beta_2)
        \end{pmatrix}
    \end{equation*}
    Solving for the characteristic polynomial gives:

    \begin{align*}
        0 & =(\lambda+\alpha_1+\beta_1)(\lambda+\alpha_2+\beta_2)-2\alpha_1\alpha_2 \\
        & = \lambda^2+(\alpha_1+\beta_1+\alpha_2+\beta_2)\lambda+(\alpha_1+\beta_1)(\alpha_2+\beta_2)-2\alpha_1\alpha_2
    \end{align*}
    that has two negative roots when:

    \begin{equation*}
        (\alpha_1+\beta_1)(\alpha_2+\beta_2)-2\alpha_1\alpha_2>0
    \end{equation*}
    which is satisfied if $\beta_1 > \alpha_1$ or $\beta_2 > \alpha_2$.
    Therefore, if either $\beta_1 > \alpha_1$ or $\beta_2 > \alpha_2$ at the steady state, then the steady state is stable.

\begin{figure}[!htb]
  \centering
      \begin{tikzpicture}[>=latex']
    \iftoggle{poster} {}
    {
    \iftoggle{article} {
    \begin{scope}[shift={(-5cm,0cm)},node distance = 2cm]
        \node (X1) {$X_1$};
        \node[left=of X1]  (X2) {$X_2$};
        \draw [->,line width=1pt,autocatacyc] (X1.south) [out=-90,in=-90] to  node [pos=0.5,above] (fa1) {$f_{a_1}$} (X2.south);
        \draw [->,line width=1pt,autocatacyc] (X2.north) [out=90,in=90] to node [pos=0.5,shape=coordinate,yshift=0.5pt] (assimpt) {} node [pos=0.6,above] (fa2) {$f_{a_2}$} (X1.north);
        \draw [line width=1pt,assimcol] (assimpt) arc (-90:-140:0.7cm) node [pos=1,above] (e) {$A$};
        \draw [->,line width=1pt,branchout] (X1.south) arc (190:270:1cm) node [pos=0.3,right,xshift=1mm] {$f_{b_1}$};
        \draw [->,line width=1pt,branchout] (X2.north) arc (10:90:1cm) node [pos=0.3,left] {$f_{b_2}$};
    \end{scope}}
    {}
}
    \begin{scope}[shift={(0cm,0cm)},node distance = 1cm]
        \node (X1) {$X_1$};
        \node[below left=of X1]  (X2) {$X_2$};
        \node[above left=of X1]  (Xn) {$X_n$};
        \draw [->,line width=1pt,autocatacyc] (X1.south) [out=-90,in=0] to  node [pos=0.7,right,xshift=1mm] (fa1) {$f_{a_1}$} (X2.east);
        \draw [->,line width=1pt,autocatacyc] (Xn.east) [out=0,in=90] to node [pos=0.4,shape=coordinate] (assimpt) {} node [pos=0.4,right,xshift=1mm] (fan) {$f_{a_n}$}(X1.north);
        \draw [line width=1pt,assimcol] (assimpt) arc (-120:-170:0.7cm) node [pos=1,above] (e) {$A$};
        \draw [->,line width=1pt,branchout] (X1.south) arc (190:270:1cm) node [pos=0.3,right,xshift=1mm] {$f_{b_1}$};
        \draw [->,line width=1pt,branchout] (X2.south) arc (-10:-90:1cm) node [pos=0.3,left] {$f_{b_2}$};
        \draw [->,line width=1pt,branchout] (Xn.north) arc (10:90:1cm) node [pos=0.3,left] {$f_{b_n}$};
        \draw [->,line width=1pt,autocatacyc,dashed] (X2.west) [out=180,in=-90] to ($(X1)+(-3,0)$) node [right] (fmid) {$f_{a_2}\dots f_{a_{n-1}}$} to [out=90,in=180] (Xn.west);
    \end{scope}
    \iftoggle{poster} {}
    {
    \iftoggle{article} {
        \node [shift={(-8cm,3cm)}] (A) {(A)};
        \node [right of=A,xshift=4cm] (B) {(B)};
    }{}
}
\end{tikzpicture}


     \caption{
     Generalization of analysis to multiple-reaction autocatalytic cycles with a single assimilating reaction.
     (A) A two reaction system.
     (B) A generic $n$-reaction system.
     The system is at steady state when the total consumption of intermediate metabolites by the branch reactions is equal to the flux through the autocatalytic reaction.
     A sufficient condition for the stability of a steady state in these systems is that the derivative of at least one branch reaction with respect to the substrate concentration is larger than the derivative of the equivalent autocatalytic reaction at the steady state concentration.
     Given the connection between derivatives of fluxes and saturation levels of reactions (see Methods), this condition implies that a sufficient condition for the stability of a steady state is that the saturation level of at least one branch reactions is smaller than the saturation level of the corresponding autocatalytic reaction.
 }
     \label{fig:multiple}
\end{figure}

    The two-metabolites cycle case can be easily extended to a larger number of intermediate metabolites and reactions, as is depicted in figure \ref{fig:multiple}B.
    We again arbitrarily assume that the autocatalytic reaction is the last reaction, $f_{a_n}$, and that the autocatalysis is in a $1:2$ ratio.

    In this case, steady state implies that the concentration of each intermediate metabolite is conserved, meaning that for all $i>1$:
    \begin{equation}
        \label{eq:complexdecrease}
      \dot{X_i}=0 \Rightarrow f_{a_{i-1}}-f_{a_i}-f_{b_i}=0 \Rightarrow f_{a_{i-1}} \geq f_{a_i}
    \end{equation}
    (for $i=1$, as $f_{a_n}$ is the autocatalytic reaction, we get that $2*f_{a_{n}} \geq f_{a_1}$).
    Also, because at steady state the total outgoing flux from the cycle must balance the total incoming flux into the system, which is the amount of autocatalysis carried out by $f_{a_n}$, we get that:

    \begin{equation*}
      \sum_{i=1}^n f_{b_i}=f_{a_n}
  \end{equation*}
implying that for all $i$:

    \begin{equation}
        \label{eq:complexsmallbranch}
        f_{b_i} \leq f_{a_n}
  \end{equation}

    Deriving a stability criterion for the multiple-reaction case, we get that in this case a steady state is stable if there exists $i$ such that $\beta_i > \alpha_i$ (see methods).
    Moreover, we qualitatively find that convergence to the steady state is faster when $\beta_i > \alpha_i$ for multiple $i$'s (see methods).

    To conclude, for the straightforward extension of the simple model to multiple reactions with a single autocatalytic reaction, steady state implies that for all $i$:

    \begin{equation}
        f_{b_i} \leq f_{a_n} \leq f_{a_i}
      \label{eq:complexstst}
    \end{equation}
    Where the left inequality is due to Equation \ref{eq:complexsmallbranch} and the right inequality is due to Equation \ref{eq:complexdecrease}.

    A sufficient condition for such a steady state point to be stable is that at the steady state point there exists at least one branching point $i$ at which the derivative of the branch reaction is larger than the derivative of the equivalent autocatalytic reaction:

    \begin{equation}
      \beta_i > \alpha_i
      \label{eq:complexstab}
    \end{equation}
    Using the connection between derivatives of reactions and saturation levels (see Methods), we conclude that a sufficient condition for stable autocatalytic fluxes is that the saturation of a single branch reaction, denoted $S(f)$, will be lower than the saturation of the corresponding autocatalytic reaction:

    \begin{equation}
        S(f_{b_i}) < S(f_{a_i})
      \label{eq:complexsat}
    \end{equation}
    This condition also dictates that the affinity of the branch reaction to the intermediate metabolite of the cycle it consumes must be lower than the affinity of the corresponding recycling reaction of the cycle.

    While having a single branch point for which equation \ref{eq:complexsat} holds is a sufficient condition for stability, we note that the larger the amount of branch points satisfying equation \ref{eq:complexsat}, the more robust the steady state point will be to perturbations, as such branch points reduce the propagation of deviations along the cycle (see methods).

    \subsubsection{Allosteric regulation affecting the affinities of enzymes can optimize network performance}
    The limit on the affinity of the branch reaction imposed by the stability requirement may be suboptimal under other flux modes.
    Furthermore, allosteric regulation can be used to accelerate the rate at which an autocatalytic cycle converges to its stable steady state mode.
    While many allosteric regulation schemes exist \cite{Leskovac2003-li}, all of these schemes affect the affinity of the regulated enzyme, and some of these schemes also affect the maximal rate.
    We analyze the qualitative expected beneficial regulation motifs for autocatalytic cycles.

    We consider three classes of regulating metabolites: the assimilated metabolite, internal metabolites of the cycle, and the products of the branch reactions out of the cycle.
    We further consider three regulation targets: the enzymes catalyzing the recycling and autocatalytic reactions, the branch enzymes, and enzymes catalyzing reactions from branch reactions products back into the cycle.
    We note that the last category of reactions, catalyzing reactions from products of branch reactions back into cycle intermediates, can intuitively be regarded as mirror images of the branch reactions as the net flux out of the cycle is the flux trough the branch reactions minus the flux trough these reverse reactions.
    From equation \ref{eq:xstst}, assuming the parameters satisfy the stable steady state conditions of equation \ref{eq:stabconds}, we get that the steady state concentration, $X^*$, increases as $K_{M,b}$ increases and $V_{\max,b}$ decreases, corresponding to inhibition of $f_b$, and that $X^*$ decreases as $K_{M,a}$ increases and $V_{\max,a}$ decreases, corresponding to inhibition of $f_a$ (see methods).
    Therefore, to increase $X^*$, and, as a result, to increase the autocatalytic and branch fluxes, the recycling and autocatalytic reactions should be activated, and the branch reactions should be inhibited.
    On the other hand, to decrease the autocatalytic and branch fluxes, the autocatalytic reactions should be inhibited and the branch reactions should be activated.

    To ensure stable, fast converging operation, we therefore expect the following interactions:
    \begin{itemize}
        \item High levels of the assimilated metabolite or the intermediate metabolites of the cycle should reduce autocatalytic flux and increase branch flux to avoid overflow of the cycle.
            Conversely, low levels of the assimilated metabolite or the intermediate metabolites of the cycle should increase autocatalytic flux and decrease branch flux to avoid drainage.
        \item High levels of branch products metabolites should reduce autocatalytic flux and increase branch reaction affinity and maximal flux, to reduce both the levels of the cycle intermediate metabolites and the flux branching out of the cycle.
    \end{itemize}

    As we show below, these predictions hold for existing autocatalytic cycles the function of which is known to be allosterically regulated.

    \subsubsection{Using different kinetics equations}
    Although we utilized the widely-used irreversible Michaelis-Menten kinetics equation to model enzyme kinetics, our results can be extended to different kinetic equations.
    Generally, two conditions must be met for a stable flux through an autocatalytic cycle to exist: (A) there should be a positive concentration of the intermediate metabolites for which the outgoing fluxes balance the autocatalytic fluxes, resulting in a steady state of fluxes, and, (B) at the steady state point at least one derivative of an outgoing reaction out of the cycle should be higher than the derivative of the corresponding autocatalytic reaction, as is implied by equation \ref{eq:complexstab}, to enforce stability in the presence of small perturbations.
    Therefore, these two conditions should be explicitly evaluated for every case with different kinetic equations and autocatalytic cycles topologies to assert whether it can carry stable fluxes or not.

    \subsection{Testing the predictions of the analysis with experimental data on functioning autocatalytic cycles}
    To evaluate the validity of our analysis of autocatalytic cycles we searched for growth conditions under which the autocatalytic cycles we identified in central carbon metabolism carry substantial flux in-vivo.
    We used recent in-vivo flux measurements in \emph{E.coli} from \cite{Gerosa2015-oq}.
    According to the data, two autocatalytic cycles carry substantial flux under at least one of the growth conditions measured:
    a cycle using the PTS transport system carries significant fluxes in growth on glucose and on fructose;
    the glyoxylate cycle carries significant flux in growth on galactose and on acetate.

    As noted above, we predict a key design principle for functioning autocatalytic cycles: that at least one branch reaction should be less saturated than the corresponding autocatalytic reaction.
    This requirement is sufficient, but not necessary, for the autocatalytic cycle to be at a stable steady state point.
    Moreover, having more than one branch point at which the branch reaction is less saturated than the autocatalytic reaction increases the robustness of the steady state flux in the cycle.
    To evaluate the saturation level of a reaction under a given condition, two values must be obtained:
    \begin{enumerate}
      \item The maximal flux capacity of the reaction under the given condition, $V_{\max}$.
        \item The actual flux through the reaction at the steady state, $f$.
    \end{enumerate}

    To estimate the maximal capacity of a reaction we followed the procedure described in \cite{Davidi2016-ga} (see methods).
    We used the data from \cite{Gerosa2015-oq} to identify the major branch points in each functioning cycle and the flux distributions in them under each of the relevant conditions.
    The results are presented in figure \ref{fig:branch} and are provided, with the relevant calculations, in the supplementary tables.

\begin{figure}[!htb]
\centering
\resizebox{1\linewidth}{!}{
    \begin{tikzpicture}
  \tikzset{
    figArrowStyle/.style={arrows={-{Stealth[inset=0pt,scale=#1,angle'=60]}}},
    figArrowStyle/.default=0.25
  }
  \tikzset{
    capArrowStyle/.style={arrows={-{Stealth[inset=0pt,scale=0.25,angle'=60,color=#1]}}},
    capArrowStyle/.default=autocatacycfl
    }

  \tikzset{
    ratioRect/.style={rectangle,fill=graybg,rounded corners=2pt}}

  \newcommand{\coloredRatio}[2]}{\mathbf{\color{autocatacyc}#2\%}}$}}
  }

\iftoggle{article} {
    \pgfmathsetlength{\nodedist}{1cm}
    \renewcommand{\fontsizedef}{\normalsize}
    \renewcommand{\ratiosizedef}{\Large}
    \renewcommand{\coloredRatio}[2]}{\mathbf{\color{autocatacyc}#2\%}}$}}
  }
}{
    \only<3-> {
        \pgfmathsetlength{\nodedist}{1cm}
        \renewcommand{\fontsizedef}{\normalsize}
        \renewcommand{\ratiosizedef}{\Large}
        \renewcommand{\coloredRatio}[2]}{\mathbf{\color{autocatacyc}##2\%}}$}}
      }
    }
}

    \pgfmathsetlength{\headlinedist}{0.3cm}
    %prediction
  \begin{scope}[shift={(-6cm,0cm)}]
      \node[ratioRect] (prediction) {\textbf{Prediction:} $\mathbf{\color{branchout}XX\%}\leq\mathbf{\color{autocatacyc}YY\%}$};
  \end{scope}
  %Galactose
  \begin{scope}[shift={(-6cm,-7.5cm)},visible on=<3->,font=\fontsizedef]
    \def\galmaxflux{0.5mm}
    \def\galglt{1.52*\galmaxflux}
    \def\galcapvalglt{20}
    \def\galcapglt{\galglt/\galcapvalglt*100}
    \def\galacn{1.52*\galmaxflux*1.5}
    \def\galacea{1.02*\galmaxflux}
    \def\galcapvalacea{30}
    \def\galcapacea{\galacea/\galcapvalacea*100}
    \def\galaceb{1.02*\galmaxflux}
    \def\galsdh{1.26*\galmaxflux}
    \def\galfum{1.26*\galmaxflux}
    \def\galmdh{2.28*\galmaxflux}
    \def\galpck{0.85*\galmaxflux}
    \def\galcapvalpck{25}
    \def\galcappck{\galpck/\galcapvalpck*100}
    \def\galicd{0.5*\galmaxflux}
    \def\galcapvalicd{10}
    \def\galcapicd{\galicd/\galcapvalicd*100}

    \node[] (galactose) {\textbf{Galactose input}};
    \node[metaboliteStyle,inputcol,below=\headlinedist of galactose] (aca) {aca};
    \node[shape=coordinate,below=of aca] (dummyglta) {};
    \node[metaboliteStyle,left=of dummyglta] (oaa) {oaa};
    \node[metaboliteStyle] at (aca -| oaa) (pep) {pep};
    \node[metaboliteStyle,right=of dummyglta] (cit) {cit};
    \node[metaboliteStyle,right=of cit] (icit) {icit};
    \node[metaboliteStyle,below=of icit.center] (akg) {akg};
    \node[metaboliteStyle,below=of akg.center] (sca) {sca};
    \node[metaboliteStyle,below=of oaa.center] (mal) {mal};
    \node[metaboliteStyle,below=of mal.center] (fum) {fum};
    \node[metaboliteStyle,right=of mal] (glx) {glx};
    \node[metaboliteStyle,right=of fum] (suc) {suc};
    \path[] (oaa) -- (cit) node [pos=0.85,shape=coordinate] (midglta) {};
    \draw[line width=\galcapglt,autocatacyc!40] ([yshift=-0.35*\galglt]oaa.east) -- ([yshift=-0.35*\galglt]midglta) ;

    \node[ratioRect,anchor=east] at (oaa.west) (galb1) {\coloredRatio{\galcapvalpck}{\galcapvalglt}};
    \node[ratioRect,anchor=south] at (icit.north) (galb2) {\coloredRatio{\galcapvalicd}{\galcapvalacea}};

    \draw[line width=\galglt,autocatacycfl] ([yshift=-0.35*\galglt]oaa.east) -- ([yshift=-0.35*\galglt]midglta);
    \draw[figArrowStyle,line width=\galglt*1.5,autocatacycfl] (midglta) -- (cit);
    \draw[capArrowStyle=branchoutfl,line width=\galcappck,branchout!40] (oaa) -- (pep);
    \draw[figArrowStyle,line width=\galpck,branchoutfl] (oaa) -- (pep);
    \draw [inputcol,line width=0.5*\galglt] (aca) [out=-70,in=180] to ([yshift=0.35*\galglt]midglta);
    \draw[figArrowStyle,line width=\galacn,autocatacycfl] (cit) -- (icit);
    \path[] (icit.south west) -- (suc) node [pos=0.3,shape=coordinate] (midacea) {};
    \draw[line width=\galcapacea*1.5,autocatacyc!40] (icit.south west) -- (midacea);
    \draw[capArrowStyle,line width=\galcapacea,autocatacyc!40] ([xshift=\galcapacea*0.35]midacea) -- ([xshift=\galcapacea*0.4]suc);
    \draw[capArrowStyle,line width=\galcapacea,autocatacyc!40] ([xshift=\galcapacea*0.38]midacea) -- ([xshift=\galcapacea*0.4]suc);
    \draw[capArrowStyle,line width=\galcapacea*0.5,autocatacyc!40] ([shift={(-\galcapacea*0.35,\galcapacea*0.35)}]midacea) [out=220,in=0] to (glx);
    \draw[capArrowStyle,line width=\galcapacea*0.5,autocatacyc!40] ([shift={(-\galacea*0.35,\galacea*0.35)}]midacea) [out=220,in=0] to (glx);
    \draw[line width=\galacea*1.5,autocatacycfl] (icit.south west) -- (midacea);
    \draw[figArrowStyle,line width=\galacea,autocatacycfl] ([xshift=\galacea*0.4]midacea) -- ([xshift=\galacea*0.4]suc);
    \draw[figArrowStyle,line width=\galacea*0.5,autocatacycfl] ([shift={(-\galacea*0.35,\galacea*0.35)}]midacea) [out=220,in=0] to (glx);
    \draw[capArrowStyle=branchoutfl,line width=\galcapicd*1.5,branchout!40] (icit) -- (akg);
    \draw[figArrowStyle,line width=\galicd*1.5,branchoutfl] (icit) -- (akg);
    \draw[->] (akg) -- (sca);
    \draw[->] (sca) -- (suc);
    \draw[figArrowStyle,line width=\galsdh,autocatacycfl] (suc) -- (fum);
    \draw[figArrowStyle,line width=\galfum,autocatacycfl](fum) -- (mal);
    \path[] (glx) -- (mal) node [pos=0.85,shape=coordinate] (midaceb) {};
    \draw[line width=\galaceb*0.5,autocatacycfl] ([yshift=-0.25*\galaceb]glx) -- ([yshift=-0.25*\galaceb]midaceb);
    \draw[inputcol,line width=0.5*\galaceb] ([xshift=-0.5mm]aca.south) [out=-90,in=0] to ([yshift=0.25*\galaceb]midaceb);
    \draw[figArrowStyle,line width=\galaceb,autocatacycfl] (midaceb) -- (mal);
    \draw[figArrowStyle,line width=\galmdh,autocatacycfl] (mal) -- (oaa);
  \end{scope}

  %acetate
  \begin{scope}[shift={(-6cm,-1.5cm)},node distance=\nodedist,font=\fontsizedef]
    \def\acemaxflux{0.2mm}
    \def\aceglt{8.83*\acemaxflux}
    \def\acecapvalglt{70}
    \def\acecapglt{\aceglt/\acecapvalglt*100}
    \def\aceacn{8.83*\acemaxflux*1.5}
    \def\aceacea{4.14*\acemaxflux}
    \def\acecapvalacea{100}
    \def\acecapacea{\aceacea}
    \def\aceaceb{4.14*\acemaxflux}
    \def\acesdh{8.4*\acemaxflux}
    \def\acefum{8.4*\acemaxflux}
    \def\acemdh{10.67*\acemaxflux}
    \def\acecapvalmdh{100}
    \def\acemae{1.87*\acemaxflux}
    \def\acecapvalmae{15}
    \def\acecapmae{\acemae/\acecapvalmae*100}
    \def\acepck{3.11*\acemaxflux}
    \def\acecapvalpck{75}
    \def\acecappck{\acepck/\acecapvalpck*100}
    \def\aceicd{4.7*\acemaxflux}
    \def\acecapvalicd{65}
    \def\acecapicd{\aceicd/\acecapvalicd*100}

    \node[] (acetate) {\textbf{Acetate input}};
    \node[metaboliteStyle,inputcol,below=\headlinedist of acetate] (aca) {aca};
    \node[shape=coordinate,below=of aca] (dummyglta) {};
    \node[metaboliteStyle,left=of dummyglta] (oaa) {oaa};
    \node[metaboliteStyle] at (aca -| oaa) (pep) {pep};
    \node[metaboliteStyle,right=of dummyglta] (cit) {cit};
    \node[metaboliteStyle,right=of cit] (icit) {icit};
    \node[metaboliteStyle,below=of icit.center] (akg) {akg};
    \node[metaboliteStyle,below=of akg.center] (sca) {sca};
    \node[metaboliteStyle,below=of oaa.center] (mal) {mal};
    \node[metaboliteStyle,below=of mal.center] (fum) {fum};
    \node[metaboliteStyle,right=of mal] (glx) {glx};
    \node[metaboliteStyle,left=of mal] (pyr) {pyr};
    \node[metaboliteStyle,right=of fum] (suc) {suc};
    \path[] (oaa) -- (cit) node [pos=0.85,shape=coordinate] (midglta) {};
    \draw[line width=\acecapglt,autocatacyc!40] ([yshift=-0.25*\aceglt]oaa.east) -- ([yshift=-0.25*\aceglt]midglta);
    \node[ratioRect,anchor=south east,visible on=<2->] at (oaa.north west) (aceb1) {\coloredRatio{\acecapvalpck}{\acecapvalglt}};
    \draw[line width=\aceglt,autocatacycfl] ([yshift=-0.25*\aceglt]oaa.east) -- ([yshift=-0.25*\aceglt]midglta);
    \draw[figArrowStyle,line width=\aceglt*1.5,autocatacycfl] (midglta) -- (cit);
    \draw[capArrowStyle=branchoutfl,line width=\acecappck,branchout!40] (oaa) -- (pep);
    \draw[figArrowStyle,line width=\acepck,branchoutfl] (oaa) -- (pep);
    \draw[capArrowStyle=branchoutfl,line width=\acecapmae,branchout!40] (mal) -- (pyr);
    \node[ratioRect,anchor=south east,visible on=<2->] at (mal.north west) (aceb2) {\coloredRatio{\acecapvalmae}{\acecapvalmdh}};
    \draw[figArrowStyle,line width=\acemae,branchoutfl] (mal) -- (pyr);
    \draw [inputcol,line width=0.5*\aceglt] (aca) [out=-70,in=180] to ([yshift=0.4*\aceglt]midglta);
    \draw[figArrowStyle,line width=\aceacn,autocatacycfl] (cit) -- (icit);
    \path[] (icit.south west) -- (suc) node [pos=0.3,shape=coordinate] (midacea) {};
    \draw[line width=\acecapacea*1.5,autocatacyc!40] (icit.south west) -- (midacea);
    \draw[capArrowStyle,line width=\acecapacea,autocatacycfl!40] ([xshift=\acecapacea*0.35]midacea) -- ([xshift=\acecapacea*0.4]suc);
    \node[ratioRect,anchor=south,visible on=<2->] at (icit.north) (aceb3) {\coloredRatio{\acecapvalicd}{\acecapvalacea}};
    \draw[capArrowStyle,line width=\acecapacea*0.5,autocatacyc!40] ([shift={(-\acecapacea*0.35,\acecapacea*0.35)}]midacea) [out=220,in=0] to (glx);
    \draw[line width=\aceacea*1.5,autocatacycfl] (icit.south west) -- (midacea);
    \draw[figArrowStyle,line width=\aceacea,autocatacycfl] ([xshift=\aceacea*0.4]midacea) -- ([xshift=\aceacea*0.4]suc);
    \draw[figArrowStyle,line width=\aceacea*0.5,autocatacycfl] ([shift={(-\aceacea*0.35,\aceacea*0.35)}]midacea) [out=220,in=0] to (glx);
    \draw[capArrowStyle=branchoutfl,line width=\acecapicd*1.5,branchout!40] (icit) -- (akg);
    \draw[figArrowStyle,line width=\aceicd*1.5,branchoutfl] (icit) -- (akg);
    \draw[->] (akg) -- (sca);
    \draw[->] (sca) -- (suc);
    \draw[figArrowStyle,line width=\acesdh,autocatacycfl] (suc) -- (fum);
    \draw[figArrowStyle,line width=\acefum,autocatacycfl](fum) -- (mal);
    \path[] (glx) -- (mal) node [pos=0.85,shape=coordinate] (midaceb) {};
    \draw[line width=\aceaceb*0.5,autocatacycfl] ([yshift=-0.25*\aceaceb]glx) -- ([yshift=-0.25*\aceaceb]midaceb);
    \draw[inputcol,line width=0.5*\aceaceb] ([xshift=-0.5mm]aca.south) [out=-90,in=0] to ([yshift=0.25*\aceaceb]midaceb);
    \draw[figArrowStyle,line width=\aceaceb,autocatacycfl] (midaceb) -- (mal);
    \draw[figArrowStyle,line width=\acemdh,autocatacycfl] (mal) -- (oaa);
 
  \end{scope}

  %Glucose
  \begin{scope}[shift={(2cm,0cm)},visible on=<3->,font=\fontsizedef]
    \def\glucmaxflux{0.35mm}
    \def\glucpgi{5.7*\glucmaxflux}
    \def\gluccapvalpgi{100}
    \def\gluccappgi{\glucpgi}
    \def\glucpfk{7.06*\glucmaxflux}
    \def\glucfba{7.06*\glucmaxflux}
    \def\gluctpi{7.06/2*\glucmaxflux}
    \def\glucgap{15.71/2*\glucmaxflux}
    \def\glucpgk{15.71/2*\glucmaxflux}
    \def\glucgpm{14.56/2*\glucmaxflux}
    \def\gluceno{14.56/2*\glucmaxflux}
    \def\glucpyk{2.49/2*\glucmaxflux}
    \def\gluccapvalpyk{35}
    \def\gluccappyk{\glucpyk/\gluccapvalpyk*100}
    \def\glucppc{2.45/2*\glucmaxflux}
    \def\gluccapvalppc{45}
    \def\gluccapppc{\glucppc/\gluccapvalppc*100}
    \def\gluczwf{3.92*\glucmaxflux}
    \def\gluccapvalzwf{55}
    \def\gluccapzwf{\gluczwf/\gluccapvalzwf*100}
    \def\glucpts{9.65*\glucmaxflux}
    \def\gluccapvalpts{100}
    \def\gluccapvalcomp{\pgfmathparse{round((\glucppc+\glucpyk)/(\gluccapppc+\gluccappyk)*20)*5}\pgfmathprintnumber{\pgfmathresult}}
    \def\gluccappts{\glucpts}

    \node[] (glucose) {\textbf{Glucose input}};
    \node[metaboliteStyle,below=\headlinedist of glucose] (g6p) {g6p};
    \node[metaboliteStyle,below=of g6p.center] (f6p) {f6p};
    %%% ptstop
    \node[shape=coordinate,left=1.1cm of g6p] (ptsmid) {};
    \node[metaboliteStyle,shift={(-11mm,2mm)},gray] at (f6p) (pyr1) {pyr};
    \node[metaboliteStyle,inputcol,left=of ptsmid] (gluc) {gluc};

    \node[metaboliteStyle,below=of f6p] (fbp) {fbp};
    \node[shape=coordinate,below=of fbp.center](fbamid) {};
    \node[metaboliteStyle,below left=of fbamid.center] (dhap) {dhap};
    \node[metaboliteStyle,below right=of fbamid] (gap) {gap};
    \node[metaboliteStyle,below=of gap.center] (bpg) {bpg};
    \node[metaboliteStyle,below=of bpg.center] (3pg) {3pg};
    \node[metaboliteStyle,below=of 3pg.center] (2pg) {2pg};
    \node[metaboliteStyle,below=of 2pg.center] (pep) {pep};
    \node[metaboliteStyle,right=of pep] (oaa) {oaa};
    \node[metaboliteStyle,below=of pep.center] (pyr) {pyr};
    \node[metaboliteStyle,right=of g6p] (6pgi) {6pgi};
    \draw[figArrowStyle,line width=\glucpgi,autocatacycfl] (g6p) -- (f6p);
    \draw[figArrowStyle,line width=\glucpfk,autocatacycfl] (f6p.south) -- (fbp.north);
    \draw [line width=\glucfba,autocatacycfl] (fbp) [out=-90,in=90] to (fbamid);
    \draw [figArrowStyle,line width=\glucfba/2,autocatacycfl] ([xshift=-\glucfba/4]fbamid) [out=-90,in=45] to (dhap);
    \draw [figArrowStyle,line width=\glucfba/2,autocatacycfl] ([xshift=\glucfba/4]fbamid) [out=-90,in=135] to (gap);
    \draw [figArrowStyle,line width=\gluctpi,autocatacycfl] (dhap) -- (gap);

    \draw[figArrowStyle,line width=\glucgap,autocatacycfl] (gap) -- (bpg);
    \draw[figArrowStyle,line width=\glucpgk,autocatacycfl] (bpg) -- (3pg);
    \draw[figArrowStyle,line width=\glucgpm,autocatacycfl] (3pg) -- (2pg);
    \draw[figArrowStyle,line width=\gluceno,autocatacycfl] (2pg) -- (pep);
    \draw[capArrowStyle=branchoutfl,line width=\gluccappyk,branchout!40] (pep) -- (pyr);
    \draw[figArrowStyle,line width=\glucpyk,branchoutfl] (pep) -- (pyr);
    \draw[capArrowStyle=branchoutfl,line width=\gluccapppc,branchout!40] (pep) -- (oaa);
    \node[ratioRect,anchor=south east] at (pep.north west) (glucb2) {\coloredRatio{\gluccapvalcomp}{\gluccapvalpts}};
    \draw[figArrowStyle,line width=\glucppc,branchoutfl] (pep) -- (oaa);
    \draw[capArrowStyle=branchoutfl,line width=\gluccapzwf,branchout!40] (g6p) -- (6pgi);
    \draw[figArrowStyle,line width=\gluczwf,branchoutfl] (g6p) -- (6pgi);
    \node[ratioRect,anchor=north west] at (g6p.south east) (glucb1) {\coloredRatio{\gluccapvalzwf}{\gluccapvalpgi}};
    \node[shape=coordinate,left=2.5cm of pep.center] (pts3) {};
    \draw[line width=\glucpts/2,autocatacycfl] (pep.west) -- (pts3);
    \draw[inputcol,line width=\glucpts] (gluc) -- (ptsmid);
    \node[shape=coordinate,left=2.5cm of pyr.center] (pts5) {};
    \draw[figArrowStyle,line width=\glucpts/2,autocataby] ([yshift=-3/4*\glucpts]ptsmid) [out=0,in=90] to (pyr1);
    \node[shape=coordinate,xshift=-2mm] at (f6p.center -| dhap.west) (ptstop) {};
    \node[shape=coordinate] at(ptstop |- 2pg.center) (ptsbottom) {};
    \draw[line width=\glucpts/2,autocatacycfl] (pts3) [in=-90,out=180] to (ptsbottom);
    \draw[line width=\glucpts/2,autocatacycfl] (ptsbottom) -- (ptstop);
    \draw[line width=\glucpts/2,autocatacycfl] (ptstop) [in=180,out=90] to ([yshift=-3/4*\glucpts]ptsmid);
    \draw[figArrowStyle,line width=\glucpts,autocatacycfl] (ptsmid) -- (g6p);
  \end{scope}

  %Fructose
  \begin{scope}[shift={(8.5cm,0cm)},visible on=<3->,font=\fontsizedef]
    \def\frucmaxflux{0.35mm}
    \def\frucfbp{2.46*\frucmaxflux}
    \def\fruccapvalfbp{100}
    \def\fruccapfbp{\frucfbp}
    \def\frucfba{5.87*\frucmaxflux}
    \def\fruccapvalfba{70}
    \def\fruccapfba{\frucfba/\fruccapvalfba*100}
    \def\fructpi{5.87/2*\frucmaxflux}
    \def\frucgap{13.46/2*\frucmaxflux}
    \def\frucpgk{13.46/2*\frucmaxflux}
    \def\frucgpm{12.6/2*\frucmaxflux}
    \def\fruceno{12.6/2*\frucmaxflux}
    \def\frucpyk{0.67/2*\frucmaxflux}
    \def\fruccapvalpyk{5}
    \def\fruccappyk{\frucpyk/\fruccapvalpyk*100}
    \def\frucppc{3.55/2*\frucmaxflux}
    \def\fruccapvalppc{50}
    \def\fruccapppc{\frucppc/\fruccapvalppc*100}
    \def\frucpts{8.33*\frucmaxflux}
    \def\fruccapvalpts{100}
    \def\fruccappts{\frucpts}
    \def\fruccapvalcomp{\pgfmathparse{round((\frucppc+\frucpyk)/(\fruccapppc+\fruccappyk)*20)*5}\pgfmathprintnumber{\pgfmathresult}}


    \node[] (fructose) {\textbf{Fructose input}};
    \node[metaboliteStyle,below=\headlinedist of fructose] (f6p) {f6p};
    \node[metaboliteStyle,below=of f6p] (fbp) {fbp};
    \node[shape=coordinate,below=of fbp.center](fbamid) {};
    %%% ptstop
    \node[shape=coordinate,left=1.1cm of fbp] (ptsmid) {};
    \node[metaboliteStyle,shift={(-11mm,0mm)},gray] at (fbamid) (pyr1) {pyr};
    \node[metaboliteStyle,inputcol,left=of ptsmid] (fruc) {fruc};

    \node[metaboliteStyle,below left=of fbamid.center] (dhap) {dhap};
    \node[metaboliteStyle,below right=of fbamid] (gap) {gap};
    \node[metaboliteStyle,below=of gap.center] (bpg) {bpg};
    \node[metaboliteStyle,below=of bpg.center] (3pg) {3pg};
    \node[metaboliteStyle,below=of 3pg.center] (2pg) {2pg};
    \node[metaboliteStyle,below=of 2pg.center] (pep) {pep};
    \node[metaboliteStyle,right=of pep] (oaa) {oaa};
    \node[metaboliteStyle,below=of pep.center] (pyr) {pyr};
    \draw[figArrowStyle,line width=\frucfbp,branchoutfl] (fbp)-- (f6p) ;
    \node[ratioRect,anchor=west] at (fbp.east) (frucb1) {\coloredRatio{\fruccapvalfbp}{\fruccapvalfba}};
    \node[at=(frucb1.north east)] (star) {*};
    \draw [line width=\fruccapfba,autocatacyc!40] (fbp) -- (fbamid);
    \draw [capArrowStyle,line width=\fruccapfba/2,autocatacyc!40] ([xshift=-\fruccapfba/4]fbamid) [out=-90,in=45] to (dhap);
    \draw [capArrowStyle,line width=\fruccapfba/2,autocatacyc!40] ([xshift=\fruccapfba/4]fbamid) [out=-90,in=135] to (gap);
    \draw [line width=\frucfba,autocatacycfl] (fbp) [out=-90,in=90] to (fbamid);
    \draw [figArrowStyle,line width=\frucfba/2,autocatacycfl] ([xshift=-\frucfba/4]fbamid) [out=-90,in=45] to (dhap);
    \draw [figArrowStyle,line width=\frucfba/2,autocatacycfl] ([xshift=\frucfba/4]fbamid) [out=-90,in=135] to (gap);
    \draw [figArrowStyle,line width=\fructpi,autocatacycfl] (dhap) -- (gap);

    \draw[figArrowStyle,line width=\frucgap,autocatacycfl] (gap) -- (bpg);
    \draw[figArrowStyle,line width=\frucpgk,autocatacycfl] (bpg) -- (3pg);
    \draw[figArrowStyle,line width=\frucgpm,autocatacycfl] (3pg) -- (2pg);
    \draw[figArrowStyle,line width=\fruceno,autocatacycfl] (2pg) -- (pep);
    \draw[capArrowStyle=branchoutfl,line width=\fruccappyk,branchout!40] (pep) -- (pyr);
    \draw[figArrowStyle,line width=\frucpyk,branchoutfl] (pep) -- (pyr);
    \draw[capArrowStyle=branchoutfl,line width=\fruccapppc,branchout!40] (pep) -- (oaa);
    \node[ratioRect,anchor=south east] at (pep.north west) (frucb2) {\coloredRatio{\fruccapvalcomp}{\fruccapvalpts}};
    \draw[figArrowStyle,line width=\frucppc,branchoutfl] (pep) -- (oaa);
    \node[shape=coordinate,left=2.5cm of pep.center] (pts3) {};
    \draw[line width=\frucpts/2,autocatacycfl] ([yshift=\frucpts/4]pep.west) -- (pts3);
    \node[shape=coordinate,left=2.5cm of pyr.center] (pts5) {};
    \draw[figArrowStyle,line width=\frucpts/2,autocataby] ([yshift=-3/4*\frucpts]ptsmid) [out=0,in=90] to (pyr1);
    \node[shape=coordinate,xshift=-2mm] at (fbamid -| dhap.west) (ptstop) {};
    \node[shape=coordinate] at(ptstop |- 2pg.center) (ptsbottom) {};
    \draw[] (pts3) [in=-90,out=180] to (ptsbottom);
    \draw[] (ptsbottom) [in=-90,out=90] to (ptstop);
    \draw[inputcol,line width=\frucpts] (fruc) -- (ptsmid);
    \draw[line width=\frucpts/2,autocatacycfl] (pts3) [in=-90,out=180] to (ptsbottom);
    \draw[line width=\frucpts/2,autocatacycfl] (ptsbottom) -- (ptstop);
    \draw[line width=\frucpts/2,autocatacycfl] (ptstop) [in=180,out=90] to ([yshift=-3/4*\frucpts]ptsmid);
    \draw[figArrowStyle,line width=\frucpts,autocatacycfl] (ptsmid) -- (fbp);
  \end{scope}
  \end{tikzpicture}


}
\caption{
  Major branch points and relative enzyme saturation in operating autocatalytic cycles.
  Solid arrow width represents carbon flux per unit time.
  Shaded arrow width represents maximal carbon flux capacity per unit time, given the expression level of the catalyzing enzyme.
  In all cases there is enough excess capacity in the branching reactions to prevent the cycle from overflowing.
  A $4\%$ flux from pep to biomass was neglected in growth under glucose and fructose.
  Only in one out of the 9 branch points observed (the branch point at fbp in growth under fructose), the outgoing reaction is significantly more saturated than the autocatalytic reaction.
  (*) A branch point at which the branching reaction is more saturated than the autocatalytic reaction, which may result from neglecting fructose transport directly as f6p when deriving fluxes (see text).
}
    \label{fig:branch}
\end{figure}

    Our results show that for any of the 4 functioning autocatalytic cycle cases, in at least one branch point the biomass generating branch reaction has a larger maximal flux capacity, and is considerably less saturated than the respective autocatalytic reaction, in accordance with our predictions.
    Moreover, out of 9 branch points analyzed, in 6 branch points the branching reactions were significantly less saturated than the autocatalytic reactions, in 2 branch points the saturation levels were similar, and only in one branch point the autocatalytic reaction was less saturated than the branching reaction.

    The branch point at which the autocatalytic reaction is less saturated than the branch reaction is the branch point from fructose-1,6-bisphosphate in growth on fructose as the carbon source.
    The high saturation of the branch reaction arises as a large flux is reported for the fbp reaction, whereas the corresponding enzyme is not highly expressed under this condition.
    The large reported flux through fbp arises due to assuming a single transport pathway for fructose, entering the cycle as fructose-1,6-bisphosphate.
    However, an alternative fructose transport pathway is known to occur in the concentration at which the measurements were made \cite{Kornberg1990-ft}.
    The alternative transport pathway produces fructose-6-phosphate from external fructose.
    Therefore, any flux through the alternative transport pathway should be directly deduced from the flux through fbp.
    Assuming $20\%$ of the consumed fructose uses this pathway suffices in order to balance the saturation levels at the fructose-1,6-bisphosphate branch point.

    We made two negative control analyses to examine whether other reasons do not underlie the trend we find.
    First, we compared the saturation levels at the same branch points in growth conditions at which the autocatalytic cycles do not function, but the reactions carry flux.
    We find that for these cases, only 4 out of 9 cases satisfy the low branch saturation condition (see supplementary tables).
    Second, we searched for branch points out of non-autocatalytic cycles and tested whether in such points branch reactions are also consistently less saturated than their corresponding cycle reactions.
    We found two flux-carrying cycles: the TCA cycle, carrying flux in glucose, fructose, and glycerol growth, and a cycle consisting of the pentose-phosphate pathway combined with gluconeogenesis, carrying flux in acetate, glycerol, and succinate growth.
    Out of the total 6 conditions-branch points cases, in 3 the branch reaction was less saturated than the cycle reaction, and in 3 the cycle reaction was less saturated than the branch reaction (see supplementary tables).
    We therefore conclude that, for cases that do not involve autocatalysis, the saturation of branch versus cycle reactions seems uniform.
    

    The consistently lower saturation values of biomass generating branch reactions demonstrate that the expressed enzymes have enough capacity to prevent the autocatalytic cycle from increasing the concentration of intermediate metabolites infinitely.
    Moreover, the lower saturation values of the biomass generating reactions suggest that at the steady state point their derivatives are higher, ensuring stable operation of the cycle.

    Another demonstration of the autocatalytic mechanism being at play is in the CBB cycle, which is not a part of the metabolic network of wild type \emph{E.coli}, and for which no flux measurements are available.
    This cycle has been recently introduced synthetically into \emph{E.coli} and was shown to carry flux in it, given further metabolic engineering of central carbon metabolism \cite{Antonovsky2016-jy}.
    The experimentally observed key evolutionary event enabling the functioning of the CBB cycle, was a mutation affecting the kinetic properties of the main branching reaction out of the CBB pathway, prs, reducing its affinity to its substrate, ribose-5p.
    The observed reduction in affinity of prs is directly in line with our predictions on the relationship between the affinity of branch reactions and the affinity of the corresponding autocatalytic reactions.

    The other examples of autocatalytic cycles we found did not carry flux in any of the conditions for which data was available.
    The pentose-phosphate cycle variants do not carry flux in any of the measured conditions, which is expected given that growth on ribose was not measured.
    The gluconeogenic FBA with ED pathway cycle also did not carry flux in any of the measured conditions.
    Although glycerol could have been a potential carbon source to use this pathway, the metabolic network allows for a more energy efficient growth by using the tpi reaction, as was indeed observed.

    To conclude, existing data supports predictions made by our model, given the requirement for stable steady state operation of autocatalytic cycles.
    This agreement between predictions and measurements is especially encouraging given the highly limited information on kinetic properties, concentrations, and fluxes under various growth conditions.

    \subsection{Allosteric regulation data matches predicted regulation in PTS-using cycle, but not in the glyoxylate cycle}
    We evaluate the expected allosteric interactions that support fast convergence and stability of autocatalytic cycles, as we qualitatively derived above, against the allosteric regulation network of the two functioning autocatalytic cycles we identified, the PTS-using autocatalytic cycle and the glyoxylate cycle (see supplementary tables, regulation data from \cite{Keseler2013-xn} and \cite{Schomburg2004-st}).

    For the PTS cycle we find that there are a total of 12 allosteric interactions, 7 inhibitions and 5 activations.
    Out of these 12 interactions, 11 interactions follow our expectations in terms of the class of the regulator, the class of the regulated enzyme, and the direction of the regulation (activation or inhibition).
    One interaction, the activation of fba by pep, does not follow our expectation.

    For the glyoxylate cycle we find that there are a total of 13 interactions, 12 inhibitions and 1 activation.
    Out of these 13 interactions, 8 interactions follow our expectation and 5 do not.
    The lack of agreement between the expected regulation direction and the actual regulation found for this cycle correlates with the observation in \cite{Gerosa2015-oq} that TCA cycle fluxes are regulated mainly by transcription and not by reactants levels.
 
    It is important to note that allosteric regulation serves many roles, and that the metabolic network faces many more challenges than just the support of stable autocatalysis.
    Therefore, the agreement we find between existing allosteric interactions and the expected regulation scheme supporting autocatalysis does not suggest that the autocatalytic nature of the PTS system is the underlying reason for these allosteric interactions.
    Nevertheless, these interactions support optimal autocatalytic function.
\section{Discussion}
Our study into the dynamics and stability of autocatalytic cycles suggests design principles applicable to both synthetic biology, in the effort to express novel heterologous cycles, and in the context of systems biology that aims to understand the function of natural networks.

Our autocatalytic modeling framework gives concrete predictions on saturation levels of branch reactions for operating autocatalytic cycles.
We find these predictions agree well with empirically measured fluxomics and proteomics data sets.
To rule out the possibility that other reasons (such as those reported in \cite{Staples1997-fq,Weiss1998-gp,Suarez1997-jo}) underlie the low saturation of branch reactions that we find, we compare the saturation levels of branch reactions versus their corresponding cycle reactions both under conditions when the autocatalytic cycle does not function, and for branch points out of non-autocatalytic cycles.
Both tests show no bias towards low saturation of branch reactions, contrary to the clear trend we find for reactions branching out of autocatalytic cycles.
Our findings thus support the notion that many, often differing causes can underlie seemingly wasteful expression of enzymes \cite{Salvador2003-fc, Salvador2006-yx}, and adds stability of intermediate metabolites of autocatalytic cycles to this list of possible causes.
The model we present can also highlight metabolic branch points at which the kinetic efficiency of enzymes is constrained due to stability requirements of a corresponding autocatalytic cycle.

A common concept in synthetic biology is that the successful implementation of novel pathways requires the expression of functional enzymes in the correct amounts in the target organism.
Here we show that in the context of autocatalytic cycles, such expression modulation may not suffice.
Specifically, changes to the substrate affinity of enzymes at branch points of the cycle may be required in order for the novel pathway to function.

Another aspect of our findings is that while it is common to assume that high affinity and catalytic rate are desirable traits for enzymes, such seeming improvements may actually lead to instability and thus to non functional metabolic cycles.
Furthermore, for reactions branching out of autocatalytic cycles, lower affinities increase the steady state concentration of intermediate metabolites, resulting in higher fluxes both through the cycle, and through the branch reaction, suggesting an unconventional strategy for optimizing fluxes through such reactions.
We show that allosteric regulation can potentially overcome the affinity restriction on branch reactions and find that in one out of the two native autocatalytic cycles that we find to function in vivo, known allosteric interactions agree with our predicted direction of interaction (activation of inhibition).

An experimental demonstration of these principles in-vivo is the recent implementation of a functional CBB cycle in \emph{E.coli} by introducing the two genes missing for its function \cite{Antonovsky2016-jy}.
The successful introduction of the genes did not suffice to make the cycle function, and further directed evolution was needed in order to achieve successful operation of the cycle.
Strikingly, most evolutionary changes occurred in branch points from the cycle.
The change which was biochemically characterized in the evolutionary process was the decrease of the value of $\sfrac{k_{\text{cat}}}{K_M}$ of phosphoribosylpyrophosphate synthetase (prs), one of the enzymes responsible for flux out of the CBB cycle, corresponding to the branch reaction in our simple model.
This is beautifully in line with the predictions of our analysis that suggest that decreasing $\sfrac{V_{\max}}{K_M}$ of branch reactions can lead to the existence of a stable flux solution.

Our observation regarding the stabilizing effect of input fluxes into an autocatalytic cycle can provide means to mitigate the stability issue in synthetic biology metabolic engineering setups.
In such setups, conducting directed evolution under gradually decreasing input fluxes allows for a pathway to gradually evolve towards sustainable, substantial flux.
This observation may also highlight the importance of seemingly redundant reactions such the one catalyzed by rpi, in growth on ribose.

While autocatalytic cycles are usually considered a small part of metabolism, we find that such cycles are at the heart of central carbon metabolism.
We use recent measurements to support our predictions on the usage of enzymes catalyzing reactions branching out of such cycles.
Our observations suggest that the constraints we find on the kinetic parameters of enzymes involved in autocatalytic cycles may limit the parameters of a broad set of enzymes central to metabolism.

Finally, while our work focuses on cycles increasing the amount of carbon in the system, we note that autocatalysis can be defined with respect to other quantities such as energy (e.g. ATP investment and production in glycolysis \cite{Teusink1998-he}), non-carbon atoms, reducing power, or other moieties \cite{Reich1981-qd}.
  Moreover, as autocatalysis is often studied with relation to the origin of life, our analysis may be useful in studying synthetic autocatalytic systems \cite{Semenov2016-ol} and 
  The analysis we present here can thus be of wide relevance for the design and analysis of metabolic networks.

  \section{Methods}
  \subsection{Formal definition of an autocatalytic metabolic cycle}
  Given a metabolic network composed of a set of reactions and metabolites, the following criteria can be used to define a subset of the network that is an autocatalytic cycle:
A set of irreversible reactions (for reversible reactions only one direction can be included in the set) and metabolites forms a cycle if every metabolite of the set can be converted, by sequential application of reactions in the set (where two reactions can be chained if a metabolite in the set is a product of the first reaction and a substrate of the second reaction), to every other metabolite in the set.
A cycle is autocatalytic if the reactions of the cycle can be applied, each reaction at an appropriate, positive number of times, such that the resulting change in the amount of each of the metabolites forming the cycle is non-negative, with at least one metabolite of the cycle having a strictly positive change.

The same definition can be stated in terms of reaction vectors and a stoichiometric matrix.
If a metabolic network has $n$ metabolites, indicated by the numbers $1$ to $n$, then every reaction, $R$, in the network can be described as a vector $V_R$ in $\mathbb{Z}^n$, such that the $i$'th coordinate of $V_R$ specifies how much of metabolite $i$ the reaction $R$ produces (if $R$ consumes a metabolite, then the value at the coordinate representing the metabolite is negative).

With this notation, a set of metabolites: $M'={m_1\cdots m_j}$ and a set of reactions, $R'={r_1\cdots r_k}$ form an autocatalytic cycle if:
\begin{enumerate}
    \item Every metabolite in $M'$ is a substrate of at least one reaction in $R'$, and a product of at least one reaction in $R'$.
    \item Every reaction in $R'$ has at least one substrate from $M'$ and at least one product from $M'$.
    \item There is a set of positive integers, $i_1\cdots i_k$ such that the total reaction vector $R^*=\sum_{l=1}^k i_lR_l$ is non negative at all the coordinates $m_1\cdots m_j$ and is strictly positive for at least one coordinate in this range.
\end{enumerate}

  \subsection{Systematic identification of autocatalytic cycles in metabolic networks}
  We implemented an algorithm to systematically search for autocatalytic cycles in metabolic networks.
  The algorithm is incomplete, meaning there may be autocatalytic cycles that will not be identified by it.
  Further work will enable a more robust algorithm to identify additional autocatalytic cycles in full metabolic networks.
  We used the algorithm on the core carbon metabolism network of \emph{E.coli} \cite{Orth2010-vl}.

  In our framework, a metabolic network is defined by a set of reactions, $(\bar{R})$.
  Each reaction is defined by a set of substrates and a set of products, with corresponding stoichiometries $R_i=(S,P,N^S,N^P)$, such that $R_i$ describes the reaction $\sum_j N_j^SS_j \rightarrow \sum_k N_k^PP_k$.
  The algorithm works as follows:
  \begin{enumerate}
    \item All co-factors are removed from the description of the metabolic network, so that it is carbon-based.
    \item The metabolic network is converted to a directed graph, G:
  The nodes of G are all the metabolites and all the reactions of the network.
  For each metabolite, $M$, and each reaction, $R$, if $M$ is a substrate of $R$ then the edge $(M,R)$ is added to the graph, and if $M$ is a product of $R$, then the edge $(R,M)$ is added to the graph.
    \item The Tarjan cycle identification algorithm is used to enumerate all the cycles in the graph \cite{Tarjan1973-gn}.
    \item for every cycle identified by the Tarjan algorithm, $C$, the algorithm checks if the cycle can be the backbone of an autocatalytic cycle as follows:
        \begin{enumerate}
            \item For every reaction in the cycle, $R$, the algorithm checks if it consumes more than one  intermediate metabolite of the cycle.
              If so, $C$ is assumed not to be autocatalytic and the algorithm continues to evaluate the next cycle.
            \item Otherwise, for every reaction in the cycle, $R$, the algorithm checks if it has more than one product that is an intermediate metabolite of the cycle.
              If so, then the algorithm lists $C$ as an autocatalytic cycle.
            \item Finally, the algorithm checks, for every reaction in the cycle, if it has a product that is not an intermediate metabolite of the cycle.
                If so, denote by $M_E$ such a metabolite.
            The algorithm proceeds to check if, for every intermediate metabolite of the cycle, $M_i$ a reaction exists from $M_E$ to $M_i$ that does not use any of the reactions of the cycle, and does not consume any of the intermediate metabolites of the cycle.
            If so then the algorithm lists $C$ as an autocatalytic cycle.
        \end{enumerate}
  \end{enumerate}

  The algorithm assumes reactions consume exactly one molecule of any of their substrates and produce exactly one molecule of any of their products, an assumption that holds for the core model of \emph{E.coli}, but not in metabolic networks in general.
  
  \subsection{Dependence of the simple cycle steady state concentration on the kinetic parameters of the autocatalytic and branch reactions}
  The simple cycle steady state concentration, $X^*$, is given in equation \ref{eq:xstst}.
  Taking the derivative of this expression with respect to $K_{M,a}$, $K_{M,b}$, $V_{\max,a}$, and $V_{\max,b}$, under the assumption that the kinetic parameters satisfy the stability conditions in equation \ref{eq:stabconds} gives:
  \begin{align}
      \begin{split}
          \label{eq:ststdep}
    \frac{\partial X^*}{\partial K_{M,a}}&=\frac{V_{\max,b}}{V_{\max,a}-V_{\max,b}}<0 \\
    \frac{\partial X^*}{\partial K_{M,b}}&=\frac{-V_{\max,a}}{V_{\max,a}-V_{\max,b}}>0 \\
    \frac{\partial X^*}{\partial V_{\max,a}}&=\frac{V_{\max,b}(K_{M,b}-K_{M,a})}{(V_{\max,a}-V_{\max,b})^2}>0 \\
    \frac{\partial X^*}{\partial V_{\max,b}}&=\frac{V_{\max,a}(K_{M,a}-K_{M,b})}{(V_{\max,a}-V_{\max,b})^2}<0
      \end{split}
  \end{align}
  So that $X^*$ increases when $K_{M,a}$ decreases or $V_{\max,a}$ increases (or both) corresponding to activation of $f_a$.
  On the other hand, $X^*$ decreases when $K_{M,b}$ decreases or $V_{\max,b}$ increases (or both) corresponding to activation of $f_b$.

  \subsection{Connecting the kinetic constants of bisubstrate reaction velocity equations, under constant substrate concentration assumption, with simple Michaelis-Menten kinetics constants}
  Two standard equations are commonly used to describe the flux through irreversible bisubstrate reactions \cite{Leskovac2003-li}.
  We show that, under the assumption that the assimilated metabolite maintains constant concentration, these equations reduce to simple Michaelis-Menten equations.
  We derive the expressions for the apparent Michaelis-Menten constants, $K_M$ and $V_{\max}$, as functions of the kinetic constants of the bisubstrate reaction and the concentration of the assimilated metabolite.
  While the substrates in these equations are generally denoted as $A$ and $B$, here, to keep the notation consistent, we will denote by $A$ the assimilated metabolite and by $X$ the internal metabolite of the cycle.

  The first equation describing a bisubstrate reaction assumes random binding order of the two substrates.
  As this equation is symmetric with respect to the two substrates, we can arbitrarily decide which of the two substrates is the assimilated metabolite, and which is the internal metabolite.
  We get that the flux through the reaction is:
  \begin{equation*}
      f=\frac{V_{\max}'AX}{K_{i,A}K_X+K_XA+K_AX+AX}
  \end{equation*}
  Rearranging to get the dependence of the flux on $X$ in a Michaelis-Menten like form we get that:
  \begin{equation*}
      f=\frac{\frac{V_{\max}'A}{K_A+A}X}{\frac{K_{i,A}+A}{K_A+A}K_X+X}
  \end{equation*}
  which gives apparent Michaelis-Menten kinetic constants of:
  \begin{align}
  \begin{split}
      \label{eq:paramssym}
      V_{max}&=\frac{V_{\max}'A}{K_A+A} \\
      K_M&=\frac{K_{i,A}+A}{K_A+A}K_X
  \end{split}
  \end{align}

  The second equation describing a bisubstrate reaction assumes ordered binding of the substrates.
  Because in ordered binding the equation is asymmetric with respect to the two substrates, analyzing this reaction is further split according to which of the two substrates is assumed to be the assimilated metabolite with constant concentration.

  If the first binding metabolite is assumed to be the assimilated metabolite we get that:
  \begin{equation*}
      f=\frac{V_{\max}'AX}{K_{i,A}K_X+K_XA+AX}=\frac{V_{\max}'X}{\frac{K_{i,A}+A}{A}K_X+X}
  \end{equation*}
  which gives apparent Michaelis-Menten kinetic constants of:
  \begin{align}
  \begin{split}
      \label{eq:paramsasymfst}
      V_{max}&=V_{\max}' \\
      K_M&=\frac{K_{i,A}+A}{A}K_X
  \end{split}
  \end{align}
  
  If the first binding metabolite is assumed to be the internal metabolite we get that:
  \begin{equation*}
      f=\frac{V_{\max}'AX}{K_{i,X}K_A+K_AX+AX}=\frac{\frac{V_{\max}'A}{K_A+A}X}{\frac{K_{i,X}K_A}{K_A+A}+X}
  \end{equation*}
  which gives apparent Michaelis-Menten kinetic constants of:
  \begin{align}
  \begin{split}
      \label{eq:paramsasymsnd}
      V_{max}&=\frac{V_{\max}'A}{K_A+A} \\
      K_M&=\frac{K_{i,X}K_A}{K_A+A}
  \end{split}
  \end{align}

  To summarize, the most common equations describing bisubstrate reactions reduce to equations of the same form as Michaelis-Menten equations, under the assumption that one of the metabolites maintains a constant concentration.
  The apparent kinetic constants of the Michaelis-Menten equivalent equations depend on the kinetic constants of the bisubstrate reactions, as well as on the concentration of the assimilated metabolite.


  \subsection{Mapping stability constraints on Michaelis-Menten kinetic constants to constraints on the concentration of the assimilated metabolite and the kinetic constants of bisubstrate reactions}
  In equation \ref{eq:stabconds} we obtain constraints on the kinetic parameters of Michaelis-Menten reactions that ensure existence and stability of a positive steady state.
  We observe that these constraints imply that even if the maximal rates of the two reactions can be easily modified, if $K_{M,b}<K_{M,a}$ then such changes cannot suffice in order to satisfy the existence and stability constraints.
  
  Here we map the same constraints from equation \ref{eq:stabconds} onto bisubstrate autocatalytic reactions.
  This mapping results in constraints on the assimilated metabolite concentration, as well as on the kinetic parameters of the bisubstrate autocatalytic reactions.
  We show that in all bisubstrate reaction schemes, there is a lower bound on the concentration of the assimilated metabolite, below which the system cannot attain a stable positive steady state.
  We further show that the nature of the bisubstrate reaction qualitatively affects the ability to satisfy the stability constraints by changing expression levels alone.
  In the case of random binding order and in the case of ordered binding, with the assimilated metabolite binding first, unless the kinetic parameters of the participating enzymes satisfy specific inequalities, changes to the maximal reaction rates alone cannot suffice in order to satisfy the existence and stability constraints.
  However, in the case of ordered binding with the internal metabolite binding first, changes to the maximal reaction rates alone suffice in order to allow for stable steady state to occur, given high enough concentration of the assimilated metabolite.
  We analyze each of the three possible bisubstrate reaction schemes separately below.

  \paragraph{Random binding order}
  In the case of random binding order, we can substitute equation \ref{eq:paramssym} into the conditions from equation \ref{eq:stabconds} to get:
  \begin{equation}
    \label{eq:stabcondsrandbind}
    \begin{dcases}
      & V_{\max,b}>\frac{V_{\max}'A}{K_A+A}=V_{\max}'\frac{A}{K_A+A} \\
      & \frac{V_{\max,b}}{K_{M,b}}<\frac{V_{\max}'A}{(K_{i,A}+A)K_X}=\frac{V_{\max}'\frac{A}{K_{i,A}+A}}{K_X}
    \end{dcases}
  \end{equation}

  We first note that from the second inequality we get that:
  \begin{equation}
    \label{eq:assimlowerbound}
      \frac{V_{\max,b}K_X}{K_{M,b}V_{\max}'}<\frac{A}{K_{i,A}+A} \rightarrow
      \frac{K_{i,A}}{\frac{K_{M,b}V_{\max}'}{V_{\max,b}K_X}-1}<A
  \end{equation}
  Giving a lower bound on the concentration of the assimilated metabolite for which a stable steady state is attainable.

  We now wish to obtain a lower bound on $K_{M,b}$.
  In order to obtain such a lower bound, we need an upper bound on $V_{\max}'\frac{A}{K_{i,A}+A}$.
  However, we only have an upper bound on $V_{\max}'\frac{A}{K_A+A}$.
  We use the first inequality in equation \ref{eq:stabcondsrandbind} to get that:
  \begin{equation*}
      \begin{split}
      V_{\max,b}&>V_{\max}'\frac{A}{K_A+A} \rightarrow \\
      V_{\max,b}\frac{A}{K_{i,A}+A}&>V_{\max}'\frac{A}{K_A+A}\frac{A}{K_{i,A}+A} \rightarrow \\
      V_{\max,b}\frac{K_A+A}{K_{i,A}+A}&>V_{\max}'\frac{A}{K_{i,A}+A}
      \end{split}
  \end{equation*}
  We note that for positive $A$, $\frac{K_A+A}{K_{i,A}+A} < \max(1,K_A/K_{i,A})$ and therefore:
  \begin{equation*}
      V_{\max,b}\max(1,K_A/K_{i,A})>V_{\max}'\frac{A}{K_{i,A}+A}
  \end{equation*}
  Substituting this inequality in the second inequality of equation \ref{eq:stabcondsrandbind} therefore gives us that:
  \begin{equation*}
    \begin{split}
      \frac{V_{\max,b}}{K_{M,b}}<\frac{V_{\max}'\frac{A}{K_{i,A}+A}}{K_X}&<\frac{V_{\max,b}\max(1,K_A/K_{i,A})}{K_X} \rightarrow\\
      \frac{K_X}{\max(1,K_A/K_{i,A})}&<K_{M,b}
    \end{split}
  \end{equation*}
  We have therefore obtained a lower bound on the affinity of the branch reaction, $K_{M,b}$, in this case.

  For the random binding order we can thus conclude that, like in the single-substrate case, a lower bound exists on the affinity of the branch reaction, below which a positive steady state is not attainable, even if the expression levels of the enzymes, and the concentration of the assimilated metabolite are modified.
  Furthermore, for any set of kinetic parameters, there is a lower bound on the concentration of $A$ for which a positive steady state is attainable.

  \paragraph{Ordered binding with the assimilated metabolite binding first}
  In the case of ordered binding, with the assimilated metabolite binding first, we can substitute equation \ref{eq:paramsasymfst} into the conditions from equation \ref{eq:stabconds} to get:
  \begin{equation}
    \label{eq:stabcondsordbindfst}
    \begin{dcases}
      & V_{\max,b}>V_{\max}' \\
      & \frac{V_{\max,b}}{K_{M,b}}<\frac{V_{\max}'A}{(K_{i,A}+A)K_X}=\frac{V_{\max}'\frac{A}{K_{i,A}+A}}{K_X}
    \end{dcases}
  \end{equation}
  As the second inequality is identical to the one in the random binding order case, we can immediately conclude that the same lower bound on the concentration of $A$ from equation \ref{eq:assimlowerbound} holds in this case as well.

  Regarding a lower bound on $K_{M,b}$, following similar reasoning as in the previous case, we first note that for any value of $A$: 
  \begin{equation*}
    \frac{A}{K_{i,A}+A}<1
  \end{equation*}
  Therefore we can deduce, by using the first inequality from equation \ref{eq:stabcondsordbindfst} in the second inequality from that equation, that:
  \begin{equation*}
    \frac{V_{\max,b}}{K_{M,b}}<\frac{V_{\max}'\frac{A}{K_{i,A}+A}}{K_X}<\frac{V_{\max}'}{K_X}<\frac{V_{\max,b}}{K_X}
  \end{equation*}
  which immediately yields:
  \begin{equation*}
    K_X<K_{M,b}
  \end{equation*}
  setting an absolute lower bound on $K_{M,b}$.

  We thus arrive at the same conclusions in this case, as we have arrived to in the previous case, namely that a lower bound exists on the affinity of the branch reaction, and that, for any set of kinetic parameters, there is a lower bound on the concentration of the assimilated metabolite, below which a positive stable steady state cannot be obtained.

  \paragraph{Ordered binding with the internal metabolite binding first}
  In the case of ordered binding, with the internal metabolite binding first, we can substitute equation \ref{eq:paramsasymsnd} into the conditions from equation \ref{eq:stabconds} to get:
  \begin{equation}
    \label{eq:stabcondsordbindsnd}
    \begin{dcases}
      & V_{\max,b}>\frac{V_{\max}'A}{K_A+A} \\
      & \frac{V_{\max,b}}{K_{M,b}}<\frac{V_{\max}'A}{K_{i,X}K_A}
    \end{dcases}
  \end{equation}
  As in the previous two cases, the second inequality can be used to obtain a lower bound on the concentration of $A$:
  \begin{equation*}
    \frac{V_{\max,b}K_{i,X}K_A}{K_{M,b}V_{\max}'}<A
  \end{equation*}
  However, unlike in the previous two cases, in this case if $V_{\max}'<V_{\max,b}$, then the first inequality in equation \ref{eq:stabcondsordbindsnd} holds for any concentration of $A$, and, for any concentration of $A$ that is larger than its lower bound, the second inequality is also satisfied, resulting in a stable steady state.
  This case is therefore more robust than the other two cases as it allows for the conditions to be satisfied, at least for high concentrations of $A$, given any set of kinetic parameters.
  \subsection{Dependence of steady state concentration on the concentration of the assimilated metabolite}
  Equation \ref{eq:xstst} shows the dependency between the concentration of $X$, the internal metabolite, and the kinetic parameters of the reactions in the system.
  Substituting the dependencies of the apparent kinetic parameters from equations \ref{eq:paramssym}, \ref{eq:paramsasymfst}, and \ref{eq:paramsasymsnd} into equation \ref{eq:xstst} gives the dependency of $X$ on the kinetic parameters of the bisubstrate reactions and the concentration of the assimilated metabolite, $A$.
  We get for these 3 cases respectively that:

  \begin{align}
      X &= \frac{\frac{V_{\max}'A}{K_A+A}K_{M,b}-V_{\max,b}\frac{K_{i,A}+A}{K_A+A}K_X}
      {V_{\max,b}-\frac{V_{\max}'A}{K_A+A}} \\
      X &= \frac{V_{\max}'K_{M,b}-V_{\max,b}\frac{K_{i,A}+A}{A}K_X}
      {V_{\max,b}-V_{\max}'} \\
      X &= \frac{\frac{V_{\max}'A}{K_A+A}K_{M,b}-V_{\max,b}\frac{K_{i,X}K_A}{K_A+A} }
      { V_{\max,b}-\frac{V_{\max}'A}{K_A+A} }
  \end{align}

  Assuming the kinetic parameters satisfy the stable steady state conditions derived in equations \ref{eq:stabcondsrandbind}, \ref{eq:stabcondsordbindfst}, and \ref{eq:stabcondsordbindsnd}, we note that when $A$ is equal to its lower bound, the numerator in all three cases is $0$, resulting in $X=0$.
  Furthermore, as $A$ decreases towards its lower bound, $X$ decreases resulting in a decrease in both $f_b$ and $f_a$ (for the two latter cases this is trivial to show, as the terms involving $A$ increase and decrease monotonically in accordance with their effect on $X$. In the first case, taking the derivative of the numerator w.r.t. $A$ shows the derivative is always positive, resulting in the same conclusion).
  Interestingly, in the first and last cases, if $V_{\max}'>V_{\max,b}$, then an upper bound on the concentration of $A$ also exists.
  As the concentration of $A$ approaches this upper bound, the denominator approaches $0$ resulting in an increase in the concentration of $X$ towards infinity.

  \subsection{Limits on derivatives of branch reactions for complex autocatalytic cycles}
  Stability analysis of a model complex autocatalytic cycle with $n$ reactions in the cycle results in the following Jacobian matrix:

  \begin{equation}
    \label{eq:jacobian}
    J=
    \begin{pmatrix}
        -(\alpha_1+\beta_1) & 0 & \cdots & 0 & 2\alpha_n \\
        \alpha_1 & -(\alpha_2+\beta_2) & \cdots & 0 & 0 \\
        \vdots & \vdots & \ddots & \vdots & \vdots \\
        0 & 0 & \cdots & -(\alpha_{n-1}+\beta_{n-1}) & 0\\
        0 & 0 & \cdots & \alpha_{n-1} & -(\alpha_n+\beta_n)
     \end{pmatrix}
  \end{equation}
   The characteristic polynomial of this matrix is given by:

    \begin{equation}
        \label{eq:complex}
        0=\prod_{i=1}^n(\lambda+\alpha_i+\beta_i)-2\prod_{i=1}^n\alpha_i
    \end{equation}

  To extract the conditions under which all the roots of the characteristic polynomial have negative real parts we use Rouche's theorem.
  Our strategy will be as follows:
  We will define a contour that contains only numbers with negative real parts.
  We will show that all the roots of the polynomial $0=\prod_{i=1}^n(\lambda+\alpha_i+\beta_i)$ lie within the area this contour encloses.
  We will find the conditions for which  $|\prod_{i=1}^n(\lambda+\alpha_i+\beta_i)|>2\prod_{i=1}^n\alpha_i$ on the contour, satisfying the premise of Rouche's theorem.
  We will then claim that under these conditions all the roots of the polynomial in Equation \ref{eq:complex} must also lie inside the contour, and therefore must have negative real parts.
  Given that all the roots of this polynomial have negative real parts, we will conclude that the eigenvalues of the Jacobian matrix at the steady state point all have negative real parts, making the steady state point stable.

  \paragraph{Proof:} We pick a large parameter $R$, such that $R>3\max_j(\alpha_j+\beta_j)$.
  We look at the closed half circle contour, $K$, composed of the segment $[(0,-iR),(0,iR)]$ and the half circle arc $(x,iy)$ such that $x\leq 0$ and $x^2+y^2=R^2$.
  We define 

  \begin{equation*}
      g(\lambda)=2\prod_{j=1}^n\alpha_j
  \end{equation*}
  noting that it is constant over all of $\mathbb{C}$ and specifically over $K$.
  We define

  \begin{equation*}
      f(\lambda)=\prod_{j=1}^n(\lambda+\alpha_j+\beta_j)
  \end{equation*}
  noting that all of $f$'s roots lie inside $K$ as the roots are $0>-(\alpha_j+\beta_j)>-R$ for all $j$.
  We check the conditions under which $|f(\lambda)|>|g(\lambda)|$ over the contour $K$.

  For the arc segment we note that, as for complex numbers it holds that $|xy|=|x||y|$, then

  \begin{equation*}
      |f(\lambda)|=\prod_{j=1}^n|\lambda+\alpha_j+\beta_j|
  \end{equation*}
  From the triangle inequality we know that $|x+y| \geq |x|-|y|$ and therefore for all $j$ it holds that

  \begin{equation*}
      |\lambda+\alpha_j+\beta_j| \geq R-(\alpha_j+\beta_j)
  \end{equation*}
  As we picked $R$ such that $R>3\max_j(\alpha_j+\beta_j)$ we get that 

  \begin{equation*}
      R-(\alpha_j+\beta_j)>2(\alpha_j+\beta_j)
  \end{equation*}
  and therefore 

  \begin{equation*}
      \prod_{j=1}^n|\lambda+\alpha_j+\beta_j|>\prod_{j=1}^n2|\alpha_j+\beta_j|>\prod_{j=1}^n2|\alpha_j| = |g(\lambda)|
  \end{equation*}
  concluding that over the arc, $|f(\lambda)|>|g(\lambda)|$.

  For the part of $K$ on the imaginary axis we note that $\lambda=iy$ where $y\in[-R,R]$.
For this segment we therefore get that

\begin{equation*}
    |f(\lambda)|=\prod_{j=1}^n|\alpha_j+\beta_j+iy|=\sqrt{\prod_{j=1}^n((\alpha_j+\beta_j)^2+y^2)} \geq \sqrt{\prod_{j=1}^n(\alpha_j+\beta_j)^2}
 \end{equation*} 
 and, as before, that

 \begin{equation*}
     |g(\lambda)|=2\sqrt{\prod_{j=1}^n\alpha_j^2}
 \end{equation*}
 To meet the condition that $|f(\lambda)|>|g(\lambda)|$, which is equivalent to: $\frac{|f(\lambda)|}{|g(\lambda)|}>1$, it is sufficient to find the conditions under which:

 \begin{equation*}
   \frac{\sqrt{\prod_{j=1}^n(\alpha_j+\beta_j)^2}}{2\sqrt{\prod_{j=1}^n\alpha_j^2}}>1
 \end{equation*}
 Simplifying this inequality gives:

 \begin{equation*}
     \frac{1}{2}\sqrt{\prod_{j=1}^n\frac{(\alpha_j+\beta_j)^2}{\alpha_j^2}}= \frac{1}{2}\prod_{j=1}^n\frac{\alpha_j+\beta_j}{\alpha_j}= \frac{1}{2}\prod_{j=1}^n(1+\frac{\beta_j}{\alpha_j})>1 \Rightarrow \prod_{j=1}^n(1+\frac{\beta_j}{\alpha_j})>2
 \end{equation*}
 A sufficient condition to satisfy this inequality, given that all the $\alpha_j$ are positive and all the $\beta_j$'s are non negative, is that there exists $j$ such that $\beta_j > \alpha_j$.

 We therefore get that if there exists $j$ such that  $\beta_j > \alpha_j$, then $|f(\lambda)|>|g(\lambda)|$ over the contour $K$.
 In this case, by Rouche's theorem, we deduce that, as all of $f$'s roots lie inside $K$, then it follows that all of $f-g$'s roots lie inside $K$, concluding that the real part of all of the eigenvalues of the characteristic polynomial of the Jacobian matrix of the complex autocatalytic cycle are negative, making any steady state that meets this criterion stable.

 \subsection{Multiple unsaturated branch reactions increase convergence speed and dampen oscillations}
 Using the Jacobian matrix from equation \ref{eq:jacobian} we can analyze the effect of multiple low saturation branch points on convergence to steady state.
 The analysis shows that the more $i$'s exist for which $\beta_i>0$, and the larger $\beta_i$ is (resulting in lower saturation of $f_{b_i}$), the faster the convergence of the cycle to steady state will be.

 We denote by $\vec{X^*}$ the steady state vector of the concentrations of the intermediate metabolites.
 We denote by $\vec{X}=\vec{X^*}+\Delta X_j$ a state where for all the intermediate metabolites that are not $X_j$, their concentration is the same as the steady state concentration, and $X_j$ differs by a small amount, $\Delta X_j$, from its steady state concentration.
 We let $F$ denote the fluxes function of the system such that $F(X)=\dot{X}\vert_{\vec{X}}$.
 Evaluating the dynamics of the system at $\vec{X}$ by noting that $F(\vec{X})\approx F(\vec{X^*})+J*\Delta X_j=J*\Delta X_j$ (where $F(\vec{X^*})=0$ as $\vec{X^*}$ is a steady state) results in $F(\vec{X})_k=0$ for all $k\neq j,j+1$.
 For $X_j$ such that $j\neq n$ we get $F(\vec{X})_j\approx -(\alpha_j+\beta_j)\Delta X_j$ and for $X_{j+1}$ we get $F(\vec{X})_{j+1}\approx \alpha_{j}\Delta X_j$.
 Therefore, the difference from the steady state decreases proportionally to $\beta_j$ (and cycles to the next intermediate metabolite, $X_{j+1}$).
 For $j=n$, we get that $F(\vec{X})_j\approx -(\alpha_j+\beta_j)\Delta X_j$, as for $j\neq n$, but $F(\vec{X})_{1}\approx 2\alpha_{j}\Delta X_j$ where the factor of $2$ is due to the effect of the assimilating reaction, that causes an amplification of the deviation from steady state (an amplification that is dampened by subsequent reactions along the cycle if the conditions for stable steady state are satisfied).

 It therefore follows that any increase in $\beta_j$, for any $j$, increases the speed of convergence to steady state and reduces the propagation of deviations from steady state for $X_j$.
 Thus, to keep deviations from steady state at check, it is beneficial to increase $b_j$, for all $j$, which implies decreasing the saturation of $f_{b_j}$.


    \subsection{The relationship between derivatives in Michaelis-Menten kinetics dictates an inverse relationship between affinities and saturation levels}
    It turns out that for the Michaelis-Menten kinetics equations, the following useful lemma can be used to connect theoretical observations on the relationships of derivatives to physiological observations on affinities and saturation levels.

    We define the saturation level of a reaction as the ratio between the flux it carries, and the maximal flux it can carry, given the expression level of the relevant enzyme, that is:

    \begin{equation*}
      S(X)=\frac{f(X)}{V_{\max}}=\frac{X}{K_M+X}
    \end{equation*}

    Given this definition we can show that if two Michaelis-Menten reactions consume the same metabolite, $X$, and at a given concentration, $X^*$, it holds that $f_a(X^*)\geq f_b(X^*)$, then if:

    \begin{equation}
        \frac{df_b}{dX}\Big\vert_{X=X^*}>\frac{df_a}{dX}\Big\vert_{X=X^*}
        \label{eq:lemmapremise}
    \end{equation}
    then it follows that:

    \begin{equation*}
    \begin{dcases}
      & K_{M,b}>K_{M,a} \\
      & S_b(X^*)<S_a(X^*)
    \end{dcases}
    \end{equation*}
    Proof: expanding the condition that $f_a(X^*)\geq f_b(X^*)$, we get that:

    \begin{equation}
        \frac{V_{\max,b}X^*}{K_{M,b}+X^*}\leq\frac{V_{\max,a}X^*}{K_{M,a}+X^*} \Rightarrow
         \frac{V_{\max,b}}{K_{M,b}+X^*}\leq\frac{V_{\max,a}}{K_{M,a}+X^*}
         \label{eq:firstmid}
     \end{equation}
     Expanding the premise of the lemma in equation \ref{eq:lemmapremise} gives us that:

     \begin{equation*}
        \frac{df_b}{dX}\Big\vert_{X=X^*}>\frac{df_a}{dX}\Big\vert_{X=X^*} \Rightarrow
        \frac{V_{\max,b}K_{M,b}}{(K_{M,b}+X^*)^2} > \frac{V_{\max,a}K_{M,a}}{(K_{M,a}+X^*)^2}
     \end{equation*}
     Because equation \ref{eq:firstmid} holds, it follows that:

     \begin{equation*}
         \frac{K_{M,b}}{K_{M,b}+X^*} > \frac{K_{M,a}}{K_{M,a}+X^*} \Rightarrow \frac{1}{1+\frac{X^*}{K_{M,b}}} > \frac{1}{1+\frac{X^*}{K_{M,a}}} \Rightarrow K_{M,b}>K_{M,a}
     \end{equation*}
    setting the affinity of the autocatalytic enzyme as a lower bound for the affinity of the branch enzyme.
    Finally, given this relation of affinities it follows that:

    \begin{equation*}
        K_{M,b}>K_{M,a} \Rightarrow X^*+K_{M,b}>X^*+K_{M,a} \Rightarrow \frac{X^*}{X^*+K_{M,b}}<\frac{X^*}{X^*+K_{M,a} } \Rightarrow S_b(X^*)<S_a(X^*)
    \end{equation*}
    concluding the proof.

    We note that a multiple reaction autocatalytic cycle at a stable steady state point satisfies equations \ref{eq:complexstst} and \ref{eq:complexstab}, so the lemma applies.

    \subsection{Evaluating maximal flux capacity of reactions under a given condition}
    To evaluate the maximal flux capacity of a reaction under a given growth condition, given expression level and flux data for a set of conditions, we follow the procedure described in \cite{Davidi2016-ga}.
    For each reaction, under every condition, we divide the flux the reaction carries (obtained from \cite{Gerosa2015-oq}) by the amount of the corresponding enzyme expressed under that condition (obtained from \cite{Schmidt2015}).
    We thus get a flux per enzyme estimate for the given reaction under each of the conditions.
    We define the enzyme maximal in-vivo catalytic rate as the maximum flux per unit enzyme it carries across all conditions (noting that this is actually only a lower bound on this rate).
    Multiplying the enzyme maximal catalytic rate by the enzyme amount at each condition therefore results in an estimate of the maximal possible flux through the given reaction under the relevant condition.
